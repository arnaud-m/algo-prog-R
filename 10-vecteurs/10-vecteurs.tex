\documentclass[10pt]{beamer}

\usepackage{../macros}
\title{Les séquences : vecteur}

\hypersetup{
  pdftitle =  {Les séquences : vecteur}
}

\begin{document}

\maketitle

%%%%%%%%%%%%%%%%%%%%%%%%%%%%%%%%%%%%%%%%%%%%%%%%%%%%%%%
\begin{frame}[fragile]
  \frametitle{Qu'est-ce qu'une séquence ?}

  \begin{alertblock}{Séquence}
    Une séquence est une suite finie de valeurs numérotées, distinctes ou pas, de n'importe quel type.   
  \end{alertblock}

 
\begin{columns}[t]
\begin{column}{0.48\textwidth}
 \begin{block}{Vecteur}
    Un vecteur est une séquence d'éléments \alert{du même type}.
    \begin{lstlisting}
 > 1:6
[1] 1 2 3 4 5 6
> c('foo', 'bar')
[1] "foo" "bar"     
    \end{lstlisting}
  \end{block}
\end{column}
\begin{column}{0.48\textwidth}
   \begin{block}{Liste}
     Une liste est une séquence d'éléments \alert{de type quelquonque}.
     \begin{lstlisting}
> list(1:6, c('foo', 'bar'))
[[1]]
[1] 1 2 3 4 5 6

[[2]]
[1] "foo" "bar"       
     \end{lstlisting}
  \end{block}
\end{column}
\end{columns}

  \begin{block}{Mutabilité}
    Les listes et vecteurs sont mutables, i.e. on peut les modifier.    
  \end{block}

  \begin{center}
      Occupons nous d’abord des vecteurs.
  \end{center}

\end{frame}


%%%%%%%%%%%%%%%%%%%%%%%%%%%%%%%%%%%%%%%%%%%%%%%%%%%%%%%
\begin{frame}[fragile]
  \frametitle{De l’utilité des vecteurs \dots}
  Programmons la fonction \texttt{Diviseurs} qui retourne la liste des diviseurs non triviaux (distincts de 1 et n) d'un entier $\mathtt{n} \geq 2$. \\
  Ne connaissant pas à l'avance le nombre d'éléments du résultat, on utilise un vecteur, initialement vide, qu'on aggrandit au fur et à mesure.
  \begin{lstlisting}[style=editor]
Diviseurs <- function(n) {
  res <- numeric(0);
  for(d in 2:(n-1)) {
    if( n %% d == 0) res <- append(res, d);
  }
  return(res)
}    
\end{lstlisting}

\begin{description}[\texttt{numeric}]
\item[\texttt{numeric}] crée un vecteur numérique. 
\item[\texttt{append}] ajoute un élément en queue d'un vecteur.
\end{description}
\begin{lstlisting}
> Diviseurs(1000)
 [1]   2   4   5   8  10  20  25  40  50 100 125 200 250 500  
\end{lstlisting}
\end{frame}

%%%%%%%%%%%%%%%%%%%%%%%%%%%%%%%%%%%%%%%%%%%%%%%%%%%%%%%
\begin{frame}[fragile]
  \frametitle{Accès par rang aux éléments d'une séquence}
  L'accès par rang est commun aux vecteurs, listes, matrices, data frames.\\
  \begin{itemize}
  \item \alert{La notation indexée par crochets. Les indices commencent à 1.}
  \item R supprime les éléments quand leurs indices sont négatifs.
  \item R vérifie la validité des arguments
  \end{itemize}

  
  \begin{lstlisting}
> vec <- 1:5
> vec[1] # Les indices commencent à 1.
[1] 1
> vec[2]
[1] 2
> vec[c(1,5)] #  accéder simultanément à plusieurs éléments.
[1] 1 5
> vec[-3] # suppression d'éléments avec un index négatif
[1] 1 2 4 5
> vec[0] # Les indices commencent à 1 !
integer(0)
> vec[6] # valeur spéciale pour les éléments manquants.
[1] NA   
\end{lstlisting}
\end{frame}

%%%%%%%%%%%%%%%%%%%%%%%%%%%%%%%%%%%%%%%%%%%%%%%%%%%%%%%
\begin{frame}[fragile]
  \frametitle{Accès par contenu}
  Dans de nombreux langages, la fonction \alert{\texttt{index}} permet de connaître le rang d'un élément dans une séquence donnée. \\
  Elle n'existe pas en R et nous allons donc la programmer !
  
\begin{columns}[t]
\begin{column}{0.50\textwidth}
\begin{lstlisting}[style=editor]
index <- function(x, vec) {
  for(i in seq_along(vec)) {
    if( vec[i] == x) return(i)
  }
  return(NA)
}
\end{lstlisting}
\end{column}
\begin{column}{0.5\textwidth}
\begin{lstlisting}
> index(2, 1:5)
[1] 2
> index(6, 1:5)
[1] NA
>index("foo",c("foo","bar")))
[1] 1
\end{lstlisting}
\end{column}
\end{columns}

La fonction \texttt{seq\_along} renvoie le vecteur des indices du vecteur \texttt{vec} !

\begin{block}{Défi}
  Programmer une fonction \texttt{positions(x,vec)} retournant le vecteur des indices \texttt{i} tels que \texttt{vec[i] == x}.
\end{block}

\begin{alertblock}{Autres fonctions d'accès par contenu}
  R ne dispose pas d’une fonction index , mais des fonctions plus puissantes et complexes : \alert{\texttt{match}} et \alert{\texttt{which}}.
\end{alertblock}
\end{frame}
%%%%%%%%%%%%%%%%%%%%%%%%%%%%%%%%%%%%%%%%%%%%%%%%%%%%%%%
\begin{frame}[fragile]
  \frametitle{Arithmétique des vecteurs}
  \begin{block}{Opérations membre à membre}
    L'opération est appliquée sur les paires d'éléments à la même position.

\begin{columns}[t]
\begin{column}{0.48\textwidth}
    \begin{lstlisting}
> 1:3 + 3:1
[1] 4 4 4
> 1:3 * 3:1
[1] 3 4 3
    \end{lstlisting}
\end{column}
\begin{column}{0.48\textwidth}
    \begin{lstlisting}
> 1:3 / 1:3
[1] 1 1 1
> (1:3) ** (3:1)
[1] 1 4 3
    \end{lstlisting}
\end{column}
\end{columns}
  \end{block}

  \begin{block}{Recyclage des éléments}
    Si deux vecteurs ont des longueurs différentes, les éléments du plus court sont recyclés.
    \begin{lstlisting}
> 1:4 + 1:2
[1] 2 4 4 6
> 1:5 + 1:2
[1] 2 4 4 6 6
Warning message:
In 1:5 + 1:2 :
  la taille d'un objet plus long n'est pas multiple de la taille d'un objet plus court
\end{lstlisting}
  \end{block}
\end{frame}


%%%%%%%%%%%%%%%%%%%%%%%%%%%%%%%%%%%%%%%%%%%%%%%%%%%%%%%
\begin{frame}[fragile]
  \frametitle{Construction d'une \emph{nouvelle} séquence}
  \begin{alertblock}{Construction en extension}
    On utilise surtout la fonction \alert{\texttt{c}}.
    \begin{lstlisting}
> c(2, 3, 5)
[1] 2 3 5
> c(TRUE, FALSE, TRUE, FALSE, FALSE)
[1]  TRUE FALSE  TRUE FALSE FALSE
\end{lstlisting}
  \end{alertblock}

  \begin{block}{Longueur d'un vecteur}
    La longueur d’un vecteur est retournée par la fonction \texttt{length}.
    \begin{lstlisting}
> length(c("aa", "bb", "cc", "dd", "ee"))
[1] 5      
    \end{lstlisting}
  \end{block}

  \begin{block}{Concaténation de deux séquences}
    On peut coller côte à côte (concaténer) deux séquences pour construire une nouvelle séquence.
    Le vecteur numérique a été transformé (\alert{coerced}) en chaînes de caractères pour la cohésion de type.
    \begin{lstlisting}
> c(1:3, c("foo", "bar"))
[1] "1"   "2"   "3"   "foo" "bar"      
\end{lstlisting}
  \end{block}
\end{frame}
%%%%%%%%%%%%%%%%%%%%%%%%%%%%%%%%%%%%%%%%%%%%%%%%%%%%%%%
\begin{frame}[fragile]
  \frametitle{Extraction d'une tranche de séquence}
La notation des tranches (\alert{slices}) est valide pour toute séquence. \\
Elle permet d’extraire une sous-séquence à partir d’un \alert{vecteur numérique d’indices}.
\begin{itemize}
\item Les indices peuvent être dupliqués.
\item Les indices peuvent être dans n'importe quel ordre.  
\end{itemize}
\begin{lstlisting}
> vec <- 1:5
> vec[c(1,3,5)]
[1] 1 3 5
> vec[c(5,1,5)]
[1] 5 1 5  
\end{lstlisting}
\end{frame}

%%%%%%%%%%%%%%%%%%%%%%%%%%%%%%%%%%%%%%%%%%%%%%%%%%%%%%%
\begin{frame}[fragile]
  \frametitle{Extraction d'une tranche de séquence \dots}
  La notation des tranches permet d’extraire une sous-séquence à partir d’un \alert{vecteur logique d’indices}.
  Le booléen au rang \texttt{k} indique si le \texttt{k}-ème élément appartient à la sous-séquence.
  \begin{itemize}
  \item Les indices ne peuvent pas être dupliqués.
  \item Les indices ne peuvent pas être dans n'importe quel ordre.  
  \end{itemize}
  \begin{lstlisting}
> vec <- c("a", "b", "c", "d", "e")
> ind <- c(TRUE, TRUE, FALSE, FALSE, FALSE)
> vec[ind]
[1] "a" "b"
\end{lstlisting}

\begin{block}{Recyclage du vecteur logique}
  \begin{lstlisting}[style=block]
> vec[c(TRUE,FALSE)]
[1] "a" "c" "e"    
  \end{lstlisting}
\end{block}
\begin{block}{Extraction d'entêtes et de queues : \texttt{head} et \texttt{tail}}

  
\begin{columns}[t]
\begin{column}{0.33\textwidth}
  \begin{lstlisting}
> head(vec,3)
[1] "a" "b" "c"
\end{lstlisting}
\end{column}
\begin{column}{0.33\textwidth}
  \begin{lstlisting}
> tail(vec,3)
[1] "c" "d" "e"
\end{lstlisting}
\end{column}
\begin{column}{0.33\textwidth}
  \begin{lstlisting}
> tail(vec, -3)
[1] "d" "e"
\end{lstlisting}
\end{column}
\end{columns}
\end{block}
\end{frame}

%%%%%%%%%%%%%%%%%%%%%%%%%%%%%%%%%%%%%%%%%%%%%%%%%%%%%%%
\begin{frame}[fragile]
  \frametitle{Pré-allocation de mémoire}
    Prévoir l’espace mémoire nécessaire avant l’exécution du programme, en spécifiant la quantité nécessaire dans le code source.
  Au chargement du programme en mémoire, juste avant l’exécution, l’espace réservé devient alors accessible.


  \begin{exampleblock}{Réécrivons \texttt{1:n} avec ou sans préallocation de mémoire}
\begin{columns}[t]
\begin{column}{0.48\textwidth}
  \begin{lstlisting}[style=editor]
Seq1 <- function(n) {
  vec <- c() # sans
  for(i in seq(n)) {
    vec <- append(vec, i)
  }
  return(vec)
}
\end{lstlisting}
\end{column}
\begin{column}{0.48\textwidth}
  \begin{lstlisting}[style=editor]
Seq2 <- function(n) {
  vec <- numeric(n) # avec
  for(i in seq(n)) {
    vec[i] <- i
  }
  return(vec)
}
\end{lstlisting}
\end{column}
\end{columns}
\smallskip
\begin{lstlisting}
> system.time(replicate(5, Seq1(20000)))
utilisateur     système      écoulé 
      1.373       0.017       1.393   
  > system.time(replicate(5, Seq2(20000)))
utilisateur     système      écoulé 
      0.013       0.000       0.012   
\end{lstlisting}
\end{exampleblock}
  
\end{frame}

%%%%%%%%%%%%%%%%%%%%%%%%%%%%%%%%%%%%%%%%%%%%%%%%%%%%%%%
\begin{frame}[fragile]
  \frametitle{Modification d'un vecteur}
  Les vecteurs et les listes sont mutables, avec l'opérateur crochet, la méthode \texttt{append} (en queue ou à un rang donné) ou les indices négatifs (supprime un ou plusieurs éléments).
  \begin{exampleblock}{Ajout d'éléments}
\begin{columns}[t]
\begin{column}{0.33\textwidth}
  \begin{lstlisting}
> v <- 1:5
> v[1] <- 5
> v
[1] 5 2 3 4 5      
\end{lstlisting}
\end{column}
\begin{column}{0.33\textwidth}
  \begin{lstlisting}
> append(v,6)
[1] 5 2 3 4 5 6
> append(v,c(6,7))
[1] 5 2 3 4 5 6 7    
  \end{lstlisting}
\end{column}
\begin{column}{0.34\textwidth}
  \begin{lstlisting}
> append(v,6,2) 
[1] 5 2 6 3 4 5
> append(v,c(6,7),2)
[1] 5 2 6 7 3 4 5  
\end{lstlisting}
\end{column}

\end{columns}
\end{exampleblock}


\begin{columns}[t]
\begin{column}{0.43\textwidth}
\begin{exampleblock}{Suppression d'éléments}
  \begin{lstlisting}[style=block]
> v[-3]
[1] 1 2 4 5
> v[c(-2, -3)]
[1] 1 4 5
\end{lstlisting}  
\end{exampleblock}
\end{column}
\begin{column}{0.53\textwidth}
  \begin{block}{Rappel sur les chaînes}
    \begin{itemize}
    \item Les chaînes sont immutables.
    \item Ce ne sont pas des séquences.
    \item Ce sont des OBJETS !
    \end{itemize}

  \begin{lstlisting}
> str <- "foo"
> str[1]
[1] "foo"
\end{lstlisting}
\end{block}
\end{column}
\end{columns}
\end{frame}



%%%%%%%%%%%%%%%%%%%%%%%%%%%%%%%%%%%%%%%%%%%%%%%%%%%%%%%
\begin{frame}[fragile]
  \frametitle{Conditions et tests vectorisés !}
  \begin{block}{Opérations de comparaison vectorisées}
    Tous les opérateurs de comparaison sont vectorisés.  
  \begin{lstlisting}
> vec <- 1:5
> vec == 2
[1] FALSE TRUE FALSE FALSE FALSE
> vec > 3
[1] FALSE FALSE FALSE  TRUE  TRUE
\end{lstlisting}
\end{block}

\begin{alertblock}{Attention aux Opérations logiques vectorisées}
  Attention, les opérateurs \texttt{\&} et \texttt{\&\&} sont très différents pour un vecteur !
 \begin{lstlisting}
> vec %% 2 == 0 & vec > 2
[1] FALSE FALSE FALSE  TRUE FALSE
> vec %% 2 == 0 && vec > 2
[1] FALSE    
\end{lstlisting}

les opérateurs \texttt{|} et \texttt{||} sont tout aussi différents
\end{alertblock}
\end{frame}

%%%%%%%%%%%%%%%%%%%%%%%%%%%%%%%%%%%%%%%%%%%%%%%%%%%%%%%
\begin{frame}[fragile]
  \frametitle{Recherche des bornes d'un vecteur}
  \begin{block}{Recherche du rang du plus petit/grand élément}
    \begin{columns}[t]
      \begin{column}{0.49\textwidth}
  \begin{lstlisting}
> which.min(c(2, 3, 1, 5, 4))
[1] 3
  \end{lstlisting}
\end{column}
\begin{column}{0.49\textwidth}
  \begin{lstlisting}
> which.max(c(2, 3, 1, 5, 4))
[1] 4    
\end{lstlisting}
\end{column}
\end{columns}
\end{block}

\begin{block}{Recherche du plus petit/grand élément}
  \begin{columns}[t]
      \begin{column}{0.49\textwidth}
  \begin{lstlisting}
> min(c(2, 3, 1, 5, 4))
[1] 1
  \end{lstlisting}
\end{column}
\begin{column}{0.49\textwidth}
  \begin{lstlisting}
> max(c(2, 3, 1, 5, 4))
[1] 5    
\end{lstlisting}
\end{column}
\end{columns}    
  \end{block}

  \begin{block}{Recherche du plus petit et du plus grand élément}
   \begin{lstlisting}[style=block]
> range(c(2, 3, 1, 5, 4))
[1] 1 5    
  \end{lstlisting}
\end{block} 
\begin{block}{Le cas du \texttt{NA} : cela ne marche plus, supprimons-les}
  \begin{lstlisting}[style=block]
> range(c(2, 3, 1, 5, 4, NA))
[1] NA NA
> range(c(2, 3, 1, 5, 4, NA), na.rm = TRUE)
[1] 1 5
  \end{lstlisting}
\end{block}
\end{frame}


%%%%%%%%%%%%%%%%%%%%%%%%%%%%%%%%%%%%%%%%%%%%%%%%%%%%%%%
\begin{frame}[fragile]
  \frametitle{Recherche dans un vecteur : \texttt{which}}
  La fonction \texttt{which} renvoie un vecteur numérique d'indices dont les éléments respectent la condition.
  \begin{lstlisting}
> vec <- 1:5
> vec == 2
[1] FALSE TRUE FALSE FALSE FALSE
> vec %% 2 == 0 & vec > 2
[1] FALSE FALSE FALSE  TRUE FALSE
> which(vec == 2)
[1] 2
> which(vec %% 2 == 0)
[1] 2 4
> which(vec %% 2 == 0 & vec > 2)
[1] 4
> which(vec %% 2 == 0 && vec > 2)
integer(0)    
\end{lstlisting}
\end{frame}





%%%%%%%%%%%%%%%%%%%%%%%%%%%%%%%%%%%%%%%%%%%%%%%%%%%%%%%
\begin{frame}[fragile]
  \frametitle{Recherche dans un vecteur : \texttt{match}}
  La fonction \texttt{match} renvoie un vecteur numérique d'indices du premier élément de son premier argument contenu dans le second.
  \begin{lstlisting}
> match(1:5, 2)
[1] NA 1 NA NA NA
> 1:5 %in% 2
[1] FALSE TRUE FALSE FALSE FALSE
> which(1:5 %in% 2)
[1] 2
> match( 1:5, c(2,6,3))
[1] NA 1 3 NA NA
> match( 1:5, c(2,6,3), nomatch=0)
[1] 0 1 3 0 0
  \end{lstlisting}
\end{frame}


%%%%%%%%%%%%%%%%%%%%%%%%%%%%%%%%%%%%%%%%%%%%%%%%%%%%%%%
\begin{frame}[fragile]
  \frametitle{Fonctions \texttt{all} et \texttt{any} sur un vecteur logique}
  % \begin{itemize}
  % \item \texttt{all} renvoie \texttt{TRUE} si tous les éléments d’un vecteur logique sont \texttt{TRUE}.
  % \item \texttt{any} renvoie \texttt{TRUE} si au moins un élément est \texttt{TRUE}.
  % \end{itemize}

  \begin{block}{\texttt{all} renvoie \texttt{TRUE} si tous les éléments sont \texttt{TRUE}.}
  \begin{lstlisting}[style=block]
> all(c(TRUE,TRUE))
[1] TRUE
> all(c(FALSE,TRUE))
[1] FALSE
> all(c(TRUE,TRUE, NA))
[1] NA
> all(c(TRUE,TRUE, NA), na.rm=TRUE)
[1] TRUE
> all(1:10 > 1)
[1] FALSE
\end{lstlisting}
  \end{block}

  \begin{block}{\texttt{any} renvoie \texttt{TRUE} si au moins un élément est \texttt{TRUE}.}
  \begin{lstlisting}[style=block]
> any(c(FALSE,FALSE))
[1] FALSE
> any(c(FALSE,TRUE))
[1] TRUE
> any(1:10 < 2)
[1] TRUE
\end{lstlisting}
  \end{block}

\end{frame}


%%%%%%%%%%%%%%%%%%%%%%%%%%%%%%%%%%%%%%%%%%%%%%%%%%%%%%%

\begin{frame}[fragile]
  \frametitle{Tri d'un vecteur : \texttt{sort}}
  \begin{block}{Dans un ordre donné}
\begin{lstlisting}[style=block]
> x <- sample(10)
> sort(x)
 [1]  1  2  3  4  5  6  7  8  9 10
> sort(x, decreasing=TRUE)
 [1] 10  9  8  7  6  5  4  3  2  1
\end{lstlisting}    
\end{block}

\begin{block}{En retournant éventuellement les indices}
\begin{lstlisting}[style=block]
> x
 [1]  9  3 10  1  4  8  5  7  6  2  
> sort(x, index.return = TRUE)
$x
 [1]  1  2  3  4  5  6  7  8  9 10
$ix
 [1]  4 10  2  5  7  9  8  6  1  3
\end{lstlisting}    
\end{block}

\begin{block}{De n'importe quel type}
\begin{lstlisting}[style=block]
> sort(head(letters, 10), decreasing=TRUE)
 [1] "j" "i" "h" "g" "f" "e" "d" "c" "b" "a"  
\end{lstlisting}    
\end{block}


\end{frame}



%%%%%%%%%%%%%%%%%%%%%%%%%%%%%%%%%%%%%%%%%%%%%%%%%%%%%%%

\begin{frame}[fragile]
  \frametitle{Méthodes usuelles sur les vecteurs}
  \begin{center}
    Consultez notre superbe \alert{reference card} !
  \end{center}
\end{frame}


%%%%%%%%%%%%%%%%%%%%%%%%%%%%%%%%%%%%%%%%%%%%%%%%%%%%%%%
\begin{frame}[fragile]
  \frametitle{Extraction des nombres pairs d'un vecteur}
  \begin{block}{Méthode classique}
    Facilement transportable dans d'autres langages de programmation.
    \begin{lstlisting}[style=editor]
ExtractionNombresPairs <- function(x) {
  res <- numeric(0);
  for(v in x) {
    if( v %% 2 == 0) res <- append(res, v);
  }
  return(res)
}      
\end{lstlisting}
\begin{lstlisting}
> x <- 1:10
> ExtractionNombresPairs(x)
[1]  2  4  6  8 10
\end{lstlisting}
  \end{block}

  \begin{block}{Méthodes plus courtes, mais spécifiques à R}
\begin{lstlisting}[style=block]
> x[ x %% 2 == 0]
> x[ which(x %% 2 == 0)]
> Filter(function(n) n %% 2 == 0, x)
\end{lstlisting}
    
  \end{block}
\end{frame}


%%%%%%%%%%%%%%%%%%%%%%%%%%%%%%%%%%%%%%%%%%%%%%%%%%%%%%%
\questionSlide

%%%%%%%%%%%%%%%%%%%%%%%%%%%%%%%%%%%%%%%%%%%%%%%%%%%%%%%
 \appendix
 \backupSlides
%%%%%%%%%%%%%%%%%%%%%%%%%%%%%%%%%%%%%%%%%%%%%%%%%%%%%%%

%%%%%%%%%%%%%%%%%%%%%%%%%%%%%%%%%%%%%%%%%%%%%%%%%%%%%%%
% \begin{frame}[fragile]{Backup slides}
%   Sometimes, it is useful to add slides at the end of your presentation to
%   refer to during audience questions.

%   The best way to do this is to include the \verb|appendixnumberbeamer|
%   package in your preamble and call \verb|\appendix| before your backup slides.

%   will automatically turn off slide numbering and progress bars for
%   slides in the appendix.
% \end{frame}


%%%%%%%%%%%%%%%%%%%%%%%%%%%%%%%%%%%%%%%%%%%%%%%%%%%%%%%
% \begin{frame}[allowframebreaks]{References}

%   % \bibliography{../bib_parallelism,../bib_others}
%   % \bibliographystyle{abbrv}

% \end{frame}

\end{document}

%%% Local Variables:
%%% mode: latex
%%% TeX-master: t
%%% End:
