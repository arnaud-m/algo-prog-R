\documentclass[10pt]{beamer}

\usepackage{../macros}
\title{Conversions}

\hypersetup{
  pdftitle =  {Conversions}
}

\begin{document}

\maketitle



%%%%%%%%%%%%%%%%%%%%%%%%%%%%%%%%%%%%%%%%%%%%%%%%%%%%%%%
\begin{frame}[fragile]
  \frametitle{Épluchage d'un entier chiffre à chiffre}

  \begin{itemize}
  \item Seul le chiffre des unités (le plus à droite) est accessible, 
  \item ainsi que l'entier obtenu en supprimant le chiffre des unités (décalage à droite).
  \end{itemize}
 
  \begin{columns}[c]
    \begin{column}{0.6\textwidth}
      \begin{center}
  \begin{tabular}{c|c}
    \toprule
    345 & 6 \\
    \midrule
    \texttt{\%/\%} &     \texttt{\%\%} \\
    \bottomrule
  \end{tabular}
\end{center}  
\end{column}
\begin{column}{0.4\textwidth}
  \begin{lstlisting}
> n <- 3456
> n %% 10
[1] 6
> n %/% 10
[1] 345
\end{lstlisting}
\end{column}
\end{columns}



\begin{columns}[t]
\begin{column}{0.62\textwidth}
\begin{lstlisting}[style=editor]
Eplucher <- function(n) {
  while(n >0) {
    print(paste(n,"j'enleve", n%%10))
    n <- n %/% 10
  }
}  
\end{lstlisting}
\end{column}
\begin{column}{0.38\textwidth}
\begin{lstlisting}
> Eplucher(3456)
[1] "3456 j'enleve 6"
[1] "345 j'enleve 5"
[1] "34 j'enleve 4"
[1] "3 j'enleve 3"  
\end{lstlisting}
\end{column}
\end{columns}

\end{frame}

%%%%%%%%%%%%%%%%%%%%%%%%%%%%%%%%%%%%%%%%%%%%%%%%%%%%%%%
\begin{frame}[fragile]
  \frametitle{Épluchage récursif d'un entier}
  \begin{block}{Une fonction récursive s'appelle elle-même.}
    \begin{itemize}
    \item Il est impératif de prévoir une condition d'arrêt à la récursion, sinon le programme ne s'arrête jamais!
    \item La récursivité fonctionne car chaque appel de fonction est différent.
    \end{itemize}

  \end{block}

  \begin{lstlisting}[style=editor]
Eplucher <- function(n) {
  if(n > 0) {
    print(paste(n,"j'enleve", n %% 10))
    Eplucher(n %/% 10)
  }
}    
  \end{lstlisting}

  \begin{block}{Procédure}
    Une procédure est une routine qui ne retourne pas de valeur.
  \end{block}

\end{frame}



\begin{frame}[fragile]
  \frametitle{Épluchage  binaire d'un entier}
  \begin{center}
    \alert{\textbf{Le principe reste le même !}}
  \end{center}
 

\begin{columns}[t]
\begin{column}{0.62\textwidth}
\begin{lstlisting}[style=editor]
Eplucher <- function(n) {
  while(n >0) {
    print(n%%2)
    n <- n %/% 2
  }
}  
\end{lstlisting}
Épluchage de la droite vers le gauche.
$$
(3456)_{10} = (110110000000)_2
$$

\end{column}
\begin{column}{0.38\textwidth}
  \begin{lstlisting}
> Eplucher(3456)
[1] 0
[1] 0
[1] 0
[1] 0
[1] 0
[1] 0
[1] 0
[1] 1
[1] 1
[1] 0
[1] 1
[1] 1    
\end{lstlisting}
\end{column}
\end{columns}
\end{frame}


%%%%%%%%%%%%%%%%%%%%%%%%%%%%%%%%%%%%%%%%%%%%%%%%%%%%%%%
\begin{frame}[fragile]
  \frametitle{Variante : bit de poids fort}
  Le bit de poids fort, (en anglais \alert{most significant bit}, ou MSB) est le bit, dans une représentation binaire donnée, ayant la plus grande valeur.

  \begin{exampleblock}{Le nombre $(9)_{10}$ s'écrit $(1001)_2$ en binaire}
    Le MSB (à gauche) contribue pour 8 unités à la valeur totale du nombre.
  \end{exampleblock}


\begin{columns}[t]
\begin{column}{0.62\textwidth}
\begin{lstlisting}[style=editor]
MSB <- function(n) {
  if(n <= 0) return(0);
  msb <- 1;
  while(msb <= n) {
    msb <- 2 * msb
  }
  return(msb/2)
}  
\end{lstlisting}
\end{column}
\begin{column}{0.38\textwidth}
  \begin{lstlisting}
> MSB(9)
[1] 8
> MSB(16)
[1] 16
> MSB(25)
[1] 16
> MSB(32)
[1] 32
> MSB(33)
[1] 32
\end{lstlisting}
\end{column}
\end{columns}
\end{frame}

%%%%%%%%%%%%%%%%%%%%%%%%%%%%%%%%%%%%%%%%%%%%%%%%%%%%%%%
\begin{frame}[fragile]
  \frametitle{Épluchage d'un nombre fractionnaire}
  \begin{itemize}
  \item Seul le chiffre le plus à gauche est accessible, 
  \item ainsi que le nombre fractionnaire obtenu en supprimant le chiffre des le plus à gauche (décalage à gauche).
  \end{itemize}
 
  \begin{columns}[c]
    \begin{column}{0.6\textwidth}
      \begin{center}
  \begin{tabular}{c|c}
    \toprule
    3                    & 0.456                         \\
    \midrule
    \texttt{floor(10*n)} & \texttt{10*n - floor(10*n)} \\
    \bottomrule
  \end{tabular}
\end{center}  
\end{column}
\begin{column}{0.4\textwidth}
  \begin{lstlisting}
> n <- 0.3456
> m <- 10 * n
> floor(m)
[1] 3
> m - floor(m)
[1] 0.456
\end{lstlisting}
\end{column}
\end{columns}



\begin{columns}[t]
\begin{column}{0.6\textwidth}
\begin{lstlisting}[style=editor]
Eplucher <- function(n) {
  while(n >0) {
    m <- 10 * n
    f <- floor(m)
    print(paste(n,"j'enleve", f))
    n <- m - f
  }
}  
\end{lstlisting}
\end{column}
\begin{column}{0.4\textwidth}
\begin{lstlisting}
> Eplucher(1/32)
[1] "0.03125 j'enleve 0"
[1] "0.3125 j'enleve 3"
[1] "0.125 j'enleve 1"
[1] "0.25 j'enleve 2"
[1] "0.5 j'enleve 5"
\end{lstlisting}
\end{column}
\end{columns}

\end{frame}


%%%%%%%%%%%%%%%%%%%%%%%%%%%%%%%%%%%%%%%%%%%%%%%%%%%%%%%
\begin{frame}[fragile]
  \frametitle{Surprise ! Le calcul fractionnaire n'est pas exact !}
\begin{lstlisting}
> Eplucher(0.3456)
[1] "0.3456 j'enleve 3"
[1] "0.456 j'enleve 4"
[1] "0.560000000000004 j'enleve 5"
[1] "0.600000000000041 j'enleve 6"
[1] "4.05009359383257e-13 j'enleve 0"
[1] "4.05009359383257e-12 j'enleve 0"
[1] "4.05009359383257e-11 j'enleve 0"
[1] "4.05009359383257e-10 j'enleve 0"
[1] "4.05009359383257e-09 j'enleve 0"
[1] "4.05009359383257e-08 j'enleve 0"
[1] "4.05009359383257e-07 j'enleve 0"
[1] "4.05009359383257e-06 j'enleve 0"
[1] "4.05009359383257e-05 j'enleve 0"
[1] "0.000405009359383257 j'enleve 0"
...
[1] "0.53125 j'enleve 5"
[1] "0.3125 j'enleve 3"
[1] "0.125 j'enleve 1"
[1] "0.25 j'enleve 2"
[1] "0.5 j'enleve 5"  
\end{lstlisting}
\end{frame}

%%%%%%%%%%%%%%%%%%%%%%%%%%%%%%%%%%%%%%%%%%%%%%%%%%%%%%%
 \questionSlide

%%%%%%%%%%%%%%%%%%%%%%%%%%%%%%%%%%%%%%%%%%%%%%%%%%%%%%%
 \appendix
 \backupSlides
%%%%%%%%%%%%%%%%%%%%%%%%%%%%%%%%%%%%%%%%%%%%%%%%%%%%%%%



%%%%%%%%%%%%%%%%%%%%%%%%%%%%%%%%%%%%%%%%%%%%%%%%%%%%%%%
% \begin{frame}[allowframebreaks]{References}

%   % \bibliography{../bib_parallelism,../bib_others}
%   % \bibliographystyle{abbrv}

% \end{frame}

\end{document}

%%% Local Variables:
%%% mode: latex
%%% TeX-master: t
%%% End:
