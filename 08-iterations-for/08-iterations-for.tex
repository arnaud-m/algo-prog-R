\documentclass[10pt]{beamer}

\usepackage{../macros}
\title{Itérations (\texttt{for})}

\hypersetup{
  pdftitle =  {Itérations (for)}
}

\begin{document}

\maketitle





%%%%%%%%%%%%%%%%%%%%%%%%%%%%%%%%%%%%%%%%%%%%%%%%%%%%%%%
\begin{frame}[fragile]
  \frametitle{La boucle \texttt{for}}
  
  En théorie, la boucle \texttt{while} permet de réaliser toutes les boucles que l'on veut.
  Toutefois, les boucles \texttt{for} sont très utilisées.

  
\begin{columns}[t]
\begin{column}{0.48\textwidth}
  \begin{lstlisting}[style=editor]
S <- function(n) {
  i <- 0
  acc <- 0
  while(i < n) {
    i <- i + 1
    acc <- acc + i
  }
  return(acc)
}
\end{lstlisting}
\end{column}
\begin{column}{0.43\textwidth}
\begin{lstlisting}[style=editor]
S <- function(n) {
  acc <- 0
  for(i in 1:n) {
    acc <- acc +i
  }
  return(acc)
}  
\end{lstlisting}
\end{column}
\end{columns}

\begin{itemize}
\item Souvent, on utilisera une boucle for pour incrémenter ou décrémenter un compteur.
\item La boucle \texttt{for} est plus pratique ici, mais pas pour l'epluchage des entiers.

\end{itemize}
\end{frame}

%%%%%%%%%%%%%%%%%%%%%%%%%%%%%%%%%%%%%%%%%%%%%%%%%%%%%%%
\begin{frame}[fragile]
  \frametitle{Échappement d'une boucle \texttt{for}}
  Lorsque l'instruction \texttt{return} ou \texttt{break} est placée à l'intérieur de la boucle \texttt{for}, cela signifie :
  \begin{center}
    \alert{Je parcours a priori \emph{tout} une séquence,\\ mais je me réserve la possibilité de \emph{m'échapper} en cours de route !}

  \end{center}
\begin{exampleblock}{Exemple : comparaison d'une valeur à une séquence}
  Est-ce que la valeur \texttt{x} est plus grande ou égale aux valeurs d’une séquence ?
\begin{columns}[t]
\begin{column}{0.48\textwidth}
  \begin{lstlisting}[style=editor]
geq <- function(x, values) {
  for(v in values) {
    if(x < v) return(FALSE)
  }
  return(TRUE)
}         
  \end{lstlisting}
\end{column}
\begin{column}{0.48\textwidth}
  \begin{lstlisting}
> values <- c(2, 4, 1, 7, 5)
> geq(0, values)
[1] FALSE
> geq(1, values)
[1] FALSE
> geq(7, values)
[1] TRUE
> geq(8, values)
[1] TRUE
\end{lstlisting}

\end{column}
\end{columns}
\end{exampleblock}
\end{frame}

%%%%%%%%%%%%%%%%%%%%%%%%%%%%%%%%%%%%%%%%%%%%%%%%%%%%%%%
\begin{frame}
  \frametitle{Une séquence ?}
  Nous commençons à pénétrer dans un continent qu'il faudra tôt ou tard aborder : celui des \alert{VECTEURS} et \alert{LISTES}.
  Nous garderons pour l'instant une idée naïve de ce qu'est un vecteur, une \alert{séquence d’éléments}.
\end{frame}

%%%%%%%%%%%%%%%%%%%%%%%%%%%%%%%%%%%%%%%%%%%%%%%%%%%%%%%
\begin{frame}[fragile]
  \frametitle{  La fonction \texttt{seq} : génèrer une séquence régulière}
  \begin{center}
    \alert{\texttt{seq(from, to, by, length.out, ...)}}
  \end{center}

\begin{lstlisting}
> seq(5) #ou mieux seq_len(5)
[1] 1 2 3 4 5
> seq(0,5)
[1] 0 1 2 3 4 5
> seq(from=0,to=5)
[1] 0 1 2 3 4 5
> seq(from=0,to=5,by=2)
[1] 0 2 4
> seq(from=0,to=5, by=1.25)
[1] 0.00 1.25 2.50 3.75 5.00
> seq(from=0,to=5, length.out=5)
[1] 0.00 1.25 2.50 3.75 5.00
\end{lstlisting}

En version courte pour générer des entiers consécutifs :   
\begin{lstlisting}
> 1:5
[1] 1 2 3 4 5
> 1:0 # /!\ ATTENTION ! 
[1] 1 0
\end{lstlisting}
\end{frame}



\begin{frame}[fragile]
  \frametitle{La boucle for parcourt un objet itérable}

  Nous avons utilisé jusqu'à présent la boucle \texttt{for} sur une séquence d’entiers.
  Or \texttt{seq} ne construit pas forcément l'intervalle, mais un itérateur sur un vecteur virtuel.
  Heureusement d'ailleurs :
\begin{columns}[t]
\begin{column}{0.58\textwidth}
  \begin{lstlisting}[style=editor]
for(i in seq(1,10**7, by=2)) {
  print(i)
  if(i > 10) break
}    
  \end{lstlisting}
\end{column}
\begin{column}{0.38\textwidth}
  \begin{lstlisting}
[1] 1
[1] 3
[1] 5
[1] 7
[1] 9
[1] 11
  \end{lstlisting}
\end{column}
\end{columns}

Les séquences (vector, list) sont des objets itérables .
\begin{block}{Syntaxe}
  
\end{block}{Syntaxe de la boucle \texttt{for}}
\begin{lstlisting}[style=edblock]
for (x in sequence) {
  ...
}  
\end{lstlisting}
  
\end{frame}




%%%%%%%%%%%%%%%%%%%%%%%%%%%%%%%%%%%%%%%%%%%%%%%%%%%%%%%
\begin{frame}[fragile]
  \frametitle{  La fonction \texttt{sample} : génèrer une séquence aléatoire}
  On va choisir aléaloirement \texttt{size} éléments du vecteur \texttt{x} avec ou sans remise (\texttt{replace}) et avec ou sans biais (\texttt{prob}).
  \begin{center}
    \alert{\texttt{sample(x, size, replace = FALSE, prob = NULL)}}
  \end{center}

 
  \begin{block}{Tirage sans remise (\texttt{replace=FALSE})}
La variante \texttt{sample.int} est un raccourci pour choisir dans \texttt{1:n}    
    \begin{lstlisting}
> sample.int(10) # permutation aléatoire de [1, 10]
[1]  8  6 10  7  5  4  3  1  2  9
> sample.int(n = 90,  size = 6) # tirage du loto
[1]  5 26 11 77 70 32
> sample.int(n = 10,  size = 11) # Choisir 11 parmi 10
Error in sample.int(10, 11) : 
  impossible de prendre un échantillon plus grand que la population lorsque 'replace = FALSE'
\end{lstlisting}
\end{block}
\end{frame}


%%%%%%%%%%%%%%%%%%%%%%%%%%%%%%%%%%%%%%%%%%%%%%%%%%%%%%%
\begin{frame}[fragile]
  \frametitle{  La fonction \texttt{sample} \dots}
  On va choisir aléaloirement \texttt{size} éléments du vecteur \texttt{x} avec ou sans remise (\texttt{replace}) et avec ou sans biais (\texttt{prob}).
  \begin{center}
    \alert{\texttt{sample(x, size, replace = FALSE, prob = NULL)}}
  \end{center}

\begin{block}{Tirage avec remise (\texttt{replace=TRUE})}
    \begin{lstlisting}[style=block]
>  sample(c('pile', 'face'),  size = 5, replace = TRUE) 
[1] "face" "pile" "pile" "pile" "pile"
\end{lstlisting}
\end{block}


\begin{block}{Tirage biaisé avec remise (\texttt{prob})}
    \begin{lstlisting}[style=block]
> sample(c('pile', 'face'),  size = 10, replace = TRUE, prob = c(5, 1)) 
 [1] "pile" "pile" "pile" "face" "face" "pile" "pile" "pile" "pile" "pile"
\end{lstlisting}
\end{block}

\end{frame}


%%%%%%%%%%%%%%%%%%%%%%%%%%%%%%%%%%%%%%%%%%%%%%%%%%%%%%%
\begin{frame}[fragile]
  \frametitle{  Génèrer des nombres approchés aléatoires}
  On va tirer des \texttt{n} nombres approchés aléatoires compris entre \texttt{min} et \texttt{max} avec une probabilité uniforme.
  \begin{center}
    \alert{\texttt{ runif(n, min = 0, max = 1)}}
  \end{center}

    \begin{lstlisting}
> runif(5)
[1] 0.36051021 0.96824951 0.08495143 0.87527313 0.16520820
> runif(n = 5, min = 0, max = 10)
[1] 7.945856 8.398957 2.757909 6.035876 1.205764
\end{lstlisting}


\begin{block}{Autres distributions}
  \texttt{rnorm}, \texttt{rpois}, \texttt{rgamma} \dots 
\end{block}
\end{frame}


%%%%%%%%%%%%%%%%%%%%%%%%%%%%%%%%%%%%%%%%%%%%%%%%%%%%%%%
\begin{frame}[fragile]
  \frametitle{Mais, il faut éviter les boucles !}
  \begin{center}
    \alert{Quand c’est possible, il faut mieux utiliser une fonction prédéfinie.}
  \end{center}

\begin{columns}[t]
\begin{column}{0.48\textwidth}
  \begin{lstlisting}[style=editor]
SP <- function(n) {
  if(n <= 0) return(0)
  return(sum(1:n))
}    
\end{lstlisting}  
\end{column}
\begin{column}{0.48\textwidth}
  \begin{lstlisting}[style=editor]
SR <- function(n) {
  if(n > 0) {
    return(n+SR(n-1))
  }
  else return(0)
}    
  \end{lstlisting}
\end{column}
\end{columns}

\begin{columns}[t]
\begin{column}{0.48\textwidth}
  \begin{lstlisting}[style=editor]
SW <- function(n) {
  i <- 0
  acc <- 0
  while(i < n) {
    i <- i + 1
    acc <- acc +i
  }
  return(acc)
}
\end{lstlisting}
\end{column}
\begin{column}{0.48\textwidth}
  \begin{lstlisting}[style=editor]
SF <- function(n) {
  if(n <= 0) return(0)
  acc <- 0
  for(i in 1:n) {
    acc <- acc +i
  }
  return(acc)
}
\end{lstlisting}
\end{column}
\end{columns}

\end{frame}



%%%%%%%%%%%%%%%%%%%%%%%%%%%%%%%%%%%%%%%%%%%%%%%%%%%%%%%
\begin{frame}[fragile]
  \frametitle{La vectorisation est plus efficace !}

\begin{lstlisting}
> system.time(replicate(10**4, invisible(SP(10**6))))
utilisateur     système      écoulé 
      0.041       0.000       0.042 
\end{lstlisting}


\begin{table}[h]
  \centering
  \begin{tabular}{lrr}
    \toprule
                & $\mathtt{n}=10^3$ & $\mathtt{n}=10^4$ \\
                & temps (s)         & temps (s)         \\
    \midrule
    \texttt{SR} & 3.200             & overflow          \\
    \texttt{SW} & 0.547             & 4.664             \\
    \texttt{SF} & 0.260             & 2.455             \\
    \texttt{SP} & 0.011             & 0.011             \\
    \bottomrule
  \end{tabular}
\end{table}

\begin{itemize}
\item<alert@1> La suppression des boucles s’appelle la vectorisation.
\item Nous découvrirons plus tard autre famille de boucles : \texttt{apply}.
\end{itemize}

\end{frame}


%%%%%%%%%%%%%%%%%%%%%%%%%%%%%%%%%%%%%%%%%%%%%%%%%%%%%%%



 \questionSlide

%%%%%%%%%%%%%%%%%%%%%%%%%%%%%%%%%%%%%%%%%%%%%%%%%%%%%%%
 \appendix
 \backupSlides
%%%%%%%%%%%%%%%%%%%%%%%%%%%%%%%%%%%%%%%%%%%%%%%%%%%%%%%

%%%%%%%%%%%%%%%%%%%%%%%%%%%%%%%%%%%%%%%%%%%%%%%%%%%%%%%
% \begin{frame}[fragile]{Backup slides}
%   Sometimes, it is useful to add slides at the end of your presentation to
%   refer to during audience questions.

%   The best way to do this is to include the \verb|appendixnumberbeamer|
%   package in your preamble and call \verb|\appendix| before your backup slides.

%   will automatically turn off slide numbering and progress bars for
%   slides in the appendix.
% \end{frame}


%%%%%%%%%%%%%%%%%%%%%%%%%%%%%%%%%%%%%%%%%%%%%%%%%%%%%%%
% \begin{frame}[allowframebreaks]{References}

%   % \bibliography{../bib_parallelism,../bib_others}
%   % \bibliographystyle{abbrv}

% \end{frame}

\end{document}

%%% Local Variables:
%%% mode: latex
%%% TeX-master: t
%%% End:
