\begin{frame}{Multiplication égyptienne}
  \begin{block}{Principe}
    \begin{itemize}
    \item Décomposition d'un des nombres en une somme.\\
      On décompose généralement le plus petit.
    \item Création d'une table de puissance pour l'autre nombre
    \item Très souvent, traduction du décimal vers le binaire.\\
      Il existe des variantes en fonction de la complexité de l'opération.
    \item Il suffit de savoir multiplier par deux et additionner !
    \end{itemize}
  \end{block}


\end{frame}



\begin{frame}{Multiplication égyptienne : $x \times y$}

  \begin{itemize}
  \item On construit la ligne $i+1$ en multipliant par deux $2^i$ et $x \times 2^i$ tant que $2^{i+1}< y$.
  \item<2> On traduit $21$ en binaire en remontant dans le tableau. 
  \item<2> On calcule $\sum_0^{n} a_i (x \times  2^i)$.
  \end{itemize}

\begin{exampleblock}{$189\times 21=3969$}
      \begin{tabular}{c|c|r|rcc}

        $i$ & $a_i$           & $2^i$ & $189 \times 2^i$                               \\
        \cline{1-4}
        0   & \onslide<2->{1} & 1     & 189  & \onslide<2->{\checkmark & (1 -1 = 0)}   \\
        1   & \onslide<2->{0} & 2     & 378  &                                         \\
        2   & \onslide<2->{1} & 4     & 756  & \onslide<2->{\checkmark & (5 -4 = 1)}   \\
        3   & \onslide<2->{0} & 8     & 1512 &                                         \\
        4   & \onslide<2->{1} & 16    & 3024 & \onslide<2->{\checkmark & (21 -16 = 5)} \\
        \cline{1-4}
            &                 &       & \onslide<2>{3969}

      \end{tabular}
\end{exampleblock}
\end{frame}

%%%%%%%%%%%%%%%%%%%%%%%%%%%%%%%%%%%%%%%%%%%%%%%%%%%%%%%%
\begin{frame}
\frametitle{Méthode genérale}

Des opérations plus complexes faisant intervenir par exemple des fractions exigeaient une décomposition avec:
\begin{itemize}
\item  les puissances de deux,
\item les fractions fondamentales,
\item les dizaines. 
\end{itemize}
La technique est rigoureusement la même mais offre plus de liberté au scribe quant à la décomposition du petit nombre.
\begin{columns}
\begin{column}[t]{0.48\textwidth}
  \begin{exampleblock}{$243\times27 = 6561$}
    \begin{tabular}{r|r}
      1 & 243 \\
      2 & 486 \\
      4 & 972 \\
      20 & 4860 \\ \hline
      27 & 6561
      \end{tabular}    
  \end{exampleblock}
\end{column}

\begin{column}[t]{0.48\textwidth}
  \begin{exampleblock}{Papyrus Rhind: $\frac{1}{14} \times \frac{7}{4} = \frac{1}{8}$}
    \begin{tabular}{r|rl}
      $1$ & $\frac{1}{14}$& \\[3pt]
      $\frac{1}{2}$ & $\frac{1}{28}$& \\[3pt]
      $\frac{1}{4}$ & $\frac{1}{56}$& \\[3pt] \cline{1-2} \\[-1.0em] 
      $\frac{7}{4}$ & $\frac{1}{8}$& $(\frac{4+2+1}{56})$ 
    \end{tabular}
  \end{exampleblock}
\end{column}
\end{columns}

\end{frame}


%%%%%%%%%%%%%%%%%%%%%%%%%%%%%%%%%%%%%%%%%%%%%%%%%%%%%%%%
\begin{frame}{Division égyptienne : $x \div y$}
  \begin{block}{Par quoi doit-on multiplier $y$ pour trouver $x$ ?}
    \begin{itemize}
    \item On construit la ligne $i+1$ en multipliant par deux $2^i$ et $y\times 2^i$ tant que $y \times 2^{i+1}< x$.
    \item<2> On décompose $x$ par les $y\times 2^i$ en remontant dans le tableau. 
    \item<2> On calcule la somme des $2^i$ de la décomposition.

    \end{itemize}
  \end{block}


\begin{exampleblock}{$539\div 7=77$}
  \begin{tabular}{c|r|rcc}
    $i$ & $2^i$ & $7 \times 2^i$& \\
    \cline{1-3}
    0& 1&7  & \onslide<2>{\checkmark & ($7-7=0$)}\\
    1& 2&14 & \\
    2& 4&28 &\onslide<2>{\checkmark & ($35-28=7$)}\\
    3& 8&56 & \onslide<2>{\checkmark & ($91-56=35$)}\\
    4& 16&112 & \\
    5& 32&224 & \\        
    6& 64&448 & \onslide<2>{\checkmark & ($539-448=91$)}\\  
    \cline{1-3}
    &  \onslide<2->{77} & 
  \end{tabular}
\end{exampleblock}
\end{frame}

%%%%%%%%%%%%%%%%%%%%%%%%%%%%%%%%%%%%%%%%%%%%%%%%%%%%
\begin{frame}{Division dont le résultat est fractionnaire}

\begin{exampleblock}{$234\div 12=19.5$}
  \begin{tabular}{r|r|rcc}
    $i$ & $2^i$ & $12 \times 2^i$& \\
    \cline{1-3}
    \onslide<3->{-1& $\frac{1}{2}$ & 6  & \checkmark & ($0$)}\\
    0& 1&12  & \onslide<2->{\checkmark & ($6$)}\\
    1& 2&24 & \onslide<2->{\checkmark & ($18$)}\\
    2& 4&48 &\\
    3& 8&96 & \\
    4& 16&192 & \onslide<2->{\checkmark & ($42$)}\\
    \cline{1-3}
    &  \onslide<3->{19.5} & 
  \end{tabular}
\end{exampleblock}
\end{frame}


%%%%%%%%%%%%%%%%%%%%%%%%%%%%%%%%%%%%%%%%%%%%%%%%%%%%%%%%
\begin{frame}{Exercices}
  
  \begin{enumerate}
  \item Multipliez 187 par 11.
  \item Multipliez 2012 par 1515 (indice: utilisez la méthode genérale).
  \end{enumerate}
\begin{columns}
\begin{column}[t]{0.48\textwidth}
  \begin{exampleblock}<2->{$187\times 11$}
    \begin{tabular}{c|c|r|r}
        $i$ & $a_i$ & $2^i$ & $11 \times 2^i$ \\
        \hline
        0& 1 & 1 & 187\\
        1& 1 & 2 & 374\\
        2& 0 & 4 & 748\\
        3& 1 & 8 & 1496\\ \hline
        &  & 11 &2057
      \end{tabular}    
  \end{exampleblock}
\end{column}
\begin{column}[t]{0.48\textwidth}
    \begin{exampleblock}<2>{$2012\times 1515$}
      \begin{tabular}{r|r}
        5& 10060 \\
        10& 20120 \\
        500 & 1006000 \\
        1000 & 2012000\\ \hline
        1515 & 3048180 
      \end{tabular}
  \end{exampleblock}
\end{column}
\end{columns}

\end{frame}


%%% Local Variables:
%%% mode: latex
%%% TeX-master: "../representation-nombres"
%%% End:
