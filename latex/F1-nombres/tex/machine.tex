%%%%%%%%%%%%%%%%%%%%%%%%%%%%%%%%%%%%%%%%%%%%%%%%%%%%
\begin{frame}{Représentation des nombres en machine}

  \begin{block}{Précision finie}
    \begin{itemize}
    \item Codés généralement sur 16, 32 ou 64 bits.
    \item Un codage sur $n$ bits permet de représenter $2^n$ valeurs distinctes.
    \end{itemize}
  \end{block}
  
  \begin{alertblock}{Un ordinateur ne calcule pas bien !}
  \begin{itemize}
   \item Pour un ordinateur, le nombre de chiffres est fixé.
   \item Pour un mathématicien, le nombre de chiffres dépend de la valeur représentée.
   \item Lorsque le résultat d’un calcul doit être représenté sur plus de chiffres que ceux disponibles, il y a dépassement de capacité.
  \end{itemize}
\end{alertblock}

\end{frame}


\begin{frame}{Représentation des entiers en machine}
  \begin{block}{Entiers naturels}
    \begin{itemize}
    \item Un codage sur $n$ bits : tous les entiers entre 0 et $2^n-1$.
    \item La conversion d’un nombre entier s’arrête toujours.
    \item On ne peut traiter $x - y$ que si $x \geq y$. 
    \end{itemize}
  \end{block}

  \begin{block}{Entiers relatifs : plusieurs représentations existent.}
    Valeur absolue signée ; complément à 1 ; complément à 2.
  \end{block}

  \begin{block}{Tous les processeurs actuels utilisent le complément à 2.}
    \begin{itemize}
    \item il y a un seul code pour 0 ;
    \item l'addition de deux nombres se fait en additionnant leurs codes ;
    \item et il est très simple d'obtenir l'opposé d'un nombre.
    \item les processeurs disposent de fonctions spéciales pour l'implémenter.
    \end{itemize}

  \end{block}
\end{frame}




%%%%%%%%%%%%%%%%%%%%%%%%%%%%%%%%%%%%%%%%%%%%%%%%%%%
\begin{frame}[fragile]{Représentation des nombres réels}                                                                                                                
  \begin{itemize}
  %\item \R est l'ensemble des nombre réels.
  \item Les ressources d'un ordinateur étant limitées, on représente seulement un \alert{sous-ensemble $\F \subset \R$ de cardinal fini}.
  \item Les éléments de \F sont appelés \alert{nombres à virgule flottante}.
  \item<alert@1> Les propriétés de \F sont différentes de celles de \R.
  \item Généralement, un nombre réel $x$ est tronqué par la machine, définissant ainsi un nouveau nombre $fl(x)$ qui ne coïncide pas forcément avec le nombre $x$ original. 
  \end{itemize}

  \begin{alertblock}{Problèmes et limitations}
    \begin{itemize}
    \item les calculs sont nécessairement arrondis.
    \item il y a des erreurs d’arrondi et de précision
    \item On ne peut plus faire les opérations de façon transparente
  \end{itemize}
\end{alertblock}

\begin{lstlisting}
> 0.1 + 0.1 + 0.1 == 0.3
[1] FALSE
> 10^20 + 1 == 10^20
[1] TRUE
\end{lstlisting}
\end{frame}



%%% Local Variables:
%%% mode: latex
%%% TeX-master: "../representation-nombres"
%%% End:
