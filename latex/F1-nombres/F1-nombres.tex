\documentclass[10pt]{beamer}
%\documentclass[handout,10pt]{beamer} % printable version

\usepackage{../macros}

\usepackage{dsfont} %% ensembles \R, \F
\def\R{\ensuremath{\mathds{R}}\xspace}
\def\F{\ensuremath{\mathds{F}}\xspace}

\title{Représentation des nombres}
\author{Arnaud Malapert}

\hypersetup{
  pdfauthor = {Arnaud Malapert},
  pdftitle =  {Représentation des nombres},
}


\metroset{sectionpage=progressbar,subsectionpage=progressbar}


\begin{document}

\maketitle


\begin{frame}{Cours en AUTOFORMATION}
  
  \begin{block}{Représentation des nombres}
    \begin{itemize}
    \item Système positionel : binaire, décimal, octal, et héxadécimal.
    \item Nombre non signé seulement (positif).
    \item Représentation en machine des entiers naturels seulement !
    \end{itemize}
  \end{block}

  \begin{block}{Prérequis}
    Savoir additionner, soustraire, multiplier, et diviser ! Surtout par 2 !
  \end{block}

  \begin{alertblock}{Évaluation : Gagner un max de points en peu de temps !
}
    \begin{itemize}
    \item QCM sur Moodle comptant un peu pour le contrôle continu. 
    \item 3 points du même QCM dans le contrôle terminal.
    \item Exercices de programmation autour des algorithmes de conversion.
    \item Activité de programmation d'une appli web de conversion.
    \end{itemize}
  \end{alertblock}

\end{frame}


%%%%%%%%%%%%%%%%%%%%%%%%%%%%%%%%%%%%%%%%%%%%%%%%%%%%%%%
\begin{frame}{Table des matières}
  \setbeamertemplate{section in toc}[sections numbered]
  % \tableofcontents[hideallsubsections]
  \tableofcontents
\end{frame}


%%%%%%%%%%%%%%%%%%%%%%%%%%%%%%%%%%%%%%%%%%%%%%%%%%%%%%%
\section{Système positionnel}
%%%%%%%%%%%%%%%%%%%%%%%%%%%%%%%%%%%%%%%%%%%%%%%%%%%%%%%
\begin{frame}{Représentation des nombres}
  \begin{center}
    \alert{On représente les nombres grâce à des symboles.}
  \end{center}


  \begin{block}{Représentation unaire: un symbole de valeur unique.}
    \begin{itemize}
    \item I=1, II = 2, III = 3, Iet IIIIIIIII = 10.
    \item Le calcul est facile.
      \begin{itemize}
      \item       I + III = IIII;
      \item II $\times$ III = II II II = IIIIII
      \end{itemize}
    \item<alert@1> mais cela devient vite \emph{incompréhensible}
    \end{itemize}
  \end{block}

  \begin{block}{Chiffres Romains : plusieurs symboles ayant des valeurs différentes.}
    \begin{itemize}
    \item Le nombre de symboles est théoriquement infini.
      \begin{itemize}
      \item I=1, V=5, X = 10, L = 50 \dots.
      \end{itemize}

    \item<alert@1> Le calcul est impossible.
    \end{itemize}
  \end{block}

\end{frame}

%%%%%%%%%%%%%%%%%%%%%%%%%%%%%%%%%%%%%%%%%%%%%%%%%%%%%%%%%%%%%%


\begin{frame}{Représentation des nombres}
  \begin{alertblock}{Système positionnel : symboles dont la valeur dépend de la position}
    \begin{itemize}
    \item 999 = 900 + 90 + 9
    \item À Babylone, système sexagésimal (60) (IIe millénaire av J-C).
    \item Transmission de l'orient vers l'occident avec le zéro (env. 825~ap.~J-C)
      \footnote{\og Al-jabr wa’l-muqâbalah \fg  Muhammad ibn Müsä al-Khuwärizmï}
    \end{itemize}
  \end{alertblock}

  \begin{block}{Un brin de cynisme}
    \begin{quote}
      Les hommes sont comme les chiffres, ils n'acquièrent de la valeur que par leur position.
      \begin{flushright}
        Napoléon Bonaparte
      \end{flushright}
    \end{quote}
  \end{block}
\end{frame}


%%%%%%%%%%%%%%%%%%%%%%%%%%%%%%%%%%%%%%%%%%%%%%%%%%%%%%%%%%%%%%
\begin{frame}{Système positionnel}
  \begin{block}{Utilisation d'une base $b$}
  \begin{itemize}
  \item Les nombres sont représentés à l’aide de $b$ symboles distincts.
  \item La valeur d’un chiffre dépend de la base.
  \end{itemize}
  \end{block}

  \begin{alertblock}{Un nombre $x$ est représenté par une suite de symboles :}
    \alert{
      \[
        x = a_n a_{n-1} \ldots a_1 a_0 .
      \]
    }
  \end{alertblock}

    \begin{description}[Hexadécimale]
  \item[Décimale] $(b = 10),\ a_i \in \{0, 1, 2, 3, 4, 5, 6, 7, 8, 9\}$
  \item[Binaire] $(b = 2),\ a_i \in \{0, 1\}$
  \item[Hexadécimale] $(b = 16),\ a_i \in \{0, 1, 2, 3, 4, 5, 6, 7, 8, 9, A, B, C, D, E, F\}$
  \end{description}
  Les bases les plus utilisées sont : 10, 2, 3, $2^k$ , 12, 16, 60, $\frac{\sqrt{5}- 1}{2}$ \dots \\
  \begin{block}{Notation}
    $(x)_b$ indique que le nombre $x$ est écrit en base $b$.
  \end{block}

\end{frame}


\subsection{Entiers naturels}

%%%%%%%%%%%%%%%%%%%%%%%%%%%%%%%%%%%%%%%%%%%%%%%%%%%%
\begin{frame}{Représentation des entiers naturels}
 \begin{block}{En base $b$}
 \[
  x = a_n a_{n-1} \ldots a_1 a_0  = \sum_{i=0}^n a_i b^i
  \]
  \begin{itemize}
  \item $a_0$ est le chiffre de poids faible,
  \item $a_n$ est le chiffre de poids fort.
  \end{itemize}
\end{block}
\begin{exampleblock}{En base 10}
    \[
      (1998)_{10} = 1 \times 10^3 + 9 \times 10^2 + 9 \times 10^1 + 8 \times 10^0 .
    \]
\end{exampleblock}

\begin{exampleblock}{En base 2}
  \begin{align*}
    (101)_2 & = 1 \times 2^2 + 0 \times 2^1 + 1 \times 2^0 \\
            &= 4 + 0 + 1 = 5.
  \end{align*}

\end{exampleblock}
\end{frame}

%%%%%%%%%%%%%%%%%%%%%%%%%%%%%%%%%%%%%%%%%%%%%%%%%%%%
\begin{frame}{Traduction vers la base 10}
  \begin{block}{Méthode simple}
    \begin{itemize}
    \item On applique simplement la formule.
    \item Cela revient donc à une simple somme.
    \item En pratique, on peut utiliser la multiplication égyptienne.
    \end{itemize}
  \end{block}

  \begin{block}{Schéma de Horner}
    \begin{itemize}
    \item Méthode générale pour calculer l'image d'un polynôme en un point.
    \item Moins d'opérations que la méthode simple.
    \item Plus efficace pour une machine, pas nécessairement pour un humain.
    \end{itemize}

  \end{block}
\end{frame}


%%%%%%%%%%%%%%%%%%%%%%%%%%%%%%%%%%%%%%%%%%%%%%%%%%%%
\begin{frame}{Schéma de Horner}
  \only<1>{
    \begin{block}{Une reformulation judicieuse de l'écriture en base $b$}
      \begin{align*}
      a_n a_{n-1} \ldots a_1 a_0  & = \sum_{i=0}^n a_i b^i \\
                                  & = ((\ldots ((a_ nb + a_{n-1})b + a_{n-2})\ldots)b + a_{1})b + a_0
      \end{align*}
    \end{block}
  }

  \begin{alertblock}{Algorithme simple et efficace}
  \begin{itemize}
  \item   Initialiser l'accumulateur : $v = 0$.
  \item Pour chaque chiffre $a_i$ en partant de la gauche : $ v \leftarrow (v \times b) + a_i$.
  \end{itemize}
\end{alertblock}

\only<2>{
\begin{columns}[t]
\begin{column}{0.48\textwidth}
  \begin{exampleblock}{$(10110)_2 = (22)_{10}$}
    \begin{tabular}{lll}
      $a_i$ & $v$  & Calcul de $v$     \\
      1     & 1    & $2 \times 0 + 1$  \\
      0     & $2$  & $2 \times 1 + 0$  \\
      1     & $5$  & $2 \times 2 + 1$  \\
      1     & $11$ & $2 \times 5 + 1$  \\
      0     & $22$ & $2 \times 11 + 0$ \\
    \end{tabular}
\end{exampleblock}

\end{column}
\begin{column}{0.48\textwidth}
  \begin{exampleblock}{ $(12021)_3 = (142)_{10}$}
    \begin{tabular}{lll}
      $a_i$ & $v$   & Calcul de $v$     \\
      1     & 1     & $3 \times 0 + 1$  \\
      2     & $5$   & $3 \times 1 + 2$  \\
      0     & $15$  & $3 \times 5 + 0$  \\
      2     & $47$  & $3 \times 15 + 2$ \\
      1     & $142$ & $3 \times 47 + 1$ \\
    \end{tabular}
\end{exampleblock}
\end{column}
\end{columns}
}


\end{frame}


% %%%%%%%%%%%%%%%%%%%%%%%%%%%%%%%%%%%%%%%%%%%%%%%%%%%%
\begin{frame}{Traduction entre des puissances de 2}
  \begin{block}{Du binaire vers une base $2^k$}
    Regrouper les bits par $k$ en partant de la droite et les traduire.
  \end{block}

   \begin{block}{D'une base $2^k$ vers le binaire}
     Traduire chacun des symboles en un nombre binaire.
   \end{block}

  \begin{exampleblock}{Du binaire vers l'héxadécimal ($2^4$)}
    Regrouper les bits par 4 en partant de la droite.
    \begin{tabular}{*{6}{c}}
      (00)10 & 0101 & 1110 & 0001 & 1101 & 1111 \\
      2      & 5    & E    & 1    & D    & F
    \end{tabular}
  \end{exampleblock}

  \begin{exampleblock}{Du binaire vers l'octal ($2^3$)}
    Regrouper les bits par 3 en partant de la droite.
    \begin{tabular}{*{8}{c}}
     (00)1 & 001 & 011 & 110 & 000 & 111 & 011 & 111 \\
      1    & 1   & 3   & 6   & 0   & 7   & 3   & 7
    \end{tabular}
  \end{exampleblock}


   \begin{block}{D'une base $2^k$ vers une base $2^p$}
     \begin{enumerate}
     \item<alert@1> Passer par le binaire : $(x)_{2^k} \rightarrow (x)_2 \rightarrow (x)_{2^p}$.
     \item Généraliser la méthode précédente si $k$ est un diviseur/multiple de $p$.
     \item Appliquer un algorithme de traduction vers une base quelquonque.
     \end{enumerate}
   \end{block}

 \end{frame}



%%%%%%%%%%%%%%%%%%%%%%%%%%%%%%%%%%%%%%%%%%%%%%%%%%%%
\begin{frame}{Traduction vers une base quelquonque}

  \begin{block}{Nombre entier}
    On procède par divisions euclidiennes successives:
    \begin{itemize}
    \item On divise le nombre par la base,
    \item puis le quotient par la base,
    \item ainsi de suite jusqu’à obtenir un quotient nul.
    \end{itemize}
    La suite des restes obtenus correspond aux chiffres de $a_0$ à $a_n$ dans la base visée.
  \end{block}


      \begin{columns}[t]
        \begin{column}{0.48\textwidth}
          \begin{exampleblock}{$(44)_{10} = (101100)_2$}
        \begin{tabular}{r@{ = }ll}
         $44$ & $22\times 2 + 0$ & $ a_0 = 0$ \\
         $22$ & $11\times 2 + 0$ & $a_1 = 0$ \\
         $11$ & $5\times 2 + 1$ & $a_2 = 1$ \\
         $5$ & $2\times 2 + 1$ & $a_3 = 1$ \\
         $2$ & $1\times 2 + 0$ & $a_4 = 0$ \\
         $1$ & $0\times 2 + 1$ & $a_5 = 1$ \\
        \end{tabular}
      \end{exampleblock}
      \end{column}


  \begin{column}{0.48\textwidth}
    \begin{exampleblock}{$(44)_{10} = (1122)_3$}
      \begin{tabular}{r@{ = }ll}
         $44$ & $14 \times 3 + 2$ & $ a_0 = 2$ \\
         $14$ & $4\times 3 + 2$ & $a_1 = 2$ \\
         $4$ & $1\times 3 + 1$ & $a_2 = 1$ \\
         $1$ & $0\times 3 + 1$ & $a_3 = 1$ \\
      \end{tabular}
    \end{exampleblock}
  \end{column}
  \end{columns}

\end{frame}


%%%%%%%%%%%%%%%%%%%%%%%%%%%%%%%%%%%%%%%%%%%%%%%%%%%%%

\begin{frame}{Traduction depuis une base quelquonque}

  \begin{block}{Il faut diviser dans la base d'origine.}
    Les calculs sont donc difficiles pour un humain !
  \end{block}

    \begin{exampleblock}{$(101100)_2 = (1122)_3$}
      \begin{tabular}{r@{ = }ll}
         $101100$ & $1110 \times 11 + 10$ & $ a_0 = 2$ \\
         $1110$ & $100 \times 11 + 10$ & $a_1 = 2$ \\
         $100$ & $1\times 11 + 1$ & $a_2 = 1$ \\
         $1$ & $0\times 11 + 1$ & $a_3 = 1$ \\
      \end{tabular}
    \end{exampleblock}

    \begin{alertblock}{Passez par le décimal !}
      $ (x)_b \rightarrow (x)_{10} \rightarrow (x)_{b^{\prime}}$.
    \end{alertblock}


\end{frame}


%%%%%%%%%%%%%%%%%%%%%%%%%%%%%%%%%%%%%%%%%%%%%%%%%%%%
  \begin{frame}{Exercices}
   \begin{itemize}
   \item Traduire $(10101)_2$ en écriture décimale.
   \item Traduire $(10101101)_2$ en écriture décimale.
   \item Traduire $(10101001110101101)_2$ en écriture hexadécimale.
   \item Traduire $(1AE3F)_{16}$ en écriture binaire.
   \item Traduire $(1AE3F)_{16}$ en écriture octal.
   \item Traduire $(927)_{10}$ en écriture binaire.
   \item Traduire $(1316)_{10}$ en écriture binaire.
  \end{itemize}
\end{frame}


%%%%%%%%%%%%%%%%%%%%%%%%%%%%%%%%%%%%%%%%%%%%%%%%%%%%
\subsection{Nombres fractionnaires}

%%%%%%%%%%%%%%%%%%%%%%%%%%%%%%%%%%%%%%%%%%%%%%%%%%%%
\begin{frame}{Représentation des nombres fractionnaires}
  \begin{block}{En base $b$}
    La formule est la même, mais il existe des exposants négatifs.
    \[
      x = a_n a_{n-1} \ldots a_1 a_0,a_{-1}a_{-2}\ldots a_{-k} = \sum_{\alert{i=-k}}^n a_i b^i
    \]
  \end{block}

  \begin{exampleblock}{En base 10}
    \[
      19.98 = 1 \times 10^1 + 9 \times 10^0 + 9 \times 10^{-1} + 8 \times 10^{-2} .
    \]
  \end{exampleblock}
  \begin{exampleblock}{En base 2}
    \begin{align*}
      (101,01)_2 &= 1 \times 2^2 + 0 \times 2^1 + 1 \times 2^0 +  0 \times 2^{-1} + 1 \times 2^{-2} \\
      &= 4  + 0 + 1 + 0 + 0.25 = 5,25.
    \end{align*}
  \end{exampleblock}

\end{frame}

% %%%%%%%%%%%%%%%%%%%%%%%%%%%%%%%%%%%%%%%%%%%%%%%%%%%%
\begin{frame}{Traduction vers une base quelquonque}
    \begin{block}{Nombre fractionnaire}
        \begin{itemize}
        \item On décompose le nombre en partie entière et fractionnaire si $x > 0$ :
          $$x = E[x] + F[x].$$
        \item On convertit la partie entière par la méthode précédente :
          $$E[x] =  a_n a_{n-1} \ldots a_1 a_0.$$
        \item On convertit la partie fractionnaire :
          $$F[x] = 0,a_{-1}a_{-2}\dots a_{-m}.$$
        \item Finalement, on additionne la partie entière et fractionnaire :
          $$x =  a_n a_{n-1} \ldots a_1 a_0,a_{-1}a_{-2}\dots a_{-m}\dots.$$
       \end{itemize}
  \end{block}

\end{frame}

% %%%%%%%%%%%%%%%%%%%%%%%%%%%%%%%%%%%%%%%%%%%%%%%%%%%%
\begin{frame}{Traduction vers une base quelquonque}
  \begin{block}{Partie fractionnaire}
    \begin{itemize}
    \item on multiplie $F[x]$ par $b$. Soit $a_{-1}$ la partie entière de ce produit,
    \item on recommence avec la partie fractionnaire du produit pour obtenir $a_{-2}$,
    \item et ainsi de suite \dots
    \item on stoppe l’algorithme si la partie fractionnaire devient nulle.
    \end{itemize}
  \end{block}

  \begin{exampleblock}{$(0,734375)_{10} = (0,BC)_{16}$}
    \begin{tabular}{r@{ = }ll}
      $0,734375 \times 16$ & $11,75$ & $a_{-1} = B$ \\
      $0,75 \times 16$     & $12$    & $a_{-2} = C$
    \end{tabular}
    % \begin{itemize}
    % \item $0, 734375\times16 = 11, 75 \Rightarrow a_{-1} = B$
    % \item $ 0, 75\times16 = 12 \Rightarrow a_{-2} = C$
    % \item Alors $(0,734375)_{10} = (0,BC)_{16}$
    % \end{itemize}
  \end{exampleblock}
\end{frame}

%%%%%%%%%%%%%%%%%%%%%%%%%%%%%%%%%%%%%%%%%%%%%%%%%%%%
\begin{frame}{Traduction de 0,3 en base 2}


  \begin{exampleblock}{$(0,3)_{10} = (0,01001100110011 \ldots)_2$}
    \begin{tabular}{r@{ = }ll}

      $0,3\times2$         & $0,6$ & $a_{-1} = 0$ \\
      $0,6\times2$         & $1,2$ & $a_{-2} = 1$ \\
      \alert{$0,2\times2$} & $0,4$ & $a_{-3} = 0$ \\
      $0,4\times2$         & $0,8$ & $a_{-4} = 0$ \\
       $0,8\times2$        & $1,6$ & $a_{-5} = 1$ \\
       $0,6\times2$        & $1,2$ & $a_{-6} = 1$ \\
      \alert{$0,2\times2$} & $0,4$ & $a_{-7} = 0$ \\

      %\dots \\
    \end{tabular}
  \end{exampleblock}

  \begin{alertblock}{La conversion d’un nombre fractionnaire ne s’arrête pas toujours.}
    \begin{itemize}
    \item En base $b$,  on ne peut représenter exactement que des nombres fractionnaires de la forme $X/b^k$
    \item  La conversion d’un nombre entier s’arrête toujours.
    \end{itemize}
  \end{alertblock}
  \begin{alertblock}{Il faudra arrondir \dots}
  \end{alertblock}
\end{frame}



% %%%%%%%%%%%%%%%%%%%%%%%%%%%%%%%%%%%%%%%%%%%%%%%%%%%%
\begin{frame}{Exercices}
  \begin{itemize}
  \item Traduire $(10101101)_2$ en écriture décimale.
  \item Traduire $(10101001110101101)_2$ en écriture hexadécimale.
  \item Traduire $(1AE3F)_{16}$ en écriture décimale.
  \item Traduire $(1AE3F)_{16}$ en écriture binaire.
  \item Traduire $(13.1)_{10}$ en écriture binaire.
  \end{itemize}
\end{frame}


%%% Local Variables:
%%% mode: latex
%%% TeX-master: "../F1-nombres"
%%% End:


%%%%%%%%%%%%%%%%%%%%%%%%%%%%%%%%%%%%%%%%%%%%%%%%%%%%%%%
\section{Multiplication et division égyptiennes}
%%%%%%%%%%%%%%%%%%%%%%%%%%%%%%%%%%%%%%%%%%%%%%%%%%%%%%%
\begin{frame}{Multiplication égyptienne}
  \begin{block}{Principe}
    \begin{itemize}
    \item Décomposition d'un des nombres en une somme.\\
      On décompose généralement le plus petit.
    \item Création d'une table de puissance pour l'autre nombre
    \item Très souvent, traduction du décimal vers le binaire.\\
      Il existe des variantes en fonction de la complexité de l'opération.
    \item Il suffit de savoir multiplier par deux et additionner !
    \end{itemize}
  \end{block}


\end{frame}



\begin{frame}{Multiplication égyptienne : $x \times y$}

  \begin{itemize}
  \item On construit la ligne $i+1$ en multipliant par deux $2^i$ et $x \times 2^i$ tant que $2^{i+1}< y$.
  \item<2> On traduit $21$ en binaire en remontant dans le tableau. 
  \item<2> On calcule $\sum_0^{n} a_i (x \times  2^i)$.
  \end{itemize}

\begin{exampleblock}{$189\times 21=3969$}
      \begin{tabular}{c|c|r|rcc}

        $i$ & $a_i$           & $2^i$ & $189 \times 2^i$                               \\
        \cline{1-4}
        0   & \onslide<2->{1} & 1     & 189  & \onslide<2->{\checkmark & (1 -1 = 0)}   \\
        1   & \onslide<2->{0} & 2     & 378  &                                         \\
        2   & \onslide<2->{1} & 4     & 756  & \onslide<2->{\checkmark & (5 -4 = 1)}   \\
        3   & \onslide<2->{0} & 8     & 1512 &                                         \\
        4   & \onslide<2->{1} & 16    & 3024 & \onslide<2->{\checkmark & (21 -16 = 5)} \\
        \cline{1-4}
            &                 &       & \onslide<2>{3969}

      \end{tabular}
\end{exampleblock}
\end{frame}

%%%%%%%%%%%%%%%%%%%%%%%%%%%%%%%%%%%%%%%%%%%%%%%%%%%%%%%%
\begin{frame}
\frametitle{Méthode genérale}

Des opérations plus complexes faisant intervenir par exemple des fractions exigeaient une décomposition avec:
\begin{itemize}
\item  les puissances de deux,
\item les fractions fondamentales,
\item les dizaines. 
\end{itemize}
La technique est rigoureusement la même mais offre plus de liberté au scribe quant à la décomposition du petit nombre.
\begin{columns}
\begin{column}[t]{0.48\textwidth}
  \begin{exampleblock}{$243\times27 = 6561$}
    \begin{tabular}{r|r}
      1 & 243 \\
      2 & 486 \\
      4 & 972 \\
      20 & 4860 \\ \hline
      27 & 6561
      \end{tabular}    
  \end{exampleblock}
\end{column}

\begin{column}[t]{0.48\textwidth}
  \begin{exampleblock}{Papyrus Rhind: $\frac{1}{14} \times \frac{7}{4} = \frac{1}{8}$}
    \begin{tabular}{r|rl}
      $1$ & $\frac{1}{14}$& \\[3pt]
      $\frac{1}{2}$ & $\frac{1}{28}$& \\[3pt]
      $\frac{1}{4}$ & $\frac{1}{56}$& \\[3pt] \cline{1-2} \\[-1.0em] 
      $\frac{7}{4}$ & $\frac{1}{8}$& $(\frac{4+2+1}{56})$ 
    \end{tabular}
  \end{exampleblock}
\end{column}
\end{columns}

\end{frame}


%%%%%%%%%%%%%%%%%%%%%%%%%%%%%%%%%%%%%%%%%%%%%%%%%%%%%%%%
\begin{frame}{Division égyptienne : $x \div y$}
  \begin{block}{Par quoi doit-on multiplier $y$ pour trouver $x$ ?}
    \begin{itemize}
    \item On construit la ligne $i+1$ en multipliant par deux $2^i$ et $y\times 2^i$ tant que $y \times 2^{i+1}< x$.
    \item<2> On décompose $x$ par les $y\times 2^i$ en remontant dans le tableau. 
    \item<2> On calcule la somme des $2^i$ de la décomposition.

    \end{itemize}
  \end{block}


\begin{exampleblock}{$539\div 7=77$}
  \begin{tabular}{c|r|rcc}
    $i$ & $2^i$ & $7 \times 2^i$& \\
    \cline{1-3}
    0& 1&7  & \onslide<2>{\checkmark & ($7-7=0$)}\\
    1& 2&14 & \\
    2& 4&28 &\onslide<2>{\checkmark & ($35-28=7$)}\\
    3& 8&56 & \onslide<2>{\checkmark & ($91-56=35$)}\\
    4& 16&112 & \\
    5& 32&224 & \\        
    6& 64&448 & \onslide<2>{\checkmark & ($539-448=91$)}\\  
    \cline{1-3}
    &  \onslide<2->{77} & 
  \end{tabular}
\end{exampleblock}
\end{frame}

%%%%%%%%%%%%%%%%%%%%%%%%%%%%%%%%%%%%%%%%%%%%%%%%%%%%
\begin{frame}{Division dont le résultat est fractionnaire}

\begin{exampleblock}{$234\div 12=19.5$}
  \begin{tabular}{r|r|rcc}
    $i$ & $2^i$ & $12 \times 2^i$& \\
    \cline{1-3}
    \onslide<3->{-1& $\frac{1}{2}$ & 6  & \checkmark & ($0$)}\\
    0& 1&12  & \onslide<2->{\checkmark & ($6$)}\\
    1& 2&24 & \onslide<2->{\checkmark & ($18$)}\\
    2& 4&48 &\\
    3& 8&96 & \\
    4& 16&192 & \onslide<2->{\checkmark & ($42$)}\\
    \cline{1-3}
    &  \onslide<3->{19.5} & 
  \end{tabular}
\end{exampleblock}
\end{frame}


%%%%%%%%%%%%%%%%%%%%%%%%%%%%%%%%%%%%%%%%%%%%%%%%%%%%%%%%
\begin{frame}{Exercices}
  
  \begin{enumerate}
  \item Multipliez 187 par 11.
  \item Multipliez 2012 par 1515 (indice: utilisez la méthode genérale).
  \end{enumerate}
\begin{columns}
\begin{column}[t]{0.48\textwidth}
  \begin{exampleblock}<2->{$187\times 11$}
    \begin{tabular}{c|c|r|r}
        $i$ & $a_i$ & $2^i$ & $21 \times 2^i$ \\
        \hline
        0& 1 & 1 & 187\\
        1& 1 & 2 & 374\\
        2& 0 & 4 & 748\\
        3& 1 & 8 & 1496\\ \hline
        &  & 11 &2057
      \end{tabular}    
  \end{exampleblock}
\end{column}
\begin{column}[t]{0.48\textwidth}
    \begin{exampleblock}<2>{$2012\times 1515$}
      \begin{tabular}{r|r}
        5& 10060 \\
        10& 20120 \\
        500 & 1006000 \\
        1000 & 2012000\\ \hline
        1515 & 3048180 
      \end{tabular}
  \end{exampleblock}
\end{column}
\end{columns}

\end{frame}


%%% Local Variables:
%%% mode: latex
%%% TeX-master: "../representation-nombres"
%%% End:



%%%%%%%%%%%%%%%%%%%%%%%%%%%%%%%%%%%%%%%%%%%%%%%%%%%%%%%
\section{Arithmétique binaire}
%%%%%%%%%%%%%%%%%%%%%%%%%%%%%%%%%%%%%%%%%%%%%%%%%%%%%%%
%%%%%%%%%%%%%%%%%%%%%%%%%%%%%%%%%%%%%%%%%%%%%%%%%%%%
\newcommand{\carry}{\scriptsize{1}}

\begin{frame}{Représentation de l'information en machine}
  \begin{itemize}
  \item Informations en général représentées et manipulées sous forme binaire.
  \item L’unité d’information est le chiffre binaire ou bit (binary digit).
  \item Les opérations arithmétiques de base sont faciles à exprimer en base~2.
  \item La représentation binaire est facile à réaliser : systèmes à deux états obtenus à l’aide de transistors.
  \end{itemize}
\end{frame}



%%%%%%%%%%%%%%%%%%%%%%%%%%%%%%%%%%%%%%%%%%%%%%%%%%%%
\begin{frame}{Addition binaire}
    \begin{block}{Tables d'addition}
    \begin{itemize}
    \item $0 + 0 = 0$
    \item $1 + 0 = 1$
    \item $0 + 1 = 1$
    \item $1 + 1 = 10$ ($0$ et on retient $1$)
    \end{itemize}
  \end{block}

  \begin{exampleblock}{$91+71=162$}
    \begin{tabular}{c*{7}{@{\,}c}}
      \carry &   & \carry & \carry & \carry & \carry & \carry &   \\
             & 1 & 0      & 1      & 1      & 0      & 1      & 1 \\
      +      & 1 & 0      & 0      & 0      & 1      & 1      & 1 \\
      \hline
      1      & 0 & 1      & 0      & 0      & 0      & 1      & 0 
    \end{tabular}
  \end{exampleblock}

\end{frame}

%%%%%%%%%%%%%%%%%%%%%%%%%%%%%%%%%%%%%%%%%%%%%%%%%%%%
\begin{frame}{Soustraction binaire}
  \begin{block}{Tables de soustraction}
    \begin{itemize}
    \item $0 - 0 = 0$
    \item $1 - 0 = 1$
    \item $(1)0 - 1 = 1$ ($1$ et on retient $1$)
    \item $1 - 1 = 0$
    \end{itemize}
  \end{block}

  \begin{exampleblock}{$83- 79=4$}
    \begin{tabular}{c*{7}{@{\,}c}}
        &   &   &        & \carry & \carry &   &   \\
        & 1 & 0 & 1      & 0      & 0      & 1 & 1 \\
      - & 1 & 0 & 0      & 1      & 1      & 1 & 1 \\
        &   &   & \carry & \carry &        &   &   \\
      \hline
        & 0 & 0 & 0      & 0      & 1      & 0 & 0 
    \end{tabular}
  \end{exampleblock}

  \begin{alertblock}{Limitation: résultat négatif}
    On ne peut traiter $x - y$ que si $x \geq y$. 
  \end{alertblock}
\end{frame}

%%%%%%%%%%%%%%%%%%%%%%%%%%%%%%%%%%%%%%%%%%%%%%%%%%%%
\begin{frame}{Multiplication binaire}
  \begin{block}{Tables de multiplication}
    \begin{itemize}
    \item $0 \times 0 = 0$
    \item $1 \times 0 = 0$
    \item $1 \times 1 = 1$
    \end{itemize}
  \end{block}
  \begin{exampleblock}{$23\times11=253$}
    \begin{tabular}{c*{9}{@{\,}c}}
               &   &        &        & 1      & 0      & 1      & 1 & 1 \\
      $\times$ &   &        &        &        & 1      & 0      & 1 & 1 \\
      \hline
               &   & \carry & \carry & \carry & \carry & \carry &   &   \\
               &   &        &        & 1      & 0      & 1      & 1 & 1 \\
      +        &   &        & 1      & 0      & 1      & 1      & 1 &   \\
      +        & 1 & 0      & 1      & 1      & 1      &        &   &   \\\hline
               & 1 & 1      & 1      & 1      & 1      & 1      & 0 & 1 
    \end{tabular}
  \end{exampleblock}
\end{frame}

%%%%%%%%%%%%%%%%%%%%%%%%%%%%%%%%%%%%%%%%%%%%%%%%%%%%
\begin{frame}{Division binaire}
  \begin{block}{Soustractions et décalages comme la division décimale}
    \begin{itemize}
    \item sauf que les digits du quotient ne peuvent être que 1 ou 0.
    \item Le bit du quotient est 1 si on peut soustraire le diviseur, sinon il est~0. 
    \end{itemize}
    
  \end{block}
  \begin{exampleblock}{$231 \div 11 = 21$}
        \begin{tabular}{c*{8}{@{\,}c}|l}
          & 1 & 1   & 1 & 0   & 0 & 1 & 1 & 1 & 1011  \\ \cline{10-10}
      $-$ & 1 & 0   & 1 & 1   &   &   &   &   & 10101 \\ \cline{2-5}
          &   &     & 1 & 1   & 0 & 1 &   &   &       \\
          &   & $-$ & 1 & 0   & 1 & 1 &   &   &       \\\cline{4-7}
          &   &     &   &     & 1 & 0 & 1 & 1 &       \\
          &   &     &   & $-$ & 1 & 0 & 1 & 1 &       \\\cline{6-9}
          &   &     &   &     &   &   &   & 0 &       \\
      
    \end{tabular}
  \end{exampleblock}

  \begin{alertblock}{Limitation: division entière}
    Pour l'instant, on ne peut pas calculer la partie fractionnaire.
  \end{alertblock}

\end{frame}


%%%%%%%%%%%%%%%%%%%%%%%%%%%%%%%%%%%%%%%%%%%%%%%%%%%%
\begin{frame}{Exercices}
  \begin{enumerate}
  \item Traduire $(1100101)_2$ et $(10101111)_2$ en écriture décimale.
  \item Calculez la somme $(1100101)_2 + (10101111)_2$.
  \item Calculez le produit $(10101111)_2 \times (1100101)$.
 \item Calculez la soustraction $(10101111)_2 - (1100101)_2$.
 \item Vérifier tous les résultat en les traduisant en écriture décimale.  
 \end{enumerate}

\end{frame}

%%% Local Variables:
%%% mode: latex
%%% TeX-master: "../representation-nombres"
%%% End:



%%%%%%%%%%%%%%%%%%%%%%%%%%%%%%%%%%%%%%%%%%%%%%%%%%%%%%%
\section{Représentation des nombres en machine}
%%%%%%%%%%%%%%%%%%%%%%%%%%%%%%%%%%%%%%%%%%%%%%%%%%%%%%%
%%%%%%%%%%%%%%%%%%%%%%%%%%%%%%%%%%%%%%%%%%%%%%%%%%%%
\begin{frame}{Représentation des nombres en machine}

  \begin{block}{Précision finie}
    \begin{itemize}
    \item Codés généralement sur 16, 32 ou 64 bits.
    \item Un codage sur $n$ bits permet de représenter $2^n$ valeurs distinctes.
    \end{itemize}
  \end{block}
  
  \begin{alertblock}{Un ordinateur ne calcule pas bien !}
  \begin{itemize}
   \item Pour un ordinateur, le nombre de chiffres est fixé.
   \item Pour un mathématicien, le nombre de chiffres dépend de la valeur représentée.
   \item Lorsque le résultat d’un calcul doit être représenté sur plus de chiffres que ceux disponibles, il y a dépassement de capacité.
  \end{itemize}
\end{alertblock}

\end{frame}


\begin{frame}{Représentation des entiers en machine}
  \begin{block}{Entiers naturels}
    \begin{itemize}
    \item Un codage sur $n$ bits : tous les entiers entre 0 et $2^n-1$.
    \item La conversion d’un nombre entier s’arrête toujours.
    \item On ne peut traiter $x - y$ que si $x \geq y$. 
    \end{itemize}
  \end{block}

  \begin{block}{Entiers relatifs : plusieurs représentations existent.}
    Valeur absolue signée ; complément à 1 ; complément à 2.
  \end{block}

  \begin{block}{Tous les processeurs actuels utilisent le complément à 2.}
    \begin{itemize}
    \item il y a un seul code pour 0 ;
    \item l'addition de deux nombres se fait en additionnant leurs codes ;
    \item et il est très simple d'obtenir l'opposé d'un nombre.
    \item les processeurs disposent de fonctions spéciales pour l'implémenter.
    \end{itemize}

  \end{block}
\end{frame}




%%%%%%%%%%%%%%%%%%%%%%%%%%%%%%%%%%%%%%%%%%%%%%%%%%%
\begin{frame}[fragile]{Représentation des nombres réels}                                                                                                                
  \begin{itemize}
  %\item \R est l'ensemble des nombre réels.
  \item Les ressources d'un ordinateur étant limitées, on représente seulement un \alert{sous-ensemble $\F \subset \R$ de cardinal fini}.
  \item Les éléments de \F sont appelés \alert{nombres à virgule flottante}.
  \item<alert@1> Les propriétés de \F sont différentes de celles de \R.
  \item Généralement, un nombre réel $x$ est tronqué par la machine, définissant ainsi un nouveau nombre $fl(x)$ qui ne coïncide pas forcément avec le nombre $x$ original. 
  \end{itemize}

  \begin{alertblock}{Problèmes et limitations}
    \begin{itemize}
    \item les calculs sont nécessairement arrondis.
    \item il y a des erreurs d’arrondi et de précision
    \item On ne peut plus faire les opérations de façon transparente
  \end{itemize}
\end{alertblock}

\begin{lstlisting}
> 0.1 + 0.1 + 0.1 == 0.3
[1] FALSE
> 10^20 + 1 == 10^20
[1] TRUE
\end{lstlisting}
\end{frame}



%%% Local Variables:
%%% mode: latex
%%% TeX-master: "../representation-nombres"
%%% End:



%%%%%%%%%%%%%%%%%%%%%%%%%%%%%%%%%%%%%%%%%%%%%%%%%%%%%%%
 \questionSlide

%%%%%%%%%%%%%%%%%%%%%%%%%%%%%%%%%%%%%%%%%%%%%%%%%%%%%%%
 \appendix
 \backupSlides
%%%%%%%%%%%%%%%%%%%%%%%%%%%%%%%%%%%%%%%%%%%%%%%%%%%%%%%

%%%%%%%%%%%%%%%%%%%%%%%%%%%%%%%%%%%%%%%%%%%%%%%%%%%%%%%
% \begin{frame}[fragile]{Backup slides}
%   Sometimes, it is useful to add slides at the end of your presentation to
%   refer to during audience questions.

%   The best way to do this is to include the \verb|appendixnumberbeamer|
%   package in your preamble and call \verb|\appendix| before your backup slides.

%   will automatically turn off slide numbering and progress bars for
%   slides in the appendix.
% \end{frame}


%%%%%%%%%%%%%%%%%%%%%%%%%%%%%%%%%%%%%%%%%%%%%%%%%%%%%%%
% \begin{frame}[allowframebreaks]{References}

%   % \bibliography{../bib_parallelism,../bib_others}
%   % \bibliographystyle{abbrv}

% \end{frame}

\end{document}

%%% Local Variables:
%%% mode: latex
%%% TeX-master: t
%%% End:
