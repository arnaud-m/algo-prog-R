\documentclass[10pt]{beamer}

\usepackage{../macros}
\title{Nombres premiers}

\hypersetup{
  pdftitle =  {Nombres premiers}
}

\begin{document}

\maketitle



%%%%%%%%%%%%%%%%%%%%%%%%%%%%%%%%%%%%%%%%%%%%%%%%%%%%%%%
\begin{frame}[fragile]
  \frametitle{Qu'est-ce qu'un nombre premier ?}
  \begin{alertblock}{Nombre premier}
    \alert{Un nombre premier est un entier naturel qui admet exactement deux diviseurs distincts entiers et positifs.}\\
    \begin{itemize}
    \item Ces deux diviseurs sont 1 et le nombre considéré, puisque tout nombre a pour diviseurs 1 et lui-même (comme le montre l’égalité $ n=1\times n$).
    \item Les nombres premiers étant ceux qui n’en possèdent aucun autre.
    \item Selon cette définition, les nombres 0 et 1 ne sont pas premiers.
    \end{itemize}
    \begin{flushright}
      Wikipedia.
    \end{flushright}
  \end{alertblock}

  \begin{center}
    \textbf{\alert{Calculons le vecteur des nombres premiers inférieurs à $n$!}}
  \end{center}


\end{frame}



\begin{frame}{Détermination des nombres premiers inférieurs à $n$}
  Comme d'habitude, on va construire une itération.
  \begin{block}{Passer à l'étape suivante (\alert{ITÉRATION})}
    Étant en possession de $\mathtt{prems}$, un vecteur contenant les nombres premiers de $[0, \mathtt{d}]$.
    \begin{itemize}
    \item Il suffira donc d'ajouter 1 à $\mathtt{d}$,
    \item puis d'ajouter $\mathtt{d}$ à \texttt{prems} si \texttt{d} est premier (\alert{test de primalité}).
    \end{itemize}

  \end{block}
  \begin{block}{Détecter si le calcul est terminé (\alert{TERMINAISON})}
    On aura terminé lorsque $\mathtt{d}$ sera égal à $n$ puisqu'alors $\mathtt{prems}$ contiendra tous les nombres premiers inférieurs à $n$.
  \end{block}

  \begin{block}{Trouver les valeurs initiales des variables (\alert{INITIALISATION})}
    Au début du calcul, je peux prendre $\mathtt{prems}$ vide et $\mathtt{d} =0$. \\
  \end{block}
\end{frame}

%%%%%%%%%%%%%%%%%%%%%%%%%%%%%%%%%%%%%%%%%%%%%%%%%%%%%%%
\begin{frame}[fragile]
  \frametitle{Test de primalité I}
  Pour savoir si un nombre $n \geq 2$ est premier, il suffit donc d'examiner les nombres entiers $d$ de $[2,n-1]$ à la recherche d'un diviseur de $n$.
  \begin{itemize}
  \item Si l'on en trouve un, on s'arrête au premier avec le résultat \texttt{FALSE}.
  \item Sinon, le résultat sera \texttt{TRUE}.
  \end{itemize}
  \begin{lstlisting}[style=editor]
EstPremier <- function(n) { # version I
  if(n < 2) return(FALSE) # 0 et 1 ne sont pas premiers
  d <- 2 #le premier diviseur non trivial
  while( d < n) {
    if( n %% d == 0) return(FALSE) # Échappement
    d <- d + 1
  }
  return(TRUE)
}
\end{lstlisting}


\begin{columns}[t]
\begin{column}{0.4\textwidth}
  \begin{lstlisting}
> EstPremier(1003)
[1] FALSE
> EstPremier(2003)
[1] TRUE
  \end{lstlisting}
\end{column}
\begin{column}{0.6\textwidth}

\begin{block}{Comment améliorer ce test ?}
  Il est appelé $n-1$ fois pour déterminer les nombres premiers inférieurs à $n$
\end{block}
\end{column}
\end{columns}

\end{frame}


%%%%%%%%%%%%%%%%%%%%%%%%%%%%%%%%%%%%%%%%%%%%%%%%%%%%%%%
\begin{frame}[fragile]
  \frametitle{Test de primalité II}
  En fait, il suffit \alert{de tester la parité et d'examiner les nombres entiers impairs dans $\left[3,\sqrt{n}\right]$} à la recherche d'un diviseur de $n$.

  \begin{lstlisting}[style=editor]
EstPremier <- function(n) { # version II
  if(n < 2) return(FALSE) # 0 et 1 ne sont pas premiers
  if( n %% 2 == 0) return(ifelse( n == 2, TRUE, FALSE)) # test de parité
  d <- 3 # le deuxième diviseur non trivial
  m <- floor(sqrt(n)) # on calcule la racine une seule fois
  while( d <= m) { # on compare avec la racine
    if( n %% d == 0) return(FALSE) # Échappement
    d <- d + 2 # on itére sur les nombres impairs
  }
  return(TRUE)
}
\end{lstlisting}

\end{frame}


%%%%%%%%%%%%%%%%%%%%%%%%%%%%%%%%%%%%%%%%%%%%%%%%%%%%%%%
\begin{frame}[fragile]
  \frametitle{Performance des tests de primalité }

  Comparons les performances des deux versions du test de primalité  sur des nombres de Mersenne premiers. \\
  Les nombres de Mersenne sont de la forme : une puissance de 2 moins 1.

  \begin{lstlisting}
> system.time(replicate(50, EstPremier(2**19 - 1)))
utilisateur     système      écoulé
      0.018       0.000       0.019
  \end{lstlisting}

  \begin{table}[ht]
    \centering
    \begin{tabular}{l|rr}
      \toprule
      \texttt{n} & version I & version II \\
      \midrule
      $2^{13}-1$ & 152       & 2          \\
      $2^{17}-1$ & 1145      & 13         \\
      $2^{19}-1$ & 4260      & 19         \\
      \bottomrule
    \end{tabular}
    \caption{Durée du test de primalité en millisecondes.}
  \end{table}

\end{frame}


%%%%%%%%%%%%%%%%%%%%%%%%%%%%%%%%%%%%%%%%%%%%%%%%%%%%%%%
\begin{frame}[fragile]
  \frametitle{Construction du vecteur des premiers inférieurs à $n$ }

  \textbf{Seuls les nombres impairs peuvent être premiers.}

\begin{lstlisting}[style=editor]
Premiers <- function(n) { # Version I
  if(n < 2) return(numeric(0))
  prems <- c(2)
  if(n > 2) {
    for(i in seq(3, n, 2)) { ## erreur de seq si n =2
      if(EstPremier(i)) prems <- append(prems, i)
    }
  }
  return(prems)
}
\end{lstlisting}


\begin{lstlisting}[style=editor]
Premiers <- function(n) { # Version II
  if(n < 2) return(numeric(0))
  if(n == 2) return(c(2))
  return(c(2, Filter(EstPremier, seq(3, n, 2))))
}
\end{lstlisting}

\end{frame}


%%%%%%%%%%%%%%%%%%%%%%%%%%%%%%%%%%%%%%%%%%%%%%%%%%%%%%%
\begin{frame}[fragile]
  \frametitle{Construction parallèle des premiers inférieurs à $n$ }

  \textbf{Seuls les nombres impairs peuvent être premiers.}

\begin{lstlisting}[style=editor]
Premiers <- function(n) { # Version III
  if(n < 2) return(numeric(0))
  if(n == 2) return(c(2))
  require(parallel) # chargement d'un package
  prems <- seq(3, n, 2)
  ind <- mcmapply(prems, FUN = EstPremier, mc.cores = 8)
  return(c(2, prems[ind]))
}
\end{lstlisting}
\end{frame}

%%%%%%%%%%%%%%%%%%%%%%%%%%%%%%%%%%%%%%%%%%%%%%%%%%%%%%%


\begin{frame}[fragile]
  \frametitle{Performance du calcul des premiers inférieurs à $n$ }

  Comparons les performances des trois versions de la construction du vecteur des nombres premiers inférieurs à $n$ pour quelques valeurs de $n$.

  \begin{lstlisting}
> system.time(replicate(5, Premiers(10**5)))
utilisateur     système      écoulé
      0.018       0.000       0.019
  \end{lstlisting}

  \begin{table}[ht]
    \centering
    \begin{tabular}{l|rrr}
      \toprule
      \texttt{n} & version I & version II & version III \\
      \midrule
      $10^4$     & 0.08      & 0.07       & 0.13        \\
      $10^5$     & 1.85      & 1.44       & 0.82        \\
      $10^6$     & 57.25     & 32.20      & 10.95       \\
      \bottomrule
    \end{tabular}
    \caption{Durée du calcul des premiers en secondes.}
  \end{table}

  \begin{exampleblock}{Crible d'Ératosthène (à faire en TP)}
    Il suffit d'examiner les diviseurs premiers d'un nombre.
  \end{exampleblock}
\end{frame}


%%%%%%%%%%%%%%%%%%%%%%%%%%%%%%%%%%%%%%%%%%%%%%%%%%%%%%%
\questionSlide

%%%%%%%%%%%%%%%%%%%%%%%%%%%%%%%%%%%%%%%%%%%%%%%%%%%%%%%
 \appendix
 \backupSlides
%%%%%%%%%%%%%%%%%%%%%%%%%%%%%%%%%%%%%%%%%%%%%%%%%%%%%%%

%%%%%%%%%%%%%%%%%%%%%%%%%%%%%%%%%%%%%%%%%%%%%%%%%%%%%%%
% \begin{frame}[fragile]{Backup slides}
%   Sometimes, it is useful to add slides at the end of your presentation to
%   refer to during audience questions.

%   The best way to do this is to include the \verb|appendixnumberbeamer|
%   package in your preamble and call \verb|\appendix| before your backup slides.

%   will automatically turn off slide numbering and progress bars for
%   slides in the appendix.
% \end{frame}


%%%%%%%%%%%%%%%%%%%%%%%%%%%%%%%%%%%%%%%%%%%%%%%%%%%%%%%
% \begin{frame}[allowframebreaks]{References}

%   % \bibliography{../bib_parallelism,../bib_others}
%   % \bibliographystyle{abbrv}

% \end{frame}

\end{document}

%%% Local Variables:
%%% mode: latex
%%% TeX-master: t
%%% End:
