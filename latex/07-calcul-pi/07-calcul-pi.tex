\documentclass[10pt]{beamer}

\usepackage{../macros}
\title{Approximation du nombre $\pi$}

\hypersetup{
  pdftitle =  {Approximation du nombre PI}
}

\begin{document}

\maketitle



%%%%%%%%%%%%%%%%%%%%%%%%%%%%%%%%%%%%%%%%%%%%%%%%%%%%%%%
\begin{frame}
  \frametitle{Le nombre $\pi$}
  \begin{itemize}
  \item $\pi$ est défini comme le rapport constant entre la circonférence d’un cercle et son diamètre dans le plan euclidien.
  \item De nos jours, les mathématiciens définissent $\pi$ par l’analyse réelle à l’aide des fonctions trigonométriques elles-mêmes introduites sans référence à la géométrie.
  \item Le nombre $\pi$ est \alert{irrationnel}, ce qui signifie qu'on ne peut pas l’écrire comme une fraction.
  \item Le nombre $\pi$ est \alert{transcendant} ce qui signifie qu'il n’existe pas de polynôme à coefficients rationnels dont $\pi$ soit une racine.
  \end{itemize}
\end{frame}


\begin{frame}
  \frametitle{Calcul de $\pi$ par la formule de Leibniz}
  On utilisera la formule de Leibniz issue du développement en série de Taylor en 0 de $arctan(x)$ évalué au point 1 :
  \alert{
    \[
      \sum_{k=0}^{+\infty} \frac{(-1)^k}{2k+ 1} = 1 - \frac{1}{3} + \frac{1}{5}  - \frac{1}{9} + \frac{1}{11} + \ldots = \frac{\pi}{4}
    \]
    }

Elle a été découverte en Occident au XVIIe, mais apparaît déjà chez Madhava, mathématicien indien de la province du Kerala, vers 1400.
\begin{flushright}
c.f. Wikipedia  
\end{flushright}

Nous allons développer un algorithme d'approximation de $\pi$.
  \begin{description}
  \item[ITÉRATION] Comment améliorer l'approximation courante ?
  \item[TERMINAISON] Mon approximation courante est-elle assez bonne ? Est-ce que le calcul prend trop de temps ? 
  \item[INITIALISATION] Comment initialiser la première approximation ? 
  \end{description}

\end{frame}


%%%%%%%%%%%%%%%%%%%%%%%%%%%%%%%%%%%%%%%%%%%%%%%%%%%%%%%
\begin{frame}[fragile]
  \frametitle{Algorithme d'approximation de $\pi$}
  \begin{block}{ITÉRATION}
    Pour \alert{améliorer} l'approximation, étant en possession de la somme $\mathtt{acc}$ des $i$ premiers termes, on voudra obtenir la somme des $i+1$ premiers. Il suffira donc d'incrémenter $\mathtt{i}$, \alert{puis} d'ajouter $ \frac{(-1)^{i}}{2i+ 1}$ à $\mathtt{acc}$.    
    \begin{lstlisting}[style=editor]
i <- i + 1
term <- (-1)**i / (2*i + 1)
acc <- acc + term

    \end{lstlisting}    
  \end{block}
  
  \begin{block}{TERMINAISON}
    Mon approximation courante a est-elle \alert{assez bonne} ? Elle est assez bonne lorsque je n'arrive plus à l'améliorer. Notons $\mathtt{h}$ la précision. \\
     Est-ce que le calcul \alert{prend trop de temps} ? Notons $\mathtt{n}$ le nombre maximum de termes à calculer.
\begin{lstlisting}[style=editor]
 abs(term) < h || i > n
\end{lstlisting}
\end{block}
\begin{block}{INITIALISATION}
  \begin{lstlisting}[style=edblock]
i <- 0
acc <- 1
\end{lstlisting}
\end{block}
\end{frame}

%%%%%%%%%%%%%%%%%%%%%%%%%%%%%%%%%%%%%%%%%%%%%%%%%%%%%%%
\begin{frame}[fragile]
  \frametitle{Programme d'approximation de $\pi$}
  \begin{lstlisting}[style=editor]
LeibnizPi <- function(n = 10**4, h = 2^(-20)) {
  i <- 0
  term <- 1
  acc <- 1
  while( (i <= n) && 4*abs(term) > h) {
    i <- i + 1
    term <- (-1)**i / (2*i + 1)
    acc <- acc + term
  }
  return(4*acc)
}
\end{lstlisting}

\begin{lstlisting}
> LeibnizPi(n = 100, h = 0)
[1] 3.131789
> LeibnizPi(n = 100000, h = 0)
[1] 3.141583
> pi
[1] 3.141593  
\end{lstlisting}

\end{frame}
%%%%%%%%%%%%%%%%%%%%%%%%%%%%%%%%%%%%%%%%%%%%%%%%%%%%%%%
\begin{frame}[fragile]
  \frametitle{Analyse de la convergence vers $\pi$}
  \begin{lstlisting}[style=editor]
ErreurRelativePi <- function(v) return(1 - v/pi)
ErreurRelativeLeibnizPi <- function(n) {
  approxPi <- LeibnizPi(n = n, h = 0)
  return(ErreurRelativePi(approxPi))
}
\end{lstlisting}



\begin{columns}[t]
\begin{column}{0.48\textwidth}
  \begin{exampleblock}{Formule de Leibniz}
    \begin{tabular}{lr}
      \toprule
    $\mathtt{n}$ & Erreur   \\
    \midrule
    100          & 0.003121 \\
    1000         & 0.000318 \\
    10000        & 0.000032 \\
    100000       & 0.000003 \\
      \bottomrule
  \end{tabular}
    
  \end{exampleblock}

\end{column}
\begin{column}{0.48\textwidth}
  \begin{exampleblock}{Représentation décimale}
    \begin{tabular}{lr}
    \toprule
    Fraction               & Erreur          \\
    \midrule
    $\frac{22}{7}$         & -0.000402499435 \\
    $\frac{355}{113}$      & -0.000000084914 \\
    $\frac{103993}{33102}$ & 0.000000000184  \\
    $\frac{104348}{33215}$ & -0.000000000106 \\
    \bottomrule
  \end{tabular}
  \end{exampleblock}
\end{column}
\end{columns}



\begin{alertblock}{Observation}
  \alert{La formule de Leibniz converge lentement.}
\end{alertblock}

\end{frame}

%%%%%%%%%%%%%%%%%%%%%%%%%%%%%%%%%%%%%%%%%%%%%%%%%%%%%%%
\begin{frame}[fragile]
  \frametitle{Analyse des performances du programme}
  Calculons le temps nécessaire pour atteindre une précision donnée sans limiter le nombre d'itérations.
  \begin{lstlisting}
> system.time(LeibnizPi(n = Inf, h = 10**(-4)))
utilisateur     système      écoulé 
      0.006       0.000       0.006 
    \end{lstlisting}

    
\begin{columns}[t]
\begin{column}{0.7\textwidth}
  \begin{itemize}
  \item Le temps d’exécution et le nombre d'itérations augmentent linéairement avec la précision.
  \item La recherche d’une estimation très précise de $\pi$ demande un temps de calcul important.
  
  \item En extrapolant ces résultats, il faudrait  $5 \times 10^8 secondes$ ($\geq$ 15 ans) pour obtenir une estimation de $\pi$ à la précision machine (approximativement 15 décimales).
  \item<alert@1> Certaines formules convergent beaucoup plus rapidement.
  \end{itemize}
\end{column}
\begin{column}{0.3\textwidth}
  \begin{table}[h]
    \centering
    \begin{tabular}{lr}
      \toprule
      Précision & Temps (s) \\
      \midrule
      $10^{-4}$ & 0.006     \\
      $10^{-5}$ & 0.147     \\
      $10^{-6}$ & 0.599     \\
      $10^{-7}$ & 5.347     \\
      $10^{-8}$ & 52.860    \\
      \bottomrule
    \end{tabular}
  \end{table}
\end{column}
\end{columns}
\end{frame}


%%%%%%%%%%%%%%%%%%%%%%%%%%%%%%%%%%%%%%%%%%%%%%%%%%%%%%%
\begin{frame}[fragile]
  \frametitle{Optimisation du programme}
  \begin{alertblock}{Exploitons la récurrence pour accélérer les calculs.}
    \alert{Les multiplications, divisions, et puissances sont plus coûteuse en temps de calcul que les additions et soustractions.}
  \end{alertblock}

  \begin{lstlisting}
LeibnizPi2 <- function(n = 10**4, h = 2^(-20)) {
  i <- 0
  term <- 1
  acc <- 1
  h <- h / 4 ## éviter la multiplication du test
  sign <- 1 ## mémoriser le signe du terme 
  denom <- 1 ## mémoriser le dénominateur du terme
  while( (i <=n) && abs(term) > h) {
    i <- i + 1
    sign <- -1 * sign # éviter une puissance
    denom <- denom + 2 # éviter une multiplication
    term <- sign / denom
    acc <- acc + term
  }
  return(4*acc)
}
    
  \end{lstlisting}
\end{frame}

%%%%%%%%%%%%%%%%%%%%%%%%%%%%%%%%%%%%%%%%%%%%%%%%%%%%%%%
\begin{frame}
  \frametitle{Comparaison de programmes}

  
    \begin{table}[h]
    \centering
    \begin{tabular}{lr}
      \toprule
      Précision           & Temps (s) \\
      \midrule
      \texttt{LeibnizPi}  & 5.347     \\
      \texttt{LeibnizPi2} & 4.157     \\
      En langage C        & 0.007     \\
             \bottomrule
    \end{tabular}
  \end{table}

  \begin{itemize}
  \item Les optimisations du programme offrent un gain supérieur à $20\%$.
  \item R est donc un langage interprété de haut niveau ce qui se paie au niveau des performances.
  \item Le langage C, entre autres, est beaucoup plus rapide.
  \item Le langage C est un langage impératif, généraliste et de bas niveau où chaque instruction du langage est compilée.
  \end{itemize}

\end{frame}

%%%%%%%%%%%%%%%%%%%%%%%%%%%%%%%%%%%%%%%%%%%%%%%%%%%%%%%
 \questionSlide

%%%%%%%%%%%%%%%%%%%%%%%%%%%%%%%%%%%%%%%%%%%%%%%%%%%%%%%
 \appendix
 \backupSlides
%%%%%%%%%%%%%%%%%%%%%%%%%%%%%%%%%%%%%%%%%%%%%%%%%%%%%%%

%%%%%%%%%%%%%%%%%%%%%%%%%%%%%%%%%%%%%%%%%%%%%%%%%%%%%%%
\begin{frame}[fragile]{La formule d'Euler converge vers $\frac{\pi}{2}$}
    $$
    P(n) = 1 + \frac{2^2}{1 \times 3} + \frac{4^2}{3 \times 5} + + \frac{6^2}{5 \times 7} + \ldots + \frac{4k^2}{4k^2-1}
    = \prod_{k=1}^n \frac{4k^2}{4k^2-1}
    $$
    \vspace{-10pt} 
    \begin{columns}[t]
      \begin{column}{0.53\textwidth}
  \begin{block}{Récurrence}
    \begin{lstlisting}[style=edblock]
P <- function(n) {      
  if(n <=0) return(0)
  else {
    t <- 4*(n**2)
    return((t/(t-1)) * P(n-1))}
}  
\end{lstlisting}
\end{block}
\begin{block}{Vectorisation (plus tard)}
    \begin{lstlisting}[style=edblock]
P <- function(n) {
  if(n <=0) return(0)
  terms <- 4*(1:n)**2
  return(prod(terms/(terms-1)))
}
\end{lstlisting}
  \end{block}

\end{column}
\begin{column}{0.47\textwidth}
  \begin{block}{Itération}
    \begin{lstlisting}[style=edblock]
P <- function(n) {      
  if(n <=0) return(0)
  acc <- 1
  i <- 0
  while(i < n) {
    i <- i + 1
    t <- 4*(i**2)
    acc <- acc * t/(t-1)
  }
  return(acc)
}  
\end{lstlisting}
  \end{block}
\end{column}
\end{columns}

\end{frame}


%%%%%%%%%%%%%%%%%%%%%%%%%%%%%%%%%%%%%%%%%%%%%%%%%%%%%%%
\begin{frame}[fragile]
  \frametitle{Elle converge lentement \dots}
  \begin{lstlisting}
> 2*P(100)
[1] 3.133787
> 2*P(10000)
[1] 3.141514
> 2*P(100000)
[1] 3.141585
> 2*P(100000)
[1] 3.141585
> 2*P(1000000)
[1] 3.141592
> pi
[1] 3.141593
\end{lstlisting}

\end{frame}

%%%%%%%%%%%%%%%%%%%%%%%%%%%%%%%%%%%%%%%%%%%%%%%%%%%%%%%
% \begin{frame}[allowframebreaks]{References}

%   % \bibliography{../bib_parallelism,../bib_others}
%   % \bibliographystyle{abbrv}

% \end{frame}

\end{document}

%%% Local Variables:
%%% mode: latex
%%% TeX-master: t
%%% End:
