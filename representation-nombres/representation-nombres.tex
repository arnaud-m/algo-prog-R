\documentclass[10pt]{beamer}

\usepackage{../macros}

\title{Représentation des nombres}
\author{Arnaud Malapert}

\hypersetup{
  pdfauthor = {Arnaud Malapert},
  pdftitle =  {Représentation des nombres},
}


\begin{document}

\maketitle


\begin{frame}{Cours en AUTOFORMATION}
  
  \begin{block}{Représentation des nombres}
    \begin{itemize}
    \item Système positionel : binaire, décimal, octal, et héxadécimal.
    \item Nombre non signé seulement (positif).
    \item Pas d'étude de la représentation en machine.
    \end{itemize}
  \end{block}

  \begin{block}{Prérequis}
    Savoir additionner, soustraire, multiplier, et diviser ! Surtout par 2 !
  \end{block}

  \begin{alertblock}{Évaluation : Gagner un max de points en peu de temps !
}
    \begin{itemize}
    \item QCM sur Moodle comptant un peu pour le contrôle continu. 
    \item 3 points du même QCM dans le contrôle terminal.
    \item Exercices de programmation autour des algorithmes de conversion.
    \item Activité de programmation d'une appli web de conversion.
    \end{itemize}
  \end{alertblock}

\end{frame}


%%%%%%%%%%%%%%%%%%%%%%%%%%%%%%%%%%%%%%%%%%%%%%%%%%%%%%%
\begin{frame}{Table of contents}
  \setbeamertemplate{section in toc}[sections numbered]
  % \tableofcontents[hideallsubsections]
  \tableofcontents
\end{frame}


%%%%%%%%%%%%%%%%%%%%%%%%%%%%%%%%%%%%%%%%%%%%%%%%%%%%%%%
\section{UCAnCODE}
%%%%%%%%%%%%%%%%%%%%%%%%%%%%%%%%%%%%%%%%%%%%%%%%%%%%%%%


%%%%%%%%%%%%%%%%%%%%%%%%%%%%%%%%%%%%%%%%%%%%%%%%%%%%%%%
\section{UCAnCODE II}
%%%%%%%%%%%%%%%%%%%%%%%%%%%%%%%%%%%%%%%%%%%%%%%%%%%%%%%
%%%%%%%%%%%%%%%%%%%%%%%%%%%%%%%%%%%%%%%%%%%%%%%%%%%%%%%
\subsection{UCAnCODE III}
%%%%%%%%%%%%%%%%%%%%%%%%%%%%%%%%%%%%%%%%%%%%%%%%%%%%%%%


\begin{frame}[fragile]
  \frametitle{Test Code R}
  \begin{lstlisting}[caption={My Caption}]
    print(sample(1:3))
    print(sample(1:3, size=3, replace=FALSE))  # same as previous line
    print(sample(c(2,5,3), size=4, replace=TRUE)
    print(sample(1:2, size=10, prob=c(1,3), replace=TRUE))    
  \end{lstlisting}


  \begin{exampleblock}{Code dans block}
      \begin{lstlisting}
print(sample(1:3))
    print(sample(1:3, size=3, replace=FALSE))  # same as previous line
    print(sample(c(2,5,3), size=4, replace=TRUE)
    print(sample(1:2, size=10, prob=c(1,3), replace=TRUE))
    x <- 5
    if(x > 0){
      print("Positive number")
    }
  \end{lstlisting}
  \end{exampleblock}
\end{frame}


%%%%%%%%%%%%%%%%%%%%%%%%%%%%%%%%%%%%%%%%%%%%%%%%%%%%%%%
 \questionSlide

%%%%%%%%%%%%%%%%%%%%%%%%%%%%%%%%%%%%%%%%%%%%%%%%%%%%%%%
\appendix
%%%%%%%%%%%%%%%%%%%%%%%%%%%%%%%%%%%%%%%%%%%%%%%%%%%%%%%

%%%%%%%%%%%%%%%%%%%%%%%%%%%%%%%%%%%%%%%%%%%%%%%%%%%%%%%
% \begin{frame}[fragile]{Backup slides}
%   Sometimes, it is useful to add slides at the end of your presentation to
%   refer to during audience questions.

%   The best way to do this is to include the \verb|appendixnumberbeamer|
%   package in your preamble and call \verb|\appendix| before your backup slides.

%   will automatically turn off slide numbering and progress bars for
%   slides in the appendix.
% \end{frame}

% \begin{frame}
%   \frametitle{test I}
  
% \end{frame}

\backupSlides

\begin{frame}
  \frametitle{test II}
  
\end{frame}
%%%%%%%%%%%%%%%%%%%%%%%%%%%%%%%%%%%%%%%%%%%%%%%%%%%%%%%
% \begin{frame}[allowframebreaks]{References}

%   % \bibliography{../bib_parallelism,../bib_others}
%   % \bibliographystyle{abbrv}

% \end{frame}

\end{document}

%%% Local Variables:
%%% mode: latex
%%% TeX-master: t
%%% End:
