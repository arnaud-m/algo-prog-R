\documentclass[10pt]{beamer}

\usepackage{../macros}
\title{Chaînes de caractères}

\hypersetup{
  pdftitle =  {Chaînes de caractères}
}

\begin{document}

\maketitle


%%%%%%%%%%%%%%%%%%%%%%%%%%%%%%%%%%%%%%%%%%%%%%%%%%%%%%%
\begin{frame}[fragile]
  \frametitle{Les chaînes de caractères}
  \begin{block}{Un texte est une suite finie de caractères.}
    Lettres majuscules ou minuscules, chiffres, ponctuations, symboles mathématiques, etc.
    En programmation, un texte est une chaîne de caractères (\alert{string}).
  \end{block}

\begin{block}{Chaîne de caractères délimitée par des apostrophes}
  \begin{lstlisting}[style=edblock]
> texte <- '1984 : Orwell !'
> texte
[1] "1984 : Orwell !"    
  \end{lstlisting}
\end{block}

\begin{exampleblock}{Attention aux types des données}
  Ne confondez pas 1984 qui est un entier et '1984' qui est une chaîne.
  \begin{columns}[t]
    \begin{column}{0.48\textwidth}
      \begin{lstlisting}
> typeof(1984)
[1] "double"
> typeof('1984')
[1] "character"            
      \end{lstlisting}
\end{column}
\begin{column}{0.48\textwidth}
  \begin{lstlisting}
> as.integer('1984')
[1] 1984
> as.character(1984)
[1] "1984"    
  \end{lstlisting}
\end{column}
\end{columns}

\end{exampleblock}
\end{frame}


%%%%%%%%%%%%%%%%%%%%%%%%%%%%%%%%%%%%%%%%%%%%%%%%%%%%%%%
\begin{frame}[fragile]
  \frametitle{Apostrophes et guillemets}
\begin{block}{Chaîne de caractères délimitée par des guillemets}
  \begin{lstlisting}[style=edblock]
> texte <- "1984 : Orwell !"
  \end{lstlisting}
\end{block}

\begin{exampleblock}{Comment inclure une apostrophe dans une chaîne ?}
  \begin{lstlisting}[style=block]
> 'Mais j'ai froid'
Erreur : unexpected symbol in "'Mais j'ai"    
\end{lstlisting}
\begin{columns}[t]
  \begin{column}{0.48\textwidth}
    Délimiter avec des guillemets.
  \begin{lstlisting}
> "Mais j'ai froid"
[1] "Mais j'ai froid"    
\end{lstlisting}
\end{column}
\begin{column}{0.48\textwidth}
  Protéger avec un backslash (\textbackslash).
  \begin{lstlisting}
> 'Mais j\'ai froid'
[1] "Mais j'ai froid"
\end{lstlisting}
\end{column}
\end{columns}

\end{exampleblock}
\end{frame}


%%%%%%%%%%%%%%%%%%%%%%%%%%%%%%%%%%%%%%%%%%%%%%%%%%%%%%%
\begin{frame}[fragile]
  \frametitle{Une chaîne est une suite de caractères}
  \begin{block}{Combien y a t-il de caractères dans une chaîne ?}
    \begin{lstlisting}[style=block]
> texte <- '1984 : Orwell !'
> nchar(texte)
[1] 15      
    \end{lstlisting}
  \end{block}

  \begin{block}{Comment accéder au k-ème caractère ?}
    Pas très pratique, on extrait une sous-chaîne de taille 1 !
    \begin{lstlisting}[style=block]
> substr(texte,1,1)
[1] "1"
> substr(texte,4,4)
[1] "4"
> substr(texte,nchar(texte),nchar(texte))
[1] "! "
> substr(texte,nchar(texte)+1,nchar(texte)+1)
[1] ""
    \end{lstlisting}
    En fait, R n’est pas conçu pour ce genre de manipulation.    
  \end{block}
\end{frame}

%%%%%%%%%%%%%%%%%%%%%%%%%%%%%%%%%%%%%%%%%%%%%%%%%%%%%%%
\begin{frame}[fragile]
  \frametitle{Création d'une chaîne}
  La fonction \alert{\texttt{paste}} combine de multiples chaînes en une seule.
  \begin{lstlisting}
  > paste("Sa", "lut")
[1] "Sa lut"
> paste("Sa", "lut",sep="") # ou paste0("Sa", "lut")
[1] "Salut"
\end{lstlisting}
En fait, la fonction paste fonctionne aussi sur les vecteurs/listes comme presque toutes les fonctions en R.
\begin{lstlisting}
> mots <- c("pomme", "banane", "raisin")
> paste(mots, mots)
[1] "pomme pomme" "banane banane" "raisin raisin"
> paste(mots, mots, collapse="|")
[1] "pomme pomme|banane banane|raisin raisin"
> paste(mots, mots, sep="-",collapse="|")
[1] "pomme-pomme|banane-banane|raisin-raisin"  
\end{lstlisting}
La fonction \alert{\texttt{sprintf}} formatte une chaîne à la C.
\begin{lstlisting}
> sprintf("%s%s", "Sa", "lut")
[1] "Salut"
\end{lstlisting}
\end{frame}

%%%%%%%%%%%%%%%%%%%%%%%%%%%%%%%%%%%%%%%%%%%%%%%%%%%%%%%
\begin{frame}[fragile]
  \frametitle{Extraction d'une tranche de sous-chaîne}
  R propose une fonction (\alert{\texttt{substr}} ou \alert{\texttt{substring}}) pour extraire une tranche (slice) d'une chaîne repérée par ses positions extrêmes :
  \begin{lstlisting}
> texte <- "salut tout le monde"
> substring(texte,first=12)
[1] "le monde"
> substring(texte,first=1,last=11)
[1] "salut tout "    
\end{lstlisting}

\begin{alertblock}{Puis-je modifier le k-ème caractère d'une chaîne ?}
  \alert{NON} : une chaîne est un objet non mutable. 
\end{alertblock}

\begin{exampleblock}{Mais, on peut quand même \dots}
  On peut modifier la valeur de la variable avec une affectation, mais cela reste compliquée.
  \begin{lstlisting}
> substr(texte,1,5) <- 'Hello'
> texte
[1] "Hello tout le monde"    
  \end{lstlisting}
\end{exampleblock}
\end{frame}


%%%%%%%%%%%%%%%%%%%%%%%%%%%%%%%%%%%%%%%%%%%%%%%%%%%%%%%
\begin{frame}[fragile]
  \frametitle{Éclatement d’une chaîne}
La fonction \alert{\texttt{strsplit}} sépare une chaîne de caractères en une liste de sous-chaînes en utilisant un séparateur.  
\begin{lstlisting}
> word <- "apple|banana|grape"
> v <- strsplit(word,split="|",fixed=TRUE)
> v
[[1]]
[1] "apple" "banana" "grape"
> v[[1]][1]
[1] "apple"  
\end{lstlisting}

\begin{itemize}
\item Le paramètre \texttt{fixed} indique que le séparateur doit être interprété comme une chaîne et non pas comme une expression régulière.
\item Le résultat est renvoyé sous la forme d’une liste, une structure de données essentielle que nous découvrirons plus tard.
\end{itemize}
\end{frame}
%%%%%%%%%%%%%%%%%%%%%%%%%%%%%%%%%%%%%%%%%%%%%%%%%%%%%%%
\begin{frame}[fragile]
  \frametitle{Expressions régulières}
  \begin{alertblock}{Expression régulière}
    une expression régulière (regular expression ou regexp) est un motif qui décrit un ensemble de chaînes de caractères. \\
\end{alertblock}
\begin{itemize}
\item Une regexp est donc une "instruction" donnée à une fonction définissant la recherche (match) et les remplacements (replace) dans une chaîne de caractères.
\item R propose une famille de fonctions similaires aux commandes shell \texttt{grep} pour la recherche et \texttt{sed} pour le remplacement.  
\end{itemize}
\begin{lstlisting}
> l <- c("apple","banana","grape","10","green.pepper")
> grep("a",l)
[1] 1 2 3
> sub("a","$",l)
[1] "$pple" "b$nana" "gr$pe" "10" "green.pepper"
> gsub("a","$",l)
[1] "$pple" "b$n$n$" "gr$pe" "10" "green.pepper"  
\end{lstlisting}
\end{frame}
%%%%%%%%%%%%%%%%%%%%%%%%%%%%%%%%%%%%%%%%%%%%%%%%%%%%%%%

\begin{frame}
  \frametitle{Beaucoup de primitives sur les chaînes \dots}
  Un langage de programmation vaut aussi par l'étendue de ses fonctionnalités.
  Avant de programmer, cherchez dans la doc si la fonction convoitée n'est pas une primitive !

  \begin{block}{\dots mais certaines ont une drôle de tête !}
    Quasiment toutes les fonctions sur les chaînes prennent en argument et renvoient des VECTEURS ou LISTES.
    Pour l'instant, nous garderons une idée naïve de ce qu'est un vecteur ou une liste, une séquence d’éléments.    
  \end{block}

\end{frame}

%%%%%%%%%%%%%%%%%%%%%%%%%%%%%%%%%%%%%%%%%%%%%%%%%%%%%%%
\begin{frame}[fragile]
  \frametitle{Autres méthodes sur des chaînes}
  \begin{lstlisting}
> l <- c('', 'pommes', '12', 'CERISES')
> grepl("^[[:digit:]]+$", l) # que des chiffres ?
[1] FALSE FALSE TRUE FALSE
> grepl("^[[:lower:]]+$", l) # que des minuscules ?
[1] FALSE TRUE FALSE FALSE
> grepl("^[[:alpha:]]+$", l) # que des lettres?
[1] FALSE TRUE FALSE TRUE
> grepl("^[[:alnum:]]+$", l) # que des chiffres ou des lettres?
[1] FALSE TRUE TRUE TRUE 
> tolower(l) # mettre en minuscules
[1] "" "pommes" "12" "cerises"
> toupper(l) # mettre en majuscules
[1] "" "POMMES" "12" "CERISES"
> chartr("oei", "OEI", l) #remplacement (c.f. tr en shell) 
[1] "" "pOmmEs" "12" "CERISES"
  \end{lstlisting}
\end{frame}




%%%%%%%%%%%%%%%%%%%%%%%%%%%%%%%%%%%%%%%%%%%%%%%%%%%%%%%
\begin{frame}[fragile]
  \frametitle{Le code ASCII}
  Les \alert{caractères américains (dits internationaux)} sont numérotés de 0 à 127, c'est le code ASCII (American Standard Code for Information Interchange).
  Par exemple le code ASCII de 'A' est égal à 65 mais on n'a pas besoin de le mémoriser :
  
\begin{columns}[t]
\begin{column}{0.33\textwidth}
  \begin{lstlisting}
> utf8ToInt('A')
[1] 65    
> utf8ToInt('a')
[1] 97
  \end{lstlisting}
\end{column}
\begin{column}{0.33\textwidth}
  \begin{lstlisting}
> utf8ToInt('?')
[1] 63
> utf8ToInt('!')
[1] 33    
  \end{lstlisting}
\end{column}
\begin{column}{0.33\textwidth}
  \begin{lstlisting}
> intToUtf8(65)
[1] "A"
> intToUtf8(63)
[1] "?"    
  \end{lstlisting}
\end{column}
\end{columns}

Les caractères dont le code ASCII tombe entre 0 et 31, ainsi que le caractère numéro 127 ne sont pas affichables .
Ce sont les \alert{caractères de contrôle} nommés symboliquement NUL , LF , BEL , CR , ESC , DEL \dots
\begin{columns}[t]
\begin{column}{0.48\textwidth}
  \begin{lstlisting}
> intToUtf8(27) # ESC
[1] "\033"
  \end{lstlisting}
\end{column}
\begin{column}{0.48\textwidth}
  \begin{lstlisting}
> intToUtf8(10) # LF
[1] "\n"
  \end{lstlisting}

\end{column}
\end{columns}
\end{frame}


%%%%%%%%%%%%%%%%%%%%%%%%%%%%%%%%%%%%%%%%%%%%%%%%%%%%%%%
\begin{frame}[fragile,fragile]
  \frametitle{Table ASCII}
  {
    \footnotesize
\begin{verbatim}
> Ascii()
 [1]  32:    33:!   34:"   35:#   36:$   37:%   38:&   39:'   40:(   41:) 
[11]  42:*   43:+   44:,   45:-   46:.   47:/   48:0   49:1   50:2   51:3 
[21]  52:4   53:5   54:6   55:7   56:8   57:9   58::   59:;   60:<   61:= 
[31]  62:>   63:?   64:@   65:A   66:B   67:C   68:D   69:E   70:F   71:G 
[41]  72:H   73:I   74:J   75:K   76:L   77:M   78:N   79:O   80:P   81:Q 
[51]  82:R   83:S   84:T   85:U   86:V   87:W   88:X   89:Y   90:Z   91:[ 
[61]  92:\\  93:]   94:^   95:_   96:`   97:a   98:b   99:c  100:d  101:e 
[71] 102:f  103:g  104:h  105:i  106:j  107:k  108:l  109:m  110:n  111:o 
[81] 112:p  113:q  114:r  115:s  116:t  117:u  118:v  119:w  120:x  121:y 
[91] 122:z  123:{  124:|  125:}  126:~
\end{verbatim}
  }

  \begin{block}{À faire en TP}
    \begin{lstlisting}[style=edblock]
Ascii <- function() {
  # ...
}
\end{lstlisting}
  \end{block}
\end{frame}


%%%%%%%%%%%%%%%%%%%%%%%%%%%%%%%%%%%%%%%%%%%%%%%%%%%%%%%
\begin{frame}[fragile]
  \frametitle{Le code ASCII étendu}
  Le code du caractère accentué 'é' est 233, il tombe entre 128 et 255.
  Il s'agit du code ASCII étendu, qui contient des caractères propres à chaque pays.
  Nous utilisons le codage Iso-Latin-1 , qui couvre les langues de l'Europe de l'Ouest, avec le \'e, \`a, \~n, \"o, etc.
  Les russes utilisent KOI8-R, les chinois sont bien ennuyés, etc.

  \begin{block}{Pourquoi ce nombre magique 255 ?}
  Parce que l'unité d'information sur un ordinateur est l'octet : un bloc de 8 bits (un bit = 0 ou 1).
Sur les 8 bits, un bit était utilisé pour la vérification de la bonne transmission (le bit de parité), d'où un
codage sur 7 bits, c'est l'ASCII.
Avec la fiabilité des transmissions, le 8ème bit fut finalement utilisé pour l'ASCII étendu \dots

\begin{tabular}{|*{8}{c|}r}
  \cline{1-8}
  1 & 1 & 1& 0 & 1 & 0 & 0 & 1 & é\\ 
  \cline{1-8}
\end{tabular}
\end{block}
\end{frame}


%%%%%%%%%%%%%%%%%%%%%%%%%%%%%%%%%%%%%%%%%%%%%%%%%%%%%%%
\begin{frame}
  \frametitle{Le code UNICODE}
Et les chinois furent contents ! En effet, le consortium \alert{Unicode} (env 1990, initié par Xerox et Apple) proposait d'abandonner la limitation des 255 caractères en traitant d'un seul coup toutes les
langues du monde (plus de 65000 caractères !) \dots
    \begin{itemize}
    \item \alert{Unicode} donne à tout caractère de n’importe quel système d'écriture un nom et un identifiant numérique, et ce de manière unifiée, quelle que soit la plate-forme informatique ou le logiciel.
    \item Le jeu de caractères est la liste des caractères, leur nom et leur index, le point de code.
    \item<alert@1> Unicode sépare la définition du jeu de caractères de celle du codage.   
    \end{itemize}

  \begin{block}{UTF-8}
    \begin{itemize}
    \item \alert{UTF-8 est un format de transformation}, il définit un codage pour tout caractère Unicode.
    \item<alert@1> UTF-8 est compatible avec la norme ASCII.
    \end{itemize}
  \end{block}

\end{frame}

%%%%%%%%%%%%%%%%%%%%%%%%%%%%%%%%%%%%%%%%%%%%%%%%%%%%%%%

\begin{frame}[fragile]
  \frametitle{Usage du code UNICODE}
  Les caractères Unicode sont numérotés en hexadécimal (base 16).
  Par exemple, le symbole euro a pour numéro '20AC' en hexa :
  \begin{lstlisting}
> '\u20ac' # symbole euro 
> '\u6211\u662F\u6cd5\u56fd\u4eba' # je suis français
> for(i in 1633:1642) print(intToUtf8(i)) # ?
\end{lstlisting}

\begin{exampleblock}{Conversion en hexadécimal}
\begin{lstlisting}[style=block]
> x <- as.hexmode(8364)
> x
[1] "20ac"
> x <- as.integer(x)
> x
[1] 8364  
\end{lstlisting}
\end{exampleblock}
\end{frame}
%%%%%%%%%%%%%%%%%%%%%%%%%%%%%%%%%%%%%%%%%%%%%%%%%%%%%%%

\begin{frame}
  \frametitle{Jules César et les messages secrets}
  La cryptographie est la science des codes secrets.
  Une des premières formes de codage d'un message provient de Jules César durant la Guerre des Gaules.
  Ici, on ne code que les MAJUSCULES...

  \begin{block}{Pour coder un message}
    \begin{itemize}
    \item Choisir une clé de codage \texttt{k} dans [1,25]. 
    \item Initialiser la chaîne résultat à la chaîne vide.
    \item Pour chaque caractère \texttt{c} du message à coder :
      \begin{itemize}
      \item Si le caractère c est une lettre majuscule : coder le caractère c en le décalant de k positions à droite insérer le nouveau caractère dans la chaîne résultat.
      \item sinon : insérer c dans la chaîne résultat
      \end{itemize}
    \item Rendre en résultat la chaîne résultat    

    \end{itemize}
  \end{block}

  \begin{exampleblock}{Code César avec une clé \texttt{k}=3}
    'TOUS au ZOO !'  $\Longrightarrow$ 'WRXV au CRR !'
  \end{exampleblock}
\end{frame}


%%%%%%%%%%%%%%%%%%%%%%%%%%%%%%%%%%%%%%%%%%%%%%%%%%%%%%%
\questionSlide

%%%%%%%%%%%%%%%%%%%%%%%%%%%%%%%%%%%%%%%%%%%%%%%%%%%%%%% 
 \appendix
 \backupSlides
%%%%%%%%%%%%%%%%%%%%%%%%%%%%%%%%%%%%%%%%%%%%%%%%%%%%%%%

%%%%%%%%%%%%%%%%%%%%%%%%%%%%%%%%%%%%%%%%%%%%%%%%%%%%%%%
% \begin{frame}[fragile]{Backup slides}
%   Sometimes, it is useful to add slides at the end of your presentation to
%   refer to during audience questions.

%   The best way to do this is to include the \verb|appendixnumberbeamer|
%   package in your preamble and call \verb|\appendix| before your backup slides.

%   will automatically turn off slide numbering and progress bars for
%   slides in the appendix.
% \end{frame}


\begin{frame}[fragile]{Backup slides}
  Les diapositives suivantes, issues du cours 7, doivent probablement être déplacées dans un autre cours. \\
  En effet, on utilise les vecteurs à bloc ! 
\end{frame}
 

%%%%%%%%%%%%%%%%%%%%%%%%%%%%%%%%%%%%%%%%%%%%%%%%%%%%%%%
\begin{frame}[fragile]{Codage d'un entier en binaire I}
  Coder en binaire un entier consiste à remarquer que ses chiffres (lus de droite à gauche) ne sont autres que les restes des divisions par 2.
  \begin{exampleblock}{$(13)_{10}$ s'écrit $(1101)_2$ en binaire : $13 = 6 \times 2 + 1$}
    \begin{itemize}
    \item Le prefixe 110 correspond au codage du quotient (6).
    \item Le suffixe 1 correspond au reste (1).
    \end{itemize}
  \end{exampleblock}

  \begin{block}{Par récurrence}
    \begin{lstlisting}[style=edblock]
codebin <- function(n) {
  if(n < 2) {
    return(as.character(n))
  } else {
    return(paste(codebin(n%/%2),n %% 2,sep=''))
  }}      
    \end{lstlisting}
  \end{block}

  \begin{lstlisting}
> codebin(13)
[1] '1101'
> codebin(0)
[1] '0'
\end{lstlisting}


\end{frame}


%%%%%%%%%%%%%%%%%%%%%%%%%%%%%%%%%%%%%%%%%%%%%%%%%%%%%%%
\begin{frame}[fragile]{Codage d'un entier en binaire II}
  \begin{block}{Par itération}
    \begin{lstlisting}[style=edblock]
codebin <- function(n) {
  res <- c()
  while( n > 0) {
    res <- append(n %% 2, res)
    n <- n %/%2
  }
  return(paste(res,collapse=''))
}      
    \end{lstlisting}
  \end{block}

  \begin{exampleblock}{Questions}
    \begin{itemize}
    \item Que renvoie l'appel \texttt{codebin(13)} ?
    \item Que renvoie l'appel \texttt{codebin(0)} ?
    \item Comment peut-on corriger la fonction \texttt{codebin} ?
    \end{itemize}
  \end{exampleblock}

\end{frame}


%%%%%%%%%%%%%%%%%%%%%%%%%%%%%%%%%%%%%%%%%%%%%%%%%%%%%%%
\begin{frame}[fragile]{Décodage du binaire vers le décimal : méthode naïve}

  Le \alert{mauvais algorithme} consiste à calculer des puissances de 2.
  $$(1101)_2 = 1*2^3 + 1*2^2 + 0*2^1 + 1*2^0$$
  \begin{lstlisting}[style=editor]
decodebin <- function(n) {
  digits <- utf8ToInt(n)- utf8ToInt('0')
  res <- 0
  for(i in seq_along(digits)) {
    res <- res + digits[i]* 2**(length(digits)-i)
  }
  return(res)
}    
\end{lstlisting}

\begin{lstlisting}
> decodebin('1101')
[1] 13  
\end{lstlisting}
\end{frame}

%%%%%%%%%%%%%%%%%%%%%%%%%%%%%%%%%%%%%%%%%%%%%%%%%%%%%%%
\begin{frame}[fragile]{Décodage du binaire vers le décimal : schéma de Horner}

  Le \alert{bon algorithme} consiste à implémente un schéma de Hörner : calcul optimal de
  la valeur d’un polynôme en un point sans calculs de puissances.
  $$(1101)_2 = (1 \times 2 + 1) \times 2 + 0) \times 2 + 1$$
  \begin{lstlisting}[style=editor]
decodebin <- function(n) {
  horner <- function(digits) {
    res <- 0
    if(length(digits) > 0) {
      res <- 2* horner(head(digits,-1)) + tail(digits,1)
    }
    return(res)
  }
  digits <- utf8ToInt(n)- utf8ToInt('0')
  return(horner(digits))
}
\end{lstlisting}

\begin{lstlisting}
> decodebin('1101')
[1] 13  
\end{lstlisting}
\end{frame}

%%%%%%%%%%%%%%%%%%%%%%%%%%%%%%%%%%%%%%%%%%%%%%%%%%%%%%%
\begin{frame}[fragile]
  \frametitle{Traitement des vecteurs}
  Une chaîne de caractères n’est pas mutable, mais on peut la transformer en un vecteur de code de caractères, et un vecteur est mutable \dots
  \begin{exampleblock}{Transformation des chiffres en '?'}
    
\begin{columns}[c]
\begin{column}{0.66\textwidth}
    \begin{lstlisting}[style=edblock]
cacher <- function(str) {
  chars <- utf8ToInt(str)
  c0 <- utf8ToInt('0')
  c9 <- utf8ToInt('9')
  res <- c()
  for(i in seq_along(chars)) {
    if(chars[i]>=c0 && chars[i]<=c9) {
      res <- append(res,utf8ToInt('?'))
    } else {
      res <- append(res, chars[i])
    }
  }
  return(intToUtf8(res))
}      
\end{lstlisting}

\end{column}
\begin{column}{0.34\textwidth}
  \begin{lstlisting}
> cacher('324DAC75')
[1] '???DAC??'    
\end{lstlisting}


\end{column}
\end{columns}
\alert{Inconvénient : construction incrémentale de \texttt{res} par \texttt{append} est coûteuse.}
  \end{exampleblock}
\end{frame}



\begin{frame}[fragile]
  \frametitle{Traitement vectorisé des vecteurs }
  La solution consiste à basculer la chaîne dans un vecteur en un vecteur de code de caractères, puis à effectuer un \alert{traitement de vecteur mutable}, puis à rebasculer le vecteur liste en une chaîne. Intellectuel non ?

  \begin{lstlisting}[style=editor]
cacher <- function(str) {
  chars <- utf8ToInt(str)
  c0 <- utf8ToInt('0')
  c9 <- utf8ToInt('9')
  numbers <- which( chars >= c0 & chars <= c9)
  chars[numbers] <- utf8ToInt('?')
  return(intToUtf8(chars))
}    
\end{lstlisting}

\begin{exampleblock}{concaténer un vecteur de chaînes : une opération fréquente}
\begin{lstlisting}[style=block]
> paste(c('a','bb','c'),collapse='')
[1] "abbc"  
\end{lstlisting}
  
\end{exampleblock}
\end{frame}


%%%%%%%%%%%%%%%%%%%%%%%%%%%%%%%%%%%%%%%%%%%%%%%%%%%%%%%
% \begin{frame}[allowframebreaks]{References}

%   % \bibliography{../bib_parallelism,../bib_others}
%   % \bibliographystyle{abbrv}

% \end{frame}

\end{document}

%%% Local Variables:
%%% mode: latex
%%% TeX-master: t
%%% End:
