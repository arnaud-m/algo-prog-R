\documentclass[11pt]{article}

\usepackage[utf8]{inputenc}

\usepackage{twemojis}

\usepackage[T1]{fontenc}
\usepackage[cyr]{aeguill}
\usepackage[francais]{babel}
\usepackage{hyperref}
\usepackage{amsmath,amsthm,amssymb}
\usepackage[normalem]{ulem}

\textwidth 16cm
%\textheight 22cm
\evensidemargin 0cm
\oddsidemargin 0cm

\usepackage{colors}
\usepackage{encarts}
\usepackage{tikzstyles}
\usepackage{theoremes}
\usepackage{macros}

\title{Activité : Partition}
\date{}

\begin{document}

  \maketitle

  Le problème de partition \cite{PP} est un problème NP-complet, c'est un problème pour lequel on ne connaît pas (encore ?) d'algorithme efficace pour le résoudre.

  \section{Description du problème}
  
  \begin{definition}{Partition d'un ensemble}
    Un \emph{multiensemble} (parfois appelé sac) est un ensemble dans lequel chaque élément peut apparaître plusieurs fois. 

    Soit un multiensemble $S$ de $n$ entiers naturels. 

    $$S = \{s_i\ |\ s_i > 0\}_{1\leq i \leq n}.$$

    Une \emph{(bi)partition} de $S$ est constituée de deux sous-multiensembles $S_1$ et $S_2$ tels que :
    \begin{itemize}
      \item $S_1$ et $S_2$ sont non vides:  $S_1 \neq \emptyset$ et $S_2 \neq \emptyset$ ;
      \item $S_1$ et $S_2$ sont disjoints:  $S_1 \cap S_2 = \emptyset$ ;
      \item $S_1$ et $S_2$ recouvrent $S$:  $S_1 \cup S_2 = S$ ;
    \end{itemize}
  \end{definition}

  \begin{exemple}{Partition d'un ensemble}
    Soit un multiensemble $S = \{1,2,3,4,5\}$. 

    \begin{itemize}
      \item Les multiensembles $S_1$ et $S_2$ forment une partition de $S$.
        \begin{itemize}
          \item $S_1 = \{1\}$ et $S_2 = \{2,3,4,5\}$
          \item $S_1 = \{2, 4\}$ et $S_2 = \{1,3,5\}$
        \end{itemize}
      \item Les multiensembles $S_1$ et $S_2$ ne forment pas une partition de $S$.
        \begin{itemize}
          \item $S_1 = \{1,2,3,4,5\}$ et $S_2 = \emptyset$, car $S_2$ est vide.
          \item $S_1 = \{1,2,3\}$ et $S_2 = \{3,4,5\}$, car leur intersection est non vide.
          \item $S_2 = \{1,2\}$ et $S_2 = \{4, 5\}$, car 3 est dans 4, mais n'appartient ni à $S_1$ ni à $S_2$. 
        \end{itemize}
    \end{itemize}
  \end{exemple}
  
  \begin{definition}{Partition parfaite d'un multiensemble pair}
    Un multiensemble d'entiers $S$ est dit \emph{pair} si la somme des entiers de $S$ est pair.

    Une \emph{partition parfaite} d'un multiensemble pair est une partition telle que la valeur absolue de la différence entre la somme des entiers de $S_1$ et la somme des entiers de $S_2$ est 0.
  \end{definition}
  
  \begin{exemple}{Partition parfaite d'un multiensemble pair}
    $S = \{1,2,3,4\}$ est un multiensemble pair.

    \begin{itemize}
      \item $S_1 = \{1,3\}$ et $S_2 = \{2,4\}$ ne forment pas une partition parfaite.
      \item $S_1 = \{1,4\}$ et $S_2 = \{2,3\}$ forment une partition parfaite.
    \end{itemize}
  \end{exemple}
  
  \begin{definition}{Partition parfaite d'un multiensemble impair}
    Un multiensemble d'entiers $S$ est dit \emph{impair} si la somme des entiers de $S$ est impair.

    Une \emph{partition parfaite} d'un multiensemble impair est une partition telle que la valeur absolue de la différence entre la somme des entiers de $S_1$ et la somme des entiers de $S_2$ est 1.
  \end{definition}
  
  \begin{exemple}{Partition parfaite d'un multiensemble impair}
    $S = \{1,2,3,4, 5\}$ est un multiensemble impair.

    \begin{itemize}
      \item $S_1 = \{2, 4\}$ et $S_2 = \{1,3,5\}$ ne forment pas une partition parfaite.
      \item $S_1 = \{1, 2, 5\}$ et $S_2 = \{3,4\}$ forment une partition parfaite.
    \end{itemize}
  \end{exemple}

  En informatique, le problème de partitionnement consiste à déterminer si une partition parfaite d'un ensembles d'entiers existe. C'est un problème \emph{NP-complet}. Cependant, il existe plusieurs algorithmes qui résolvent efficacement le problème que ce soit de manière approchée ou optimale. Pour ces raisons, il est réputé ``le plus facile des problèmes difficiles" \cite{Mertens2003}.

  Le problème de partitionnement se décline aussi en problème d'optimisation dans lequel on recherche une partition minimisant la valeur absolue de la différence entre la somme des entiers des deux sous-ensembles de la partition.
  Ce problème d'optimisation est \emph{NP-difficile}.

  \begin{exercice}{}
    Trouver une partition parfaite de $[1, 8]$.
  \end{exercice}

  % \begin{solution}
  %   Une solution possible : $S_1 = \{1, 2, 7, 8\} (18), S_2 = \{3, 4, 5, 6\} (18)$
  % \end{solution}

  \begin{exercice}{}
    Trouver une partition parfaite de $[1, 9]$.
  \end{exercice}

  % \begin{solution}
  %   Une solution possible : $S_1 = \{1, 2, 3, 8, 9\} (23), S_2 = \{4, 5, 6, 7\} (22)$
  % \end{solution}

  \begin{exercice}{}
    Généraliser pour trouver une partition parfaite de $[1, n]$ avec $n=2p$, $p \geq 0$ quand 
    \begin{itemize}
      \item $p$ est pair ;
      \item $p$ est impair.
    \end{itemize}
  \end{exercice}

  \begin{indice}
    Les nombres de $1$ à $n$ avec $n = 2p$ peuvent être représentés ainsi :

    \begin{tabular}{|*{5}{c |}}
      \hline
      1 & 2 & $\cdots$ & $p-1$ & $p$\\
      $2p$ & $2p-1$ & $\cdots$ & $p+2$ & $p+1$\\\hline
    \end{tabular}

    Que remarquez-vous pour chacune des colonnes de ce tableau ?
  \end{indice}

  % \begin{solution}
  %   Dans l'indice on remarque que la somme des nombres de chaque colonne est la même $2p + 1$. 

  %   Quand $p$ est pair on met les nombres des $\frac{p}{2}$ premières colonnes dans un ensemble et ceux des $\frac{p}{2}$ colonnes suivantes dans l'autre ensemble. Les deux ensembles ont donc la même somme. Soit 

  %   $$\begin{align*}
  %     S_1 = & \left\{1, 2, \dots, \frac{p}{2}, 2p - \frac{p}{2}+1, 2p - \frac{p}{2}+2, \dots, 2p\right\}\\
  %     S_2 = & \left\{\frac{p}{2} + 1, \frac{p}{2} + 2, \dots, 2p - \frac{p}{2}\right\}.
  %   \end{align*}$$

  %   Quand $p$ est impair on met les nombres des $\frac{p - 1}{2}$ premières colonnes dans un ensemble et ceux des $\frac{p - 1}{2}$ colonnes suivantes dans l'autre ensemble. Il reste la dernière colonne contenant $p$ et $p+1$, on met donc $p$ dans $S_1$ et $p+1$ dans $S_2$. Soit 
    
  %   $$\begin{align*}
  %     S_1 = & \left\{1, 2, \dots, \frac{p-1}{2}, p, 2p - \frac{p-1}{2}+1, 2p - \frac{p-1}{2}+2, \dots, 2p\right\}\\
  %     S_2 = & \left\{\frac{p-1}{2} + 1, \frac{p-1}{2} + 2, \dots, p-1, p+1, p+2, \dots, 2p - \frac{p-1}{2}\right\}.
  %   \end{align*}$$
  % \end{solution}

  \begin{exercice}{}
    Généraliser pour trouver une partition parfaite de $[1, n]$ avec $n=2p + 1$, $p \geq 0$ quand 
    \begin{itemize}
      \item $p$ est pair ;
      \item $p$ est impair.
    \end{itemize}
  \end{exercice}

  \begin{indice}
    Les nombres de $1$ à $n$ avec $n = 2p + 1$ peuvent être représentés ainsi :

    \begin{tabular}{|*{6}{c |}}
      \hline
      1 & 2 & 3 & $\cdots$ & $p$ & $p + 1$\\
      & $2p + 1$ & $2p$ & $\cdots$ & $p+3$ & $p+2$\\\hline
    \end{tabular}
    
    Que remarquez-vous pour chacune des colonnes (exceptée la première) de ce tableau ?
  \end{indice}

  % \begin{solution}
  %   Dans l'indice on remarque que la somme des nombres de chaque colonne est la même $2p + 3$ sauf pour la première colonne où la somme est égale à 1. On fait donc pareil que précédemment en supprimant la première colonne.

  %   Quand $p$ est pair on met les nombres des $\frac{p}{2}$ premières colonnes dans un ensemble et ceux des $\frac{p}{2}$ colonnes suivantes dans l'autre ensemble. Les deux ensembles ont donc la même somme. On ajoute le 1 à $S_1$ Soit 

  %   $$\begin{align*}
  %     S_1 = & \left\{1, 2, \dots, \frac{p}{2} + 1, 2p - \frac{p}{2}+2, 2p - \frac{p}{2}+3, \dots, 2p + 1\right\}\\
  %     S_2 = & \left\{\frac{p}{2} + 2, \frac{p}{2} + 3, \dots, 2p - \frac{p}{2} + 1\right\}.
  %   \end{align*}$$

  %   Quand $p$ est impair on met les nombres des $\frac{p-1}{2}$ premières colonnes dans un ensemble et ceux des $\frac{p-1}{2}$ colonnes suivantes dans l'autre ensemble. Il reste la dernière colonne contenant $p+1$ et $p+2$, on met donc $p+1$ dans $S_1$ et $p+2$ dans $S_2$. On ajoute le 1 à $S_1$. Soit 
    
  %   $$\begin{align*}
  %     S_1 = & \left\{1, 2, \dots, \frac{p+1}{2}, p+1, 2p - \frac{p-1}{2}+2, 2p - \frac{p-1}{2}+3, \dots, 2p + 1\right\}\\
  %     S_2 = & \left\{\frac{p+1}{2} + 1, \frac{p+1}{2} + 2, \dots, p, p+2, p+3, \dots, 2p - \frac{p-1}{2} + 1\right\}.
  %   \end{align*}$$
  % \end{solution}

  \begin{exercice}{}
    Trouver une partition parfaite pour le multiensemble $S=\{5, 7, 5, 8, 2, 9\}$.
  \end{exercice}

  % \begin{solution}
  %   $S_1 = \{5, 5, 8\} (18), S_2 = \{7, 2, 9\} (18)$
  % \end{solution}

  \begin{exercice}{}
    Trouver une partition parfaite pour le multiensemble $S=\{13, 11, 8, 20, 20, 11, 16, 17,$ $ 4, 13\}$.
  \end{exercice}

  % \begin{solution}
  %   $S_1 = \{13, 20, 16, 17\} (66), S_2 = \{11, 8, 20, 11, 4, 13\} (67)$
  % \end{solution}

  \begin{remarque}{}
    Ce problème admet une symétrie évidente puisque l'on peut inverser la partition, c'est-à-dire échanger les ensembles $S_1$ et $S_2$. 
    De manière générale, on peut toujours échanger deux nombres égaux entre $S_1$ et $S_2$ sans changer leurs sommes.  
  \end{remarque}

  \section{Problèmes connexes}

  \subsection{Subset Sum Problem}
  \cite{SSP}

  \subsection{Ordonnancement}
  \cite{MS}
  \cite{Korf2013}

  \subsection{Sac-à-dos}
  \cite{KP}
  \cite{MartelloToth1990}

  \begin{tikzpicture}
    \filldraw [fill = bleuTN] (0, 2) rectangle (8, 3);
    \filldraw [fill = rouge] (8, 2) rectangle (14, 3);

    \draw [<->, very thick, exCol] (7, 3.25) -- (8, 3.25) node [right] {Ordonnancement};
    
    \filldraw [fill = rouge] (0, 0) rectangle (6, 1);
    \filldraw [fill = bleuTN] (6, 0) rectangle (14, 1);

    \draw [<->, very thick, orange] (7, -0.25) -- (6, -0.25) node [left] {Sac-à-dos};

    \draw [<->, very thick, violet] (6.5, 1.5) -- (7.5, 1.5) node [right] {Partition};

    \draw [dashed, very thick] (7, -0.75) -- (7, 3.75) node [above] {C};
  \end{tikzpicture}


  \section{Formulations mathématiques}

  Nous allons exprimer le problème de partitionnement sous la forme d'un programme linéaire en nombres entiers \cite{PLNE}.
  
  \begin{definition}{}
    Soit un multiensemble $S$ de $n$ entiers naturels. 

    $$S = \{s_i\ |\ s_i > 0\}_{1\leq i \leq n}.$$

    La variable booléenne 

    $$x_i \in \{0,1\} \quad 1 \leq i \leq n$$

    vaut 1 si $s_i$ appartient à $S_1$ et 0 si $s_i$ appartient à $S_2$.

    On pourra représenter ces variables par le multiensemble $X$ 

    $$X = \{x_i\}_{1\leq i \leq n}.$$
  \end{definition}

  Ces variables représentent la fonction indicatrice \cite{FI} de $S_1$ dans $S$.
  En inversant les valeurs booléennes des variables, on obtient la fonction indicatrice de $S_2$.
  Donc, une affectation des variables $x_i$ représente une partition de l'ensemble $S$. 
  
  \begin{exemple}{}
    $S = \{1,2,3,4,5\}$ est un multiensemble.

    \begin{itemize}
      \item La partition $S_1 = \{1,3\}$ et $S_2 = \{2,4\}$ correspond à l'affectation des variables $x_i$ suivante $X = \{1,0,1,0\}$
      \item L'affectation des variables $x_i$, $X = \{1,0,0,1\}$ correspond à la partition suivante $S_1 = \{1, 4\}$ et $S_2 = \{2,3\}$.
    \end{itemize}
  \end{exemple}
  
  \begin{definition}{Partition optimale}
    Une \emph{partition optimale} est une affectation des variables $x_i$ minimisant :
    
    $$\left| \sum_{i=1}^n s_i x_i  - \sum_{i=1}^n s_i(1-x_i) \right| 
    = 
    \left| \sum_{i=1}^n s_i (2x_i-1) \right|.$$

    Cette fonction objectif n'est pas linéaire à strictement parler, car la valeur absolue n'est pas une fonction linéaire, mais elle peut se linéariser.
  \end{definition}

  \begin{remarque}{}
    On peut reformuler le problème comme un Subset Sum Problem \cite{SSP}.

    \begin{eqnarray*}
      \max & \sum\limits_{i=1}^n s_i x_i \\
      &\sum\limits_{i=1}^n s_i x_i  \leq \left\lfloor \frac{\sum\limits_{i=1}^n s_i}{2} \right\rfloor 
    \end{eqnarray*}

    Il existe une partition parfaite si la contrainte est saturée.

    L'inversion de S1 et S2 n'est éliminée que pour les multiensembles impairs : la somme des entiers de $S_1$ est inférieure ou égale à celle de $S_2$.
  \end{remarque}

  \section{Algorithmes gloutons}

  \begin{definition}{Algorithme}
    Un \emph{algorithme} répond à un problème. Il est composé d’un ensemble d’étapes simples nécessaires à la résolution, dont le nombre varie en fonction de la taille des données.
  \end{definition}

  \begin{remarque}{}
    Plusieurs algorithmes peuvent répondre à un même problème.
  \end{remarque}

  \begin{remarque}{}
    Un algorithme peut répondre à plusieurs problèmes.
  \end{remarque}

  \begin{definition}{Algorithme glouton}
    Un \emph{algorithme glouton} (greedy algorithm en anglais) est un algorithme qui suit le principe de faire, étape par étape, un choix optimum local. Dans certains cas cette approche permet d'arriver à un \emph{optimum global}, mais dans le cas général c'est une heuristique.
  \end{definition}

  \begin{algorithme}{Algorithme glouton GS}
    Soit $C$ la capacité maximale :  
    
    $$C = \left\lfloor \frac{\sum_{i=1}^n s_i}{2} \right\rfloor$$
    
    GS itère sur les entiers de $S$ \sout{triés par ordre décroissant} en les ajoutant à $S_1$ si la capacité $C$ n'est pas dépassée.
  \end{algorithme}

  \begin{exemple}{}
    Soit $S = \{5, 7, 5, 8, 2, 9\}$ un multiensemble. Ce multiensemble admet une partition parfaite :

    \begin{itemize}
      \item $S_1 = \{5, 5, 8\} (18)$
      \item $S_2 = \{7, 2, 9\} (18)$
    \end{itemize}

    avec $C = 18$.

    En utilisant l'algorithme glouton GS, on obtient la partition suivante : 

    \begin{itemize}
      \item $S_1 = \{5, 7, 5\} (17)$
      \item $S_2 = \{8, 2, 9\} (19)$
    \end{itemize}

    Si on trie au préalable les éléments de $S$ par ordre décroissant, $S = \{9, 8, 7, 5, 5, 2\}$, puis on applique l'algorithme glouton GS, on obtient la partition suivante :

    \begin{itemize}
      \item $S_1 = \{9, 8\} (17)$
      \item $S_2 = \{7, 5, 5, 2\} (19)$
    \end{itemize}
  \end{exemple}

  \begin{exercice}{}
    Partitionner le multiensemble $S=\{13, 11, 8, 20, 20, 11, 16, 17, 4, 13\}$ avec l'algorithme glouton GS.
  \end{exercice}

  % \begin{solution}
  %   $S_1 = \{13, 11, 8, 20, 11\} (63), S_2 = \{20, 16, 17, 4, 13\} (70)$
  % \end{solution}

  \begin{exercice}{}
    Partitionner le multiensemble $S=\{13, 11, 8, 20, 20, 11, 16, 17, 4, 13\}$ avec l'algorithme glouton GS après avoir trié $S$ par ordre décroissant.
  \end{exercice}

  % \begin{solution}
  %   $S_1 = \{20, 20, 17, 8\} (65), S_2 = \{16, 13, 13, 11, 11, 4\} (68)$
  % \end{solution}

  \begin{algorithme}{Algorithme glouton LPT}
    LPT itère sur les entiers de $S$ \sout{triés par ordre décroissant} en les ajoutant à l'ensemble $S_1$ si la somme des entiers de $S_1$ est inférieure à celle de $S_2$, et $S_2$ sinon.
  \end{algorithme}

  \begin{exemple}{}
    Soit $S = \{5, 7, 5, 8, 2, 9\}$ un multiensemble.

    En utilisant l'algorithme glouton LPT, on obtient la partition suivante : 

    \begin{itemize}
      \item $S_1 = \{7,8\} (15)$
      \item $S_2 = \{5,5,2,9\} (21)$
    \end{itemize}

    Si on trie au préalable les éléments de $S$ par ordre décroissant, $S = \{9, 8, 7, 5, 5, 2\}$, puis on applique l'algorithme glouton LPT, on obtient la partition suivante :

    \begin{itemize}
      \item $S_1 = \{8,7,2\} (17)$
      \item $S_2 = \{9,5,5\} (19)$
    \end{itemize}
  \end{exemple}

  \begin{exercice}{}
    Partitionner le multiensemble $S=\{13, 11, 8, 20, 20, 11, 16, 17, 4, 13\}$ avec l'algorithme glouton LPT après avoir trié $S$ par ordre décroissant.
  \end{exercice}

  % \begin{solution}
  %   $S_1 = \{20, 16, 13, 11, 8\} (68), S_2 = \{20, 17, 13, 11, 4\} (65)$
  % \end{solution}

  \begin{algorithme}{Algorithme glouton MTGS}
    MTGS applique GS aux $k$ plus petits entiers de $S$ pour $k$ variant de $n$ à $1$.
  \end{algorithme}

  \begin{exemple}{}
    Soit $S = \{5, 7, 5, 8, 2, 9\}$ un multiensemble.

    $\vdots$

    $\vdots$

    $\vdots$
  \end{exemple}

  \begin{exercice}{}
    Est-ce que MTGS trouve une partition parfaite du multiensemble $S=\{13, 11, 8, 20, 20, 11,$ $ 16, 17, 4, 13\}$ ?
  \end{exercice}

  % \begin{algorithme}
  %   test
  % \end{algorithme}

  % \begin{pedagogie}
  %   test
  % \end{pedagogie}

  % \begin{plusloin}
  %   test
  % \end{plusloin}

  \bibliographystyle{plain}
  \bibliography{poly}
\end{document}


