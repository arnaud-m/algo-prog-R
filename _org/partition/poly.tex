\documentclass[11pt,a4paper]{article}

\usepackage[french]{babel}
\usepackage[utf8]{inputenc}

\usepackage[T1]{fontenc}


\usepackage{booktabs}
%% Camille, change the color as you wish
\usepackage[pdftex,dvipsnames,usenames]{xcolor}
\usepackage[pdftex,colorlinks=true,urlcolor=ForestGreen,citecolor=Blue,linkcolor=BrickRed]{hyperref} % Must be loaded before cleveref
\usepackage{paralist}


\usepackage{graphicx}
\graphicspath{{./fig}}

\newif\ifdraftmode
\draftmodetrue
% \draftmodefalse

\usepackage{amsmath,amsthm,amssymb}
\usepackage[normalem]{ulem}

\textwidth 16cm
%\textheight 22cm
\evensidemargin 0cm
\oddsidemargin 0cm

\usepackage{colors}
\usepackage{encarts}
\usepackage{tikzstyles}
\usepackage{theoremes}
\usepackage{macros}

\title{Activité : Partition parfaite}
\date{}

\newif\ifnotikz
\notikztrue
\notikzfalse % <---- comment/uncomment that line

\ifnotikz
\renewenvironment{tikzpicture}[1][]{
  \comment%
}{%
  \endcomment%
}
\fi

\begin{document}

\newif\ifshowsols
\showsolstrue
\showsolsfalse % <---- comment/uncomment that line

\ifdraftmode
\selectcolormodel{gray}
\fi

\maketitle
\tableofcontents






\section{Description du problème}


  \begin{definition}{Partition d'un ensemble}
    Un \emph{multiensemble} (parfois appelé sac) est un ensemble dans lequel chaque élément peut apparaître plusieurs fois.
    Soit un multiensemble $S$ de $n$ entiers naturels :

    $$S = \{s_i\ |\ s_i > 0\}_{1\leq i \leq n}.$$

    Une \emph{(bi)partition} de $S$ est constituée de deux sous-multiensembles $S_1$ et $S_2$ tels que :
    \begin{itemize}
      \item $S_1$ et $S_2$ sont non vides:  $S_1 \neq \emptyset$ et $S_2 \neq \emptyset$ ;
      \item $S_1$ et $S_2$ sont disjoints:  $S_1 \cap S_2 = \emptyset$ ;
      \item $S_1$ et $S_2$ recouvrent $S$:  $S_1 \cup S_2 = S$.
    \end{itemize}
  \end{definition}

  \begin{exemple}{Partition d'un ensemble}
    Soit un multiensemble $S = \{1,2,3,4,5\}$.
    \begin{itemize}
      \item Les multiensembles $S_1$ et $S_2$ forment une partition de $S$.
        \begin{itemize}
          \item $S_1 = \{1\}$ et $S_2 = \{2,3,4,5\}$
          \item $S_1 = \{2, 4\}$ et $S_2 = \{1,3,5\}$
        \end{itemize}
      \item Les multiensembles $S_1$ et $S_2$ ne forment pas une partition de $S$.
        \begin{itemize}
          \item $S_1 = \{1,2,3,4,5\}$ et $S_2 = \emptyset$, car $S_2$ est vide.
          \item $S_1 = \{1,2,3\}$ et $S_2 = \{3,4,5\}$, car leur intersection est non vide.
          \item $S_2 = \{1,2\}$ et $S_2 = \{4, 5\}$, car 3 est dans $S$, mais n'appartient ni à $S_1$ ni à $S_2$.
        \end{itemize}
    \end{itemize}
  \end{exemple}

  \begin{definition}{Partition parfaite d'un multiensemble pair}
    Un multiensemble d'entiers $S$ est dit \emph{pair} si la somme des entiers de $S$ est paire.

    Une \emph{partition parfaite} d'un multiensemble pair est une partition telle que la valeur absolue de la différence entre la somme des entiers de $S_1$ et la somme des entiers de $S_2$ est 0.
  \end{definition}

  \begin{exemple}{Partition parfaite d'un multiensemble pair}
    $S = \{1,2,3,4\}$ est un multiensemble pair.
    \begin{itemize}
      \item $S_1 = \{1,3\}$ et $S_2 = \{2,4\}$ ne forment pas une partition parfaite.
      \item $S_1 = \{1,4\}$ et $S_2 = \{2,3\}$ forment une partition parfaite.
    \end{itemize}
  \end{exemple}

  \begin{definition}{Partition parfaite d'un multiensemble impair}
    Un multiensemble d'entiers $S$ est dit \emph{impair} si la somme des entiers de $S$ est impaire.

    Une \emph{partition parfaite} d'un multiensemble impair est une partition telle que la valeur absolue de la différence entre la somme des entiers de $S_1$ et la somme des entiers de $S_2$ est 1.
  \end{definition}

  \begin{exemple}{Partition parfaite d'un multiensemble impair}
    $S = \{1,2,3,4, 5\}$ est un multiensemble impair.

    \begin{itemize}
      \item $S_1 = \{2, 4\}$ et $S_2 = \{1,3,5\}$ ne forment pas une partition parfaite.
      \item $S_1 = \{1, 2, 5\}$ et $S_2 = \{3,4\}$ forment une partition parfaite.
    \end{itemize}
  \end{exemple}

  \begin{definition}{Problème de partitionnement}
    En informatique, le problème de partitionnement consiste à déterminer si une partition parfaite d'un ensembles d'entiers existe. C'est un problème \emph{NP-complet}. Cependant, il existe plusieurs algorithmes qui résolvent efficacement le problème que ce soit de manière approchée ou optimale. Pour ces raisons, il est réputé
    \begin{center}
      \emph{le plus facile des problèmes difficiles}.
    \end{center}

\end{definition}
  % Le problème de partitionnement se décline aussi en problème d'optimisation dans lequel on recherche une partition minimisant la valeur absolue de la différence entre la somme des entiers des deux sous-ensembles de la partition.
  % Ce problème d'optimisation est \emph{NP-difficile}.

\begin{remarque}{}
    Ce problème admet une symétrie évidente puisque l'on peut inverser la partition, c'est-à-dire échanger les ensembles $S_1$ et $S_2$.
    De manière générale, on peut toujours échanger des objets entre $S_1$ et $S_2$ si cela ne change pas leurs sommes.
 \end{remarque}

\section{Partition parfaite des entiers de 1 à $n$}

\begin{exercice}{}
  Trouver une partition parfaite des entiers de 1 à $n$ pour $n =4, 5, 6, 7, 8$.
  \end{exercice}


  \begin{figure}[htbp]
    \centering
    \resizebox{0.6\linewidth}{!}{
      \documentclass[border = 5mm]{standalone}

\usepackage{colors}
\usepackage{tikzstyles}

\begin{document}
  \input{partition-8_tikz.tex}
\end{document}

    }
    \caption{Partition parfaite des entiers de 1 à $n$.}
  \end{figure}


  \begin{definition}{Algorithme}
    Un \emph{algorithme} répond à un problème. Il est composé d’un ensemble d’étapes simples nécessaires à la résolution, dont le nombre varie en fonction de la taille des données.
  \end{definition}

  \begin{remarque}{}
    Plusieurs algorithmes peuvent répondre à un même problème.
  \end{remarque}

  \begin{remarque}{}
    Un algorithme peut répondre à plusieurs problèmes.
  \end{remarque}

  \begin{exercice}{}
    Donner un algorithme pour trouver une partition parfaite des entiers de 1 à $n$.
  \end{exercice}

  \begin{indice}
    Distinguer les cas en fonction du reste $r$ de la division euclidienne de $n$ par 4. C'est-à-dire qu'il existe $k \geq 0$ et $0 \leq r \leq 3$ tels quel $n = 4 \times k + r$.
  \end{indice}


  \begin{algorithme}{Partition parfaite des entiers de 1 à $n$.}
    \begin{itemize}
    \item Soit $n = 4 \times k + r$ le quotient $k$ et le reste $r$ de la division euclidienne de $n$ par 4.
      \begin{itemize}
      \item Si $r=1$, alors éliminer l'objet 1.
      \item Si $r=2$, alors ranger l'objet 1 et éliminer l'objet 2.
      \item Si $r=3$, alors ranger les objets 1 et 2 et éliminer l'objet 3.
      \end{itemize}
  \item Répéter tant que le sac n'est pas rempli :
    \begin{itemize}
    \item ranger le plus petit et le plus grand objet.
    \end{itemize}
  \end{itemize}
  \end{algorithme}

  \begin{exercice}[label=ex-16]{}
    Trouver une partition parfaite des entiers de 1 à 18.
  \end{exercice}


\begin{remarque}{}
  Toutes les paires d'objets formées par l'algorithme ont la même somme.
\end{remarque}


\begin{exercice}{}
  Trouvez d'autres partitions parfaites par échanges successifs.
\end{exercice}



  \section{Algorithmes gloutons}

  \begin{definition}{Algorithme glouton}
    Un \emph{algorithme glouton} est un algorithme qui suit le principe de faire, étape par étape, un choix optimum local.
    Dans certains cas, cette approche aboutit à un optimum global, mais dans le cas général c'est une heuristique qui n'aboutit pas nécessairement à un optimum global.
  \end{definition}


  \begin{algorithme}{Algorithme glouton}
    \label{algo:gs}
    \begin{itemize}
    \item Déterminer la capacité du sac.
    \item Trier les objets par ordre décroissant.
    \item  Répéter tant qu'il reste des objets :
      \begin{itemize}
      \item ranger le plus grand objet dans le sac si sa capacité le permet ;
      \item Sinon, retirer l'objet ;
      \item Si le sac est rempli, arrêter.
      \end{itemize}
  \end{itemize}
  \end{algorithme}

  \begin{exercice}[label=ex1-GS]{}
Appliquer l'algorithme glouton sur une instance de type 1.
\begin{center}
\begin{tabular}{ccclc}
\toprule
\# & Type & Taille & Entiers & Capacité \\
\midrule
1 & 1 & 6 & 11, 8, 7, 5, 2, 1 & 17 \\
2 & 1 & 6 & 16, 11, 10, 8, 4, 1 & 25 \\
3 & 1 & 9 & 15, 13, 8, 7, 6, 5, 4, 2, 1 & 30 \\
4 & 1 & 9 & 16, 12, 10, 9, 6, 5, 3, 2, 1 & 32 \\
5 & 1 & 12 & 16, 15, 13, 12, 9, 8, 6, 5, 4, 3, 2, 1 & 47 \\
\bottomrule
\end{tabular}
\end{center}
\end{exercice}
\begin{figure}[htbp]
\centering
\resizebox{0.8\linewidth}{!}{
\begin{tikzpicture}
\pic at (0, -2) {subset sum = {capacity = 17, sizes = {11,8,7,5,2,1}, deck = {1,0,0,0,0,0}}};
\pic at (0, -4) {subset sum = {capacity = 17, sizes = {11,8,7,5,2,1}, deck = {1,2,0,0,0,0}}};
\pic at (0, -6) {subset sum = {capacity = 17, sizes = {11,8,7,5,2,1}, deck = {1,2,2,0,0,0}}};
\pic at (0, -8) {subset sum = {capacity = 17, sizes = {11,8,7,5,2,1}, deck = {1,2,2,1,0,0}}};
\pic at (0, -10) {subset sum = {capacity = 17, sizes = {11,8,7,5,2,1}, deck = {1,2,2,1,2,0}}};
\pic at (0, -12) {subset sum = {capacity = 17, sizes = {11,8,7,5,2,1}, deck = {1,2,2,1,2,1}}};
\end{tikzpicture}
}
\caption{Application de l'algorithme glouton sur l'instance \#1 de type 1.}
\end{figure}
\ifshowsols
\begin{figure}[htbp]
\centering
\resizebox{0.8\linewidth}{!}{
\begin{tikzpicture}
\pic at (0, -2) {subset sum = {capacity = 25, sizes = {16,11,10,8,4,1}, deck = {1,0,0,0,0,0}}};
\pic at (0, -4) {subset sum = {capacity = 25, sizes = {16,11,10,8,4,1}, deck = {1,2,0,0,0,0}}};
\pic at (0, -6) {subset sum = {capacity = 25, sizes = {16,11,10,8,4,1}, deck = {1,2,2,0,0,0}}};
\pic at (0, -8) {subset sum = {capacity = 25, sizes = {16,11,10,8,4,1}, deck = {1,2,2,1,0,0}}};
\pic at (0, -10) {subset sum = {capacity = 25, sizes = {16,11,10,8,4,1}, deck = {1,2,2,1,2,0}}};
\pic at (0, -12) {subset sum = {capacity = 25, sizes = {16,11,10,8,4,1}, deck = {1,2,2,1,2,1}}};
\end{tikzpicture}
}
\caption{Application de l'algorithme glouton sur l'instance \#2 de type 1.}
\end{figure}
\begin{figure}[htbp]
\centering
\resizebox{0.8\linewidth}{!}{
\begin{tikzpicture}
\pic at (0, -2) {subset sum = {capacity = 30, sizes = {15,13,8,7,6,5,4,2,1}, deck = {1,0,0,0,0,0,0,0,0}}};
\pic at (0, -4) {subset sum = {capacity = 30, sizes = {15,13,8,7,6,5,4,2,1}, deck = {1,1,0,0,0,0,0,0,0}}};
\pic at (0, -6) {subset sum = {capacity = 30, sizes = {15,13,8,7,6,5,4,2,1}, deck = {1,1,2,0,0,0,0,0,0}}};
\pic at (0, -8) {subset sum = {capacity = 30, sizes = {15,13,8,7,6,5,4,2,1}, deck = {1,1,2,2,0,0,0,0,0}}};
\pic at (0, -10) {subset sum = {capacity = 30, sizes = {15,13,8,7,6,5,4,2,1}, deck = {1,1,2,2,2,0,0,0,0}}};
\pic at (0, -12) {subset sum = {capacity = 30, sizes = {15,13,8,7,6,5,4,2,1}, deck = {1,1,2,2,2,2,0,0,0}}};
\pic at (0, -14) {subset sum = {capacity = 30, sizes = {15,13,8,7,6,5,4,2,1}, deck = {1,1,2,2,2,2,2,0,0}}};
\pic at (0, -16) {subset sum = {capacity = 30, sizes = {15,13,8,7,6,5,4,2,1}, deck = {1,1,2,2,2,2,2,1,0}}};
\end{tikzpicture}
}
\caption{Application de l'algorithme glouton sur l'instance \#3 de type 1.}
\end{figure}
\begin{figure}[htbp]
\centering
\resizebox{0.8\linewidth}{!}{
\begin{tikzpicture}
\pic at (0, -2) {subset sum = {capacity = 32, sizes = {16,12,10,9,6,5,3,2,1}, deck = {1,0,0,0,0,0,0,0,0}}};
\pic at (0, -4) {subset sum = {capacity = 32, sizes = {16,12,10,9,6,5,3,2,1}, deck = {1,1,0,0,0,0,0,0,0}}};
\pic at (0, -6) {subset sum = {capacity = 32, sizes = {16,12,10,9,6,5,3,2,1}, deck = {1,1,2,0,0,0,0,0,0}}};
\pic at (0, -8) {subset sum = {capacity = 32, sizes = {16,12,10,9,6,5,3,2,1}, deck = {1,1,2,2,0,0,0,0,0}}};
\pic at (0, -10) {subset sum = {capacity = 32, sizes = {16,12,10,9,6,5,3,2,1}, deck = {1,1,2,2,2,0,0,0,0}}};
\pic at (0, -12) {subset sum = {capacity = 32, sizes = {16,12,10,9,6,5,3,2,1}, deck = {1,1,2,2,2,2,0,0,0}}};
\pic at (0, -14) {subset sum = {capacity = 32, sizes = {16,12,10,9,6,5,3,2,1}, deck = {1,1,2,2,2,2,1,0,0}}};
\pic at (0, -16) {subset sum = {capacity = 32, sizes = {16,12,10,9,6,5,3,2,1}, deck = {1,1,2,2,2,2,1,2,0}}};
\pic at (0, -18) {subset sum = {capacity = 32, sizes = {16,12,10,9,6,5,3,2,1}, deck = {1,1,2,2,2,2,1,2,1}}};
\end{tikzpicture}
}
\caption{Application de l'algorithme glouton sur l'instance \#4 de type 1.}
\end{figure}
\begin{figure}[htbp]
\centering
\resizebox{0.8\linewidth}{!}{
\begin{tikzpicture}
\pic at (0, -2) {subset sum = {capacity = 47, sizes = {16,15,13,12,9,8,6,5,4,3,2,1}, deck = {1,0,0,0,0,0,0,0,0,0,0,0}}};
\pic at (0, -4) {subset sum = {capacity = 47, sizes = {16,15,13,12,9,8,6,5,4,3,2,1}, deck = {1,1,0,0,0,0,0,0,0,0,0,0}}};
\pic at (0, -6) {subset sum = {capacity = 47, sizes = {16,15,13,12,9,8,6,5,4,3,2,1}, deck = {1,1,1,0,0,0,0,0,0,0,0,0}}};
\pic at (0, -8) {subset sum = {capacity = 47, sizes = {16,15,13,12,9,8,6,5,4,3,2,1}, deck = {1,1,1,2,0,0,0,0,0,0,0,0}}};
\pic at (0, -10) {subset sum = {capacity = 47, sizes = {16,15,13,12,9,8,6,5,4,3,2,1}, deck = {1,1,1,2,2,0,0,0,0,0,0,0}}};
\pic at (0, -12) {subset sum = {capacity = 47, sizes = {16,15,13,12,9,8,6,5,4,3,2,1}, deck = {1,1,1,2,2,2,0,0,0,0,0,0}}};
\pic at (0, -14) {subset sum = {capacity = 47, sizes = {16,15,13,12,9,8,6,5,4,3,2,1}, deck = {1,1,1,2,2,2,2,0,0,0,0,0}}};
\pic at (0, -16) {subset sum = {capacity = 47, sizes = {16,15,13,12,9,8,6,5,4,3,2,1}, deck = {1,1,1,2,2,2,2,2,0,0,0,0}}};
\pic at (0, -18) {subset sum = {capacity = 47, sizes = {16,15,13,12,9,8,6,5,4,3,2,1}, deck = {1,1,1,2,2,2,2,2,2,0,0,0}}};
\pic at (0, -20) {subset sum = {capacity = 47, sizes = {16,15,13,12,9,8,6,5,4,3,2,1}, deck = {1,1,1,2,2,2,2,2,2,1,0,0}}};
\end{tikzpicture}
}
\caption{Application de l'algorithme glouton sur l'instance \#5 de type 1.}
\end{figure}
\fi

  \input{ex/ex2-GS}

  \begin{remarque}

    \begin{itemize}
    \item Trouver une autre solution en changeant l'ordre du glouton.
    \item Remarquer que plusieurs ordres donne la même solution : échanger 8 et 7 dans \#1.
    \item Le glouton répété garantit de trouver des solutions distinctes.
    \end{itemize}
  \end{remarque}

  \begin{algorithme}{Algorithme glouton répété}
     \label{algo:mtgs}
    Répéter jusqu'à ce que le sac contienne tous les objets :
    \begin{itemize}
    \item appliquer l'algorithme glouton ;
    \item Si le sac est rempli, arrêter;
    \item Sinon éliminer le plus grand objet.
    \end{itemize}
  \end{algorithme}

  \begin{exercice}[label=ex2-MTGS]{}
Appliquer l'algorithme glouton répété sur une instance de type 2.
\begin{center}
\begin{tabular}{ccclc}
\toprule
\# & Type & Taille & Entiers & Capacité \\
\midrule
6 & 2 & 6 & 14, 13, 11, 7, 5, 3 & 26 \\
7 & 2 & 6 & 13, 12, 11, 10, 7, 4 & 28 \\
8 & 2 & 9 & 16, 15, 14, 13, 12, 9, 8, 6, 1 & 47 \\
9 & 2 & 9 & 15, 14, 13, 12, 11, 10, 7, 4, 3 & 44 \\
10 & 2 & 12 & 16, 15, 13, 12, 11, 10, 8, 7, 6, 4, 3, 2 & 53 \\
\bottomrule
\end{tabular}
\end{center}
\end{exercice}
\begin{figure}[htbp]
\centering
\resizebox{0.8\linewidth}{!}{
\begin{tikzpicture}
\pic at (0, 0) {subset sum = {capacity = 26, sizes = {14,13,11,7,5,3}, deck = {1,2,1,2,2,2}}};
\pic at (0, -4) {subset sum = {capacity = 26, sizes = {14,13,11,7,5,3}, deck = {2,1,1,2,2,2}}};
\pic at (0, -8) {subset sum = {capacity = 26, sizes = {14,13,11,7,5,3}, deck = {2,2,1,1,1,1}}};
\end{tikzpicture}
}
\caption{Application de l'algorithme glouton répété sur l'instance \#6 de type 2.}
\end{figure}
\ifshowsols
\begin{figure}[htbp]
\centering
\resizebox{0.8\linewidth}{!}{
\begin{tikzpicture}
\pic at (0, 0) {subset sum = {capacity = 28, sizes = {13,12,11,10,7,4}, deck = {1,1,2,2,2,2}}};
\pic at (0, -4) {subset sum = {capacity = 28, sizes = {13,12,11,10,7,4}, deck = {2,1,1,2,2,1}}};
\pic at (0, -8) {subset sum = {capacity = 28, sizes = {13,12,11,10,7,4}, deck = {2,2,1,1,1,0}}};
\end{tikzpicture}
}
\caption{Application de l'algorithme glouton répété sur l'instance \#7 de type 2.}
\end{figure}
\begin{figure}[htbp]
\centering
\resizebox{0.8\linewidth}{!}{
\begin{tikzpicture}
\pic at (0, 0) {subset sum = {capacity = 47, sizes = {16,15,14,13,12,9,8,6,1}, deck = {1,1,1,2,2,2,2,2,1}}};
\pic at (0, -4) {subset sum = {capacity = 47, sizes = {16,15,14,13,12,9,8,6,1}, deck = {2,1,1,1,2,2,2,2,1}}};
\pic at (0, -8) {subset sum = {capacity = 47, sizes = {16,15,14,13,12,9,8,6,1}, deck = {2,2,1,1,1,2,1,0,0}}};
\end{tikzpicture}
}
\caption{Application de l'algorithme glouton répété sur l'instance \#8 de type 2.}
\end{figure}
\begin{figure}[htbp]
\centering
\resizebox{0.8\linewidth}{!}{
\begin{tikzpicture}
\pic at (0, 0) {subset sum = {capacity = 44, sizes = {15,14,13,12,11,10,7,4,3}, deck = {1,1,1,2,2,2,2,2,2}}};
\pic at (0, -4) {subset sum = {capacity = 44, sizes = {15,14,13,12,11,10,7,4,3}, deck = {2,1,1,1,2,2,2,1,2}}};
\pic at (0, -8) {subset sum = {capacity = 44, sizes = {15,14,13,12,11,10,7,4,3}, deck = {2,2,1,1,1,2,1,2,2}}};
\pic at (0, -12) {subset sum = {capacity = 44, sizes = {15,14,13,12,11,10,7,4,3}, deck = {2,2,2,1,1,1,1,1,0}}};
\end{tikzpicture}
}
\caption{Application de l'algorithme glouton répété sur l'instance \#9 de type 2.}
\end{figure}
\begin{figure}[htbp]
\centering
\resizebox{0.8\linewidth}{!}{
\begin{tikzpicture}
\pic at (0, 0) {subset sum = {capacity = 53, sizes = {16,15,13,12,11,10,8,7,6,4,3,2}, deck = {1,1,1,2,2,2,1,2,2,2,2,2}}};
\pic at (0, -4) {subset sum = {capacity = 53, sizes = {16,15,13,12,11,10,8,7,6,4,3,2}, deck = {2,1,1,1,1,2,2,2,2,2,2,1}}};
\end{tikzpicture}
}
\caption{Application de l'algorithme glouton répété sur l'instance \#10 de type 2.}
\end{figure}
\fi

  \begin{exercice}[label=ex3-MTGS]{}
Appliquer l'algorithme glouton répété sur une instance de type 3.
\begin{center}
\begin{tabular}{ccclc}
\toprule
\# & Type & Taille & Entiers & Capacité \\
\midrule
11 & 3 & 6 & 13, 11, 9, 8, 6, 4 & 25 \\
12 & 3 & 6 & 16, 13, 12, 11, 7, 3 & 31 \\
13 & 3 & 9 & 16, 15, 14, 10, 9, 8, 6, 5, 3 & 43 \\
14 & 3 & 9 & 16, 15, 13, 11, 9, 7, 5, 4, 3 & 41 \\
15 & 3 & 12 & 16, 15, 14, 13, 12, 11, 10, 8, 7, 6, 4, 3 & 59 \\
\bottomrule
\end{tabular}
\end{center}
\end{exercice}
\begin{figure}[htbp]
\centering
\resizebox{0.8\linewidth}{!}{
\begin{tikzpicture}
\pic at (0, 0) {subset sum = {capacity = 25, sizes = {13,11,9,8,6,4}, deck = {1,1,2,2,2,2}}};
\pic at (0, -4) {subset sum = {capacity = 25, sizes = {13,11,9,8,6,4}, deck = {2,1,1,2,2,1}}};
\pic at (0, -8) {subset sum = {capacity = 25, sizes = {13,11,9,8,6,4}, deck = {2,2,1,1,1,2}}};
\pic at (0, -12) {subset sum = {capacity = 25, sizes = {13,11,9,8,6,4}, deck = {2,2,2,1,1,1}}};
\end{tikzpicture}
}
\caption{Application de l'algorithme glouton répété sur l'instance \#11 de type 3.}
\end{figure}
\ifshowsols
\begin{figure}[htbp]
\centering
\resizebox{0.8\linewidth}{!}{
\begin{tikzpicture}
\pic at (0, 0) {subset sum = {capacity = 31, sizes = {16,13,12,11,7,3}, deck = {1,1,2,2,2,2}}};
\pic at (0, -4) {subset sum = {capacity = 31, sizes = {16,13,12,11,7,3}, deck = {2,1,1,2,2,1}}};
\pic at (0, -8) {subset sum = {capacity = 31, sizes = {16,13,12,11,7,3}, deck = {2,2,1,1,1,2}}};
\pic at (0, -12) {subset sum = {capacity = 31, sizes = {16,13,12,11,7,3}, deck = {2,2,2,1,1,1}}};
\end{tikzpicture}
}
\caption{Application de l'algorithme glouton répété sur l'instance \#12 de type 3.}
\end{figure}
\begin{figure}[htbp]
\centering
\resizebox{0.8\linewidth}{!}{
\begin{tikzpicture}
\pic at (0, 0) {subset sum = {capacity = 43, sizes = {16,15,14,10,9,8,6,5,3}, deck = {1,1,2,1,2,2,2,2,2}}};
\pic at (0, -4) {subset sum = {capacity = 43, sizes = {16,15,14,10,9,8,6,5,3}, deck = {2,1,1,1,2,2,2,2,1}}};
\pic at (0, -8) {subset sum = {capacity = 43, sizes = {16,15,14,10,9,8,6,5,3}, deck = {2,2,1,1,1,1,2,2,2}}};
\pic at (0, -12) {subset sum = {capacity = 43, sizes = {16,15,14,10,9,8,6,5,3}, deck = {2,2,2,1,1,1,1,1,1}}};
\end{tikzpicture}
}
\caption{Application de l'algorithme glouton répété sur l'instance \#13 de type 3.}
\end{figure}
\begin{figure}[htbp]
\centering
\resizebox{0.8\linewidth}{!}{
\begin{tikzpicture}
\pic at (0, 0) {subset sum = {capacity = 41, sizes = {16,15,13,11,9,7,5,4,3}, deck = {1,1,2,2,1,2,2,2,2}}};
\pic at (0, -4) {subset sum = {capacity = 41, sizes = {16,15,13,11,9,7,5,4,3}, deck = {2,1,1,1,2,2,2,2,2}}};
\pic at (0, -8) {subset sum = {capacity = 41, sizes = {16,15,13,11,9,7,5,4,3}, deck = {2,2,1,1,1,1,2,2,2}}};
\pic at (0, -12) {subset sum = {capacity = 41, sizes = {16,15,13,11,9,7,5,4,3}, deck = {2,2,2,1,1,1,1,1,1}}};
\end{tikzpicture}
}
\caption{Application de l'algorithme glouton répété sur l'instance \#14 de type 3.}
\end{figure}
\begin{figure}[htbp]
\centering
\resizebox{0.8\linewidth}{!}{
\begin{tikzpicture}
\pic at (0, 0) {subset sum = {capacity = 59, sizes = {16,15,14,13,12,11,10,8,7,6,4,3}, deck = {1,1,1,1,2,2,2,2,2,2,2,2}}};
\pic at (0, -4) {subset sum = {capacity = 59, sizes = {16,15,14,13,12,11,10,8,7,6,4,3}, deck = {2,1,1,1,1,2,2,2,2,2,1,2}}};
\pic at (0, -8) {subset sum = {capacity = 59, sizes = {16,15,14,13,12,11,10,8,7,6,4,3}, deck = {2,2,1,1,1,1,2,1,2,2,2,2}}};
\pic at (0, -12) {subset sum = {capacity = 59, sizes = {16,15,14,13,12,11,10,8,7,6,4,3}, deck = {2,2,2,1,1,1,1,1,2,2,1,2}}};
\pic at (0, -16) {subset sum = {capacity = 59, sizes = {16,15,14,13,12,11,10,8,7,6,4,3}, deck = {2,2,2,2,1,1,1,1,1,1,1,2}}};
\pic at (0, -20) {subset sum = {capacity = 59, sizes = {16,15,14,13,12,11,10,8,7,6,4,3}, deck = {2,2,2,2,2,1,1,1,1,1,1,1}}};
\end{tikzpicture}
}
\caption{Application de l'algorithme glouton répété sur l'instance \#15 de type 3.}
\end{figure}
\fi



  \section{Programmation dynamique}

  \begin{algorithme}{Algorithme de programmation dynamique}
    \label{algo-DP}

    \begin{itemize}
    \item Déterminer la capacité du sac.
    \item Trier les objets par ordre décroissant.
    \item  Répéter tant qu'il reste des objets :
      \begin{itemize}
      \item répéter pour chaque case marquée en partant de la dernière :
      \begin{itemize}
      \item déterminer la case atteinte en rangeant l'objet immédiatement après la case marquée (utiliser l'objet comme règle) ;
      \item Si la case atteinte n'est pas marquée, placer un marqueur de l'objet.
      \end{itemize}
    \item Si le sac est rempli (la dernière case est marquée), alors arrêter.
      \end{itemize}
    \end{itemize}
  \end{algorithme}

\begin{algorithme}{Reconstruction du sac en programmation dynamique}
  Tant qu'il reste des marqueurs dans le sac :
  \begin{itemize}
  \item sélectionner le marqueur le plus à droite ;
  \item ranger l'objet à la place du marqueur en retirant des marqueurs si nécessaire.
  \end{itemize}
  \end{algorithme}

  \begin{exercice}[label=ex3-DP]{}
Appliquer la programmation dynamique sur une instance de type 3.
\begin{center}
\begin{tabular}{ccclc}
\toprule
\# & Type & Taille & Entiers & Capacité \\
\midrule
11 & 3 & 6 & 13, 11, 9, 8, 6, 4 & 25 \\
12 & 3 & 6 & 16, 13, 12, 11, 7, 3 & 31 \\
13 & 3 & 9 & 16, 15, 14, 10, 9, 8, 6, 5, 3 & 43 \\
14 & 3 & 9 & 16, 15, 13, 11, 9, 7, 5, 4, 3 & 41 \\
15 & 3 & 12 & 16, 15, 14, 13, 12, 11, 10, 8, 7, 6, 4, 3 & 59 \\
\bottomrule
\end{tabular}
\end{center}
\end{exercice}
\begin{figure}[htbp]
\centering
\resizebox{0.8\linewidth}{!}{
\begin{tikzpicture}
\pic at (0, -4) {prog dyn = {capacity = 25, sizes = {13,11,9,8,6,4}, deck = {1,0,0,0,0,0}, marks = {13/1}}};
\pic at (0, -8) {prog dyn = {capacity = 25, sizes = {13,11,9,8,6,4}, deck = {1,1,0,0,0,0}, marks = {11/2, 13/1, 24/2}}};
\pic at (0, -12) {prog dyn = {capacity = 25, sizes = {13,11,9,8,6,4}, deck = {1,1,1,0,0,0}, marks = {9/3, 11/2, 13/1, 20/3, 22/3, 24/2}}};
\pic at (0, -16) {prog dyn = {capacity = 25, sizes = {13,11,9,8,6,4}, deck = {1,1,1,1,0,0}, marks = {8/4, 9/3, 11/2, 13/1, 17/4, 19/4, 20/3, 21/4, 22/3, 24/2}}};
\pic at (0, -20) {prog dyn = {capacity = 25, sizes = {13,11,9,8,6,4}, deck = {1,1,1,1,1,0}, marks = {6/5, 8/4, 9/3, 11/2, 13/1, 14/5, 15/5, 17/4, 19/4, 20/3, 21/4, 22/3, 23/5, 24/2, 25/5}}};
\pic at (0, 0) {subset sum = {capacity = 25, sizes = {13,11,9,8,6,4}, deck = {2,1,2,1,1,0}}};
\end{tikzpicture}
}
\caption{Application de la programmation dynamique sur l'instance \#11 de type 3.}
\end{figure}
\ifshowsols
\begin{figure}[htbp]
\centering
\resizebox{0.8\linewidth}{!}{
\begin{tikzpicture}
\pic at (0, -4) {prog dyn = {capacity = 31, sizes = {16,13,12,11,7,3}, deck = {1,0,0,0,0,0}, marks = {16/1}}};
\pic at (0, -8) {prog dyn = {capacity = 31, sizes = {16,13,12,11,7,3}, deck = {1,1,0,0,0,0}, marks = {13/2, 16/1, 29/2}}};
\pic at (0, -12) {prog dyn = {capacity = 31, sizes = {16,13,12,11,7,3}, deck = {1,1,1,0,0,0}, marks = {12/3, 13/2, 16/1, 25/3, 28/3, 29/2}}};
\pic at (0, -16) {prog dyn = {capacity = 31, sizes = {16,13,12,11,7,3}, deck = {1,1,1,1,0,0}, marks = {11/4, 12/3, 13/2, 16/1, 23/4, 24/4, 25/3, 27/4, 28/3, 29/2}}};
\pic at (0, -20) {prog dyn = {capacity = 31, sizes = {16,13,12,11,7,3}, deck = {1,1,1,1,1,0}, marks = {7/5, 11/4, 12/3, 13/2, 16/1, 18/5, 19/5, 20/5, 23/4, 24/4, 25/3, 27/4, 28/3, 29/2, 30/5, 31/5}}};
\pic at (0, 0) {subset sum = {capacity = 31, sizes = {16,13,12,11,7,3}, deck = {2,1,2,1,1,0}}};
\end{tikzpicture}
}
\caption{Application de la programmation dynamique sur l'instance \#12 de type 3.}
\end{figure}
\begin{figure}[htbp]
\centering
\resizebox{0.8\linewidth}{!}{
\begin{tikzpicture}
\pic at (0, -4) {prog dyn = {capacity = 43, sizes = {16,15,14,10,9,8,6,5,3}, deck = {1,0,0,0,0,0,0,0,0}, marks = {16/1}}};
\pic at (0, -8) {prog dyn = {capacity = 43, sizes = {16,15,14,10,9,8,6,5,3}, deck = {1,1,0,0,0,0,0,0,0}, marks = {15/2, 16/1, 31/2}}};
\pic at (0, -12) {prog dyn = {capacity = 43, sizes = {16,15,14,10,9,8,6,5,3}, deck = {1,1,1,0,0,0,0,0,0}, marks = {14/3, 15/2, 16/1, 29/3, 30/3, 31/2}}};
\pic at (0, -16) {prog dyn = {capacity = 43, sizes = {16,15,14,10,9,8,6,5,3}, deck = {1,1,1,1,0,0,0,0,0}, marks = {10/4, 14/3, 15/2, 16/1, 24/4, 25/4, 26/4, 29/3, 30/3, 31/2, 39/4, 40/4, 41/4}}};
\pic at (0, -20) {prog dyn = {capacity = 43, sizes = {16,15,14,10,9,8,6,5,3}, deck = {1,1,1,1,1,0,0,0,0}, marks = {9/5, 10/4, 14/3, 15/2, 16/1, 19/5, 23/5, 24/4, 25/4, 26/4, 29/3, 30/3, 31/2, 33/5, 34/5, 35/5, 38/5, 39/4, 40/4, 41/4}}};
\pic at (0, -24) {prog dyn = {capacity = 43, sizes = {16,15,14,10,9,8,6,5,3}, deck = {1,1,1,1,1,1,0,0,0}, marks = {8/6, 9/5, 10/4, 14/3, 15/2, 16/1, 17/6, 18/6, 19/5, 22/6, 23/5, 24/4, 25/4, 26/4, 27/6, 29/3, 30/3, 31/2, 32/6, 33/5, 34/5, 35/5, 37/6, 38/5, 39/4, 40/4, 41/4, 42/6, 43/6}}};
\pic at (0, 0) {subset sum = {capacity = 43, sizes = {16,15,14,10,9,8,6,5,3}, deck = {1,2,2,1,1,1,0,0,0}}};
\end{tikzpicture}
}
\caption{Application de la programmation dynamique sur l'instance \#13 de type 3.}
\end{figure}
\begin{figure}[htbp]
\centering
\resizebox{0.8\linewidth}{!}{
\begin{tikzpicture}
\pic at (0, -4) {prog dyn = {capacity = 41, sizes = {16,15,13,11,9,7,5,4,3}, deck = {1,0,0,0,0,0,0,0,0}, marks = {16/1}}};
\pic at (0, -8) {prog dyn = {capacity = 41, sizes = {16,15,13,11,9,7,5,4,3}, deck = {1,1,0,0,0,0,0,0,0}, marks = {15/2, 16/1, 31/2}}};
\pic at (0, -12) {prog dyn = {capacity = 41, sizes = {16,15,13,11,9,7,5,4,3}, deck = {1,1,1,0,0,0,0,0,0}, marks = {13/3, 15/2, 16/1, 28/3, 29/3, 31/2}}};
\pic at (0, -16) {prog dyn = {capacity = 41, sizes = {16,15,13,11,9,7,5,4,3}, deck = {1,1,1,1,0,0,0,0,0}, marks = {11/4, 13/3, 15/2, 16/1, 24/4, 26/4, 27/4, 28/3, 29/3, 31/2, 39/4, 40/4}}};
\pic at (0, -20) {prog dyn = {capacity = 41, sizes = {16,15,13,11,9,7,5,4,3}, deck = {1,1,1,1,1,0,0,0,0}, marks = {9/5, 11/4, 13/3, 15/2, 16/1, 20/5, 22/5, 24/4, 25/5, 26/4, 27/4, 28/3, 29/3, 31/2, 33/5, 35/5, 36/5, 37/5, 38/5, 39/4, 40/4}}};
\pic at (0, -24) {prog dyn = {capacity = 41, sizes = {16,15,13,11,9,7,5,4,3}, deck = {1,1,1,1,1,1,0,0,0}, marks = {7/6, 9/5, 11/4, 13/3, 15/2, 16/1, 18/6, 20/5, 22/5, 23/6, 24/4, 25/5, 26/4, 27/4, 28/3, 29/3, 31/2, 32/6, 33/5, 34/6, 35/5, 36/5, 37/5, 38/5, 39/4, 40/4}}};
\pic at (0, -28) {prog dyn = {capacity = 41, sizes = {16,15,13,11,9,7,5,4,3}, deck = {1,1,1,1,1,1,1,0,0}, marks = {5/7, 7/6, 9/5, 11/4, 12/7, 13/3, 14/7, 15/2, 16/1, 18/6, 20/5, 21/7, 22/5, 23/6, 24/4, 25/5, 26/4, 27/4, 28/3, 29/3, 30/7, 31/2, 32/6, 33/5, 34/6, 35/5, 36/5, 37/5, 38/5, 39/4, 40/4, 41/7}}};
\pic at (0, 0) {subset sum = {capacity = 41, sizes = {16,15,13,11,9,7,5,4,3}, deck = {1,2,2,1,1,2,1,0,0}}};
\end{tikzpicture}
}
\caption{Application de la programmation dynamique sur l'instance \#14 de type 3.}
\end{figure}
\begin{figure}[htbp]
\centering
\resizebox{0.8\linewidth}{!}{
\begin{tikzpicture}
\pic at (0, -4) {prog dyn = {capacity = 59, sizes = {16,15,14,13,12,11,10,8,7,6,4,3}, deck = {1,0,0,0,0,0,0,0,0,0,0,0}, marks = {16/1}}};
\pic at (0, -8) {prog dyn = {capacity = 59, sizes = {16,15,14,13,12,11,10,8,7,6,4,3}, deck = {1,1,0,0,0,0,0,0,0,0,0,0}, marks = {15/2, 16/1, 31/2}}};
\pic at (0, -12) {prog dyn = {capacity = 59, sizes = {16,15,14,13,12,11,10,8,7,6,4,3}, deck = {1,1,1,0,0,0,0,0,0,0,0,0}, marks = {14/3, 15/2, 16/1, 29/3, 30/3, 31/2, 45/3}}};
\pic at (0, -16) {prog dyn = {capacity = 59, sizes = {16,15,14,13,12,11,10,8,7,6,4,3}, deck = {1,1,1,1,0,0,0,0,0,0,0,0}, marks = {13/4, 14/3, 15/2, 16/1, 27/4, 28/4, 29/3, 30/3, 31/2, 42/4, 43/4, 44/4, 45/3, 58/4}}};
\pic at (0, -20) {prog dyn = {capacity = 59, sizes = {16,15,14,13,12,11,10,8,7,6,4,3}, deck = {1,1,1,1,1,0,0,0,0,0,0,0}, marks = {12/5, 13/4, 14/3, 15/2, 16/1, 25/5, 26/5, 27/4, 28/4, 29/3, 30/3, 31/2, 39/5, 40/5, 41/5, 42/4, 43/4, 44/4, 45/3, 54/5, 55/5, 56/5, 57/5, 58/4}}};
\pic at (0, -24) {prog dyn = {capacity = 59, sizes = {16,15,14,13,12,11,10,8,7,6,4,3}, deck = {1,1,1,1,1,1,0,0,0,0,0,0}, marks = {11/6, 12/5, 13/4, 14/3, 15/2, 16/1, 23/6, 24/6, 25/5, 26/5, 27/4, 28/4, 29/3, 30/3, 31/2, 36/6, 37/6, 38/6, 39/5, 40/5, 41/5, 42/4, 43/4, 44/4, 45/3, 50/6, 51/6, 52/6, 53/6, 54/5, 55/5, 56/5, 57/5, 58/4}}};
\pic at (0, -28) {prog dyn = {capacity = 59, sizes = {16,15,14,13,12,11,10,8,7,6,4,3}, deck = {1,1,1,1,1,1,1,0,0,0,0,0}, marks = {10/7, 11/6, 12/5, 13/4, 14/3, 15/2, 16/1, 21/7, 22/7, 23/6, 24/6, 25/5, 26/5, 27/4, 28/4, 29/3, 30/3, 31/2, 33/7, 34/7, 35/7, 36/6, 37/6, 38/6, 39/5, 40/5, 41/5, 42/4, 43/4, 44/4, 45/3, 46/7, 47/7, 48/7, 49/7, 50/6, 51/6, 52/6, 53/6, 54/5, 55/5, 56/5, 57/5, 58/4}}};
\pic at (0, -32) {prog dyn = {capacity = 59, sizes = {16,15,14,13,12,11,10,8,7,6,4,3}, deck = {1,1,1,1,1,1,1,1,0,0,0,0}, marks = {8/8, 10/7, 11/6, 12/5, 13/4, 14/3, 15/2, 16/1, 18/8, 19/8, 20/8, 21/7, 22/7, 23/6, 24/6, 25/5, 26/5, 27/4, 28/4, 29/3, 30/3, 31/2, 32/8, 33/7, 34/7, 35/7, 36/6, 37/6, 38/6, 39/5, 40/5, 41/5, 42/4, 43/4, 44/4, 45/3, 46/7, 47/7, 48/7, 49/7, 50/6, 51/6, 52/6, 53/6, 54/5, 55/5, 56/5, 57/5, 58/4, 59/8}}};
\pic at (0, 0) {subset sum = {capacity = 59, sizes = {16,15,14,13,12,11,10,8,7,6,4,3}, deck = {2,1,2,1,1,1,2,1,0,0,0,0}}};
\end{tikzpicture}
}
\caption{Application de la programmation dynamique sur l'instance \#15 de type 3.}
\end{figure}
\fi

  \begin{exercice}[label=ex4-DP]{}
Appliquer la programmation dynamique sur une instance de type 4.
\begin{center}
\begin{tabular}{ccclc}
\toprule
\# & Type & Taille & Entiers & Capacité \\
\midrule
16 & 4 & 6 & 16, 15, 11, 4, 2, 1 & 24 \\
17 & 4 & 6 & 15, 14, 9, 8, 6, 1 & 26 \\
18 & 4 & 8 & 18, 15, 13, 10, 8, 5, 3, 2 & 37 \\
19 & 4 & 8 & 18, 15, 13, 10, 8, 5, 3, 2 & 37 \\
20 & 4 & 8 & 18, 17, 16, 15, 14, 5, 2, 1 & 44 \\
\bottomrule
\end{tabular}
\end{center}
\end{exercice}
\begin{figure}[htbp]
\centering
\resizebox{0.8\linewidth}{!}{
\begin{tikzpicture}
\pic at (0, -4) {prog dyn = {capacity = 24, sizes = {16,15,11,4,2,1}, deck = {1,0,0,0,0,0}, marks = {16/1}}};
\pic at (0, -8) {prog dyn = {capacity = 24, sizes = {16,15,11,4,2,1}, deck = {1,1,0,0,0,0}, marks = {15/2, 16/1}}};
\pic at (0, -12) {prog dyn = {capacity = 24, sizes = {16,15,11,4,2,1}, deck = {1,1,1,0,0,0}, marks = {11/3, 15/2, 16/1}}};
\pic at (0, -16) {prog dyn = {capacity = 24, sizes = {16,15,11,4,2,1}, deck = {1,1,1,1,0,0}, marks = {4/4, 11/3, 15/2, 16/1, 19/4, 20/4}}};
\pic at (0, -20) {prog dyn = {capacity = 24, sizes = {16,15,11,4,2,1}, deck = {1,1,1,1,1,0}, marks = {2/5, 4/4, 6/5, 11/3, 13/5, 15/2, 16/1, 17/5, 18/5, 19/4, 20/4, 21/5, 22/5}}};
\pic at (0, -24) {prog dyn = {capacity = 24, sizes = {16,15,11,4,2,1}, deck = {1,1,1,1,1,1}, marks = {1/6, 2/5, 3/6, 4/4, 5/6, 6/5, 7/6, 11/3, 12/6, 13/5, 14/6, 15/2, 16/1, 17/5, 18/5, 19/4, 20/4, 21/5, 22/5, 23/6}}};
\pic at (0, 0) {subset sum = {capacity = 24, sizes = {16,15,11,4,2,1}, deck = {1,2,2,1,1,1}}};
\end{tikzpicture}
}
\caption{Application de la programmation dynamique sur l'instance \#16 de type 4.}
\end{figure}
\ifshowsols
\begin{figure}[htbp]
\centering
\resizebox{0.8\linewidth}{!}{
\begin{tikzpicture}
\pic at (0, -4) {prog dyn = {capacity = 26, sizes = {15,14,9,8,6,1}, deck = {1,0,0,0,0,0}, marks = {15/1}}};
\pic at (0, -8) {prog dyn = {capacity = 26, sizes = {15,14,9,8,6,1}, deck = {1,1,0,0,0,0}, marks = {14/2, 15/1}}};
\pic at (0, -12) {prog dyn = {capacity = 26, sizes = {15,14,9,8,6,1}, deck = {1,1,1,0,0,0}, marks = {9/3, 14/2, 15/1, 23/3, 24/3}}};
\pic at (0, -16) {prog dyn = {capacity = 26, sizes = {15,14,9,8,6,1}, deck = {1,1,1,1,0,0}, marks = {8/4, 9/3, 14/2, 15/1, 17/4, 22/4, 23/3, 24/3}}};
\pic at (0, -20) {prog dyn = {capacity = 26, sizes = {15,14,9,8,6,1}, deck = {1,1,1,1,1,0}, marks = {6/5, 8/4, 9/3, 14/2, 15/1, 17/4, 20/5, 21/5, 22/4, 23/3, 24/3}}};
\pic at (0, -24) {prog dyn = {capacity = 26, sizes = {15,14,9,8,6,1}, deck = {1,1,1,1,1,1}, marks = {1/6, 6/5, 7/6, 8/4, 9/3, 10/6, 14/2, 15/1, 16/6, 17/4, 18/6, 20/5, 21/5, 22/4, 23/3, 24/3, 25/6}}};
\pic at (0, 0) {subset sum = {capacity = 26, sizes = {15,14,9,8,6,1}, deck = {1,2,1,2,2,1}}};
\end{tikzpicture}
}
\caption{Application de la programmation dynamique sur l'instance \#17 de type 4.}
\end{figure}
\begin{figure}[htbp]
\centering
\resizebox{0.8\linewidth}{!}{
\begin{tikzpicture}
\pic at (0, -4) {prog dyn = {capacity = 37, sizes = {18,15,13,10,8,5,3,2}, deck = {1,0,0,0,0,0,0,0}, marks = {18/1}}};
\pic at (0, -8) {prog dyn = {capacity = 37, sizes = {18,15,13,10,8,5,3,2}, deck = {1,1,0,0,0,0,0,0}, marks = {15/2, 18/1, 33/2}}};
\pic at (0, -12) {prog dyn = {capacity = 37, sizes = {18,15,13,10,8,5,3,2}, deck = {1,1,1,0,0,0,0,0}, marks = {13/3, 15/2, 18/1, 28/3, 31/3, 33/2}}};
\pic at (0, -16) {prog dyn = {capacity = 37, sizes = {18,15,13,10,8,5,3,2}, deck = {1,1,1,1,0,0,0,0}, marks = {10/4, 13/3, 15/2, 18/1, 23/4, 25/4, 28/3, 31/3, 33/2}}};
\pic at (0, -20) {prog dyn = {capacity = 37, sizes = {18,15,13,10,8,5,3,2}, deck = {1,1,1,1,1,0,0,0}, marks = {8/5, 10/4, 13/3, 15/2, 18/1, 21/5, 23/4, 25/4, 26/5, 28/3, 31/3, 33/2, 36/5}}};
\pic at (0, -24) {prog dyn = {capacity = 37, sizes = {18,15,13,10,8,5,3,2}, deck = {1,1,1,1,1,1,0,0}, marks = {5/6, 8/5, 10/4, 13/3, 15/2, 18/1, 20/6, 21/5, 23/4, 25/4, 26/5, 28/3, 30/6, 31/3, 33/2, 36/5}}};
\pic at (0, -28) {prog dyn = {capacity = 37, sizes = {18,15,13,10,8,5,3,2}, deck = {1,1,1,1,1,1,1,0}, marks = {3/7, 5/6, 8/5, 10/4, 11/7, 13/3, 15/2, 16/7, 18/1, 20/6, 21/5, 23/4, 24/7, 25/4, 26/5, 28/3, 29/7, 30/6, 31/3, 33/2, 34/7, 36/5}}};
\pic at (0, -32) {prog dyn = {capacity = 37, sizes = {18,15,13,10,8,5,3,2}, deck = {1,1,1,1,1,1,1,1}, marks = {2/8, 3/7, 5/6, 7/8, 8/5, 10/4, 11/7, 12/8, 13/3, 15/2, 16/7, 17/8, 18/1, 20/6, 21/5, 22/8, 23/4, 24/7, 25/4, 26/5, 27/8, 28/3, 29/7, 30/6, 31/3, 32/8, 33/2, 34/7, 35/8, 36/5}}};
\pic at (0, 0) {subset sum = {capacity = 37, sizes = {18,15,13,10,8,5,3,2}, deck = {2,1,1,2,1,2,2,2}}};
\end{tikzpicture}
}
\caption{Application de la programmation dynamique sur l'instance \#18 de type 4.}
\end{figure}
\begin{figure}[htbp]
\centering
\resizebox{0.8\linewidth}{!}{
\begin{tikzpicture}
\pic at (0, -4) {prog dyn = {capacity = 37, sizes = {18,15,13,10,8,5,3,2}, deck = {1,0,0,0,0,0,0,0}, marks = {18/1}}};
\pic at (0, -8) {prog dyn = {capacity = 37, sizes = {18,15,13,10,8,5,3,2}, deck = {1,1,0,0,0,0,0,0}, marks = {15/2, 18/1, 33/2}}};
\pic at (0, -12) {prog dyn = {capacity = 37, sizes = {18,15,13,10,8,5,3,2}, deck = {1,1,1,0,0,0,0,0}, marks = {13/3, 15/2, 18/1, 28/3, 31/3, 33/2}}};
\pic at (0, -16) {prog dyn = {capacity = 37, sizes = {18,15,13,10,8,5,3,2}, deck = {1,1,1,1,0,0,0,0}, marks = {10/4, 13/3, 15/2, 18/1, 23/4, 25/4, 28/3, 31/3, 33/2}}};
\pic at (0, -20) {prog dyn = {capacity = 37, sizes = {18,15,13,10,8,5,3,2}, deck = {1,1,1,1,1,0,0,0}, marks = {8/5, 10/4, 13/3, 15/2, 18/1, 21/5, 23/4, 25/4, 26/5, 28/3, 31/3, 33/2, 36/5}}};
\pic at (0, -24) {prog dyn = {capacity = 37, sizes = {18,15,13,10,8,5,3,2}, deck = {1,1,1,1,1,1,0,0}, marks = {5/6, 8/5, 10/4, 13/3, 15/2, 18/1, 20/6, 21/5, 23/4, 25/4, 26/5, 28/3, 30/6, 31/3, 33/2, 36/5}}};
\pic at (0, -28) {prog dyn = {capacity = 37, sizes = {18,15,13,10,8,5,3,2}, deck = {1,1,1,1,1,1,1,0}, marks = {3/7, 5/6, 8/5, 10/4, 11/7, 13/3, 15/2, 16/7, 18/1, 20/6, 21/5, 23/4, 24/7, 25/4, 26/5, 28/3, 29/7, 30/6, 31/3, 33/2, 34/7, 36/5}}};
\pic at (0, -32) {prog dyn = {capacity = 37, sizes = {18,15,13,10,8,5,3,2}, deck = {1,1,1,1,1,1,1,1}, marks = {2/8, 3/7, 5/6, 7/8, 8/5, 10/4, 11/7, 12/8, 13/3, 15/2, 16/7, 17/8, 18/1, 20/6, 21/5, 22/8, 23/4, 24/7, 25/4, 26/5, 27/8, 28/3, 29/7, 30/6, 31/3, 32/8, 33/2, 34/7, 35/8, 36/5}}};
\pic at (0, 0) {subset sum = {capacity = 37, sizes = {18,15,13,10,8,5,3,2}, deck = {2,1,1,2,1,2,2,2}}};
\end{tikzpicture}
}
\caption{Application de la programmation dynamique sur l'instance \#19 de type 4.}
\end{figure}
\begin{figure}[htbp]
\centering
\resizebox{0.8\linewidth}{!}{
\begin{tikzpicture}
\pic at (0, -4) {prog dyn = {capacity = 44, sizes = {18,17,16,15,14,5,2,1}, deck = {1,0,0,0,0,0,0,0}, marks = {18/1}}};
\pic at (0, -8) {prog dyn = {capacity = 44, sizes = {18,17,16,15,14,5,2,1}, deck = {1,1,0,0,0,0,0,0}, marks = {17/2, 18/1, 35/2}}};
\pic at (0, -12) {prog dyn = {capacity = 44, sizes = {18,17,16,15,14,5,2,1}, deck = {1,1,1,0,0,0,0,0}, marks = {16/3, 17/2, 18/1, 33/3, 34/3, 35/2}}};
\pic at (0, -16) {prog dyn = {capacity = 44, sizes = {18,17,16,15,14,5,2,1}, deck = {1,1,1,1,0,0,0,0}, marks = {15/4, 16/3, 17/2, 18/1, 31/4, 32/4, 33/3, 34/3, 35/2}}};
\pic at (0, -20) {prog dyn = {capacity = 44, sizes = {18,17,16,15,14,5,2,1}, deck = {1,1,1,1,1,0,0,0}, marks = {14/5, 15/4, 16/3, 17/2, 18/1, 29/5, 30/5, 31/4, 32/4, 33/3, 34/3, 35/2}}};
\pic at (0, -24) {prog dyn = {capacity = 44, sizes = {18,17,16,15,14,5,2,1}, deck = {1,1,1,1,1,1,0,0}, marks = {5/6, 14/5, 15/4, 16/3, 17/2, 18/1, 19/6, 20/6, 21/6, 22/6, 23/6, 29/5, 30/5, 31/4, 32/4, 33/3, 34/3, 35/2, 36/6, 37/6, 38/6, 39/6, 40/6}}};
\pic at (0, -28) {prog dyn = {capacity = 44, sizes = {18,17,16,15,14,5,2,1}, deck = {1,1,1,1,1,1,1,0}, marks = {2/7, 5/6, 7/7, 14/5, 15/4, 16/3, 17/2, 18/1, 19/6, 20/6, 21/6, 22/6, 23/6, 24/7, 25/7, 29/5, 30/5, 31/4, 32/4, 33/3, 34/3, 35/2, 36/6, 37/6, 38/6, 39/6, 40/6, 41/7, 42/7}}};
\pic at (0, -32) {prog dyn = {capacity = 44, sizes = {18,17,16,15,14,5,2,1}, deck = {1,1,1,1,1,1,1,1}, marks = {1/8, 2/7, 3/8, 5/6, 6/8, 7/7, 8/8, 14/5, 15/4, 16/3, 17/2, 18/1, 19/6, 20/6, 21/6, 22/6, 23/6, 24/7, 25/7, 26/8, 29/5, 30/5, 31/4, 32/4, 33/3, 34/3, 35/2, 36/6, 37/6, 38/6, 39/6, 40/6, 41/7, 42/7, 43/8}}};
\pic at (0, 0) {subset sum = {capacity = 44, sizes = {18,17,16,15,14,5,2,1}, deck = {1,1,2,2,2,1,1,1}}};
\end{tikzpicture}
}
\caption{Application de la programmation dynamique sur l'instance \#20 de type 4.}
\end{figure}
\fi



  \section{Tester et Générer}

  \begin{algorithme}{Algorithme Tester-et-Générer}
    \begin{itemize}
    \item Déterminer la capacité.
    \item Trier les objets par ordre décroissant.
    \item  Descente : Répéter tant que le dernier objet n'est pas dans le sac :
      \begin{itemize}
      \item Répéter pour chaque objet :
        \begin{itemize}
        \item ranger l'objet dans le sac si la capacité le permet.
        \end{itemize}
      \item Si le sac est rempli, arrêter l'algorithme.
      \item Sinon, effectuer un retour arrière simple : retirer le plus petit objet du sac.
        \end{itemize}
      \item Retour arrière sautée quand le plus petit objet du deck est dans le sac.
      \begin{itemize}
      \item Retirer les objets du sac trouver jusqu'à ce que vous trouviez le plus petit objet que vous n'avez pas pu faire rentrer.
      \item Sinon éliminer l'objet.
      \end{itemize}
    % \item  Répéter tant qu'il reste des objets et que le sac n'est pas rempli (la dernière case n'est pas marquée) :
    %   \begin{itemize}
    %   \item répéter pour chaque case marquée en partant de la dernière :
    %   \begin{itemize}
    %   \item déterminer la case atteinte en rangeant l'objet immédiatement après la case marquée (utiliser l'objet comme règle).
    %   \item Si la case atteinte n'est pas marquée, placer un marqueur de l'objet.
    %   \end{itemize}
    %   \end{itemize}
    \end{itemize}
  \end{algorithme}



  \section{Problèmes connexes}

  % \subsection{Subset Sum Problem}
  % \cite{SSP}

  % \subsection{Ordonnancement}
  % \cite{MS}
  % \cite{Korf2013}

  % \subsection{Sac-à-dos}
  % \cite{KP}
  % \cite{MartelloToth1990}


  \begin{tabular}{ccclc}
\toprule
\# & Type & Taille & Entiers & Capacité \\
\midrule
1 & 1 & 6 & 11, 8, 7, 5, 2, 1 & 17 \\
2 & 1 & 6 & 16, 11, 10, 8, 4, 1 & 25 \\
3 & 1 & 9 & 15, 13, 8, 7, 6, 5, 4, 2, 1 & 30 \\
4 & 1 & 9 & 16, 12, 10, 9, 6, 5, 3, 2, 1 & 32 \\
5 & 1 & 12 & 16, 15, 13, 12, 9, 8, 6, 5, 4, 3, 2, 1 & 47 \\
6 & 2 & 6 & 14, 13, 11, 7, 5, 3 & 26 \\
7 & 2 & 6 & 13, 12, 11, 10, 7, 4 & 28 \\
8 & 2 & 9 & 16, 15, 14, 13, 12, 9, 8, 6, 1 & 47 \\
9 & 2 & 9 & 15, 14, 13, 12, 11, 10, 7, 4, 3 & 44 \\
10 & 2 & 12 & 16, 15, 13, 12, 11, 10, 8, 7, 6, 4, 3, 2 & 53 \\
11 & 3 & 6 & 13, 11, 9, 8, 6, 4 & 25 \\
12 & 3 & 6 & 16, 13, 12, 11, 7, 3 & 31 \\
13 & 3 & 9 & 16, 15, 14, 10, 9, 8, 6, 5, 3 & 43 \\
14 & 3 & 9 & 16, 15, 13, 11, 9, 7, 5, 4, 3 & 41 \\
15 & 3 & 12 & 16, 15, 14, 13, 12, 11, 10, 8, 7, 6, 4, 3 & 59 \\
16 & 4 & 6 & 16, 15, 11, 4, 2, 1 & 24 \\
17 & 4 & 6 & 15, 14, 9, 8, 6, 1 & 26 \\
18 & 4 & 8 & 18, 15, 13, 10, 8, 5, 3, 2 & 37 \\
19 & 4 & 8 & 18, 15, 13, 10, 8, 5, 3, 2 & 37 \\
20 & 4 & 8 & 18, 17, 16, 15, 14, 5, 2, 1 & 44 \\
\bottomrule
\end{tabular}


  \begin{plusloin}{}
    test
  \end{plusloin}

  \bibliographystyle{plain}
  \bibliography{poly}
\end{document}
