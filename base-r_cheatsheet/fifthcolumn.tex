% !TEX root = base-r.tex

{\setbeamercolor{block body}{fg = black, bg = white}
\begin{block}{Lists}
  \centering
  \inlc{l <- list(x = 1:5, y = c('a', 'b'))}\\{\small A list is a collection of elements which can be of different types.}
  
  \smallskip
  {\color{gray}\hrule}
  
  \smallskip
  \begin{tabular}{*{4}{>{\centering\arraybackslash}m{0.215\linewidth}}}
    \inlc{l[[2]]} & \inlc{l[1]} & \inlc{l\$x} & \inlc{l['y']}\\
    Second element of l. & New list with only the first element. & Element named x. & New list with only element named y.
  \end{tabular}
\end{block}
}

\begin{block}{Miscellaneous}
  \blocksubtitle{Arithmetics}
  \begin{tabular}{>{\centering\arraybackslash} m{0.5\linewidth} c}
    \multicolumn{2}{c}{\inlc{16 = 3*5 + 1}}\\
    \inlc{16 @\%/\%@ 3} & Quotient (result 5).\\
    \inlc{16 @\%\%@ 3} & Remainder (result 1).
  \end{tabular}
  
  \blocksubtitle{Permutations}
    \begin{tabular}{>{\small\centering}m{0.48\linewidth} >{\footnotesize\centering\arraybackslash}m{0.47\linewidth}}
      \inline{sample(x, size, replace = FALSE)} & Give a sample of the specified size from the elements of x.\\
      \multicolumn{2}{c}{\small\inlc{sample(c(1:5), 10, replace = TRUE)}}\\
      \inline{order(x, decreasing = FALSE)} &  Returns a permutation rearranging x.\\
      \multicolumn{2}{c}{\small\inlc{order(c(5, 1, 7, 3))} (Result 3 1 4 2)}\\
      \inline{sort(x, decreasing = FALSE)} & Sort a vector or factor.\\
       \multicolumn{2}{c}{\small\inlc{sort(c(5, 1, 7, 3))} (Result 1 3 5 7)}
    \end{tabular}
\end{block}

{\setbeamercolor{block body}{fg = black, bg = white}
\begin{block}{Memo}
  \renewcommand{\arraystretch}{1.5}\color{main}
  \begin{tabular}{>{\centering\arraybackslash} m{0.95\linewidth}}
    \\\hline\\\hline
    \\\hline\\\hline
    \\\hline\\\hline
    \\\hline\\\hline
    \\\hline%\\\hline
%    \\\hline\\\hline
  \end{tabular}
  \begin{tabular}{>{\centering\arraybackslash} m{2.08\linewidth}}
    \\\hline\\\hline
  \end{tabular}

  
\end{block}
}

%\begin{block}{Matrices}
%  \begin{column}[t]{0.5\linewidth}
%    \begin{tabular}{>{\centering\arraybackslash} m{0.65cm} c @{-~~} l}
%      \begin{tikzpicture}[scale = 0.3]
%        \foreach \x in {0, 1, 2}
%          \foreach \y in {0, 2}
%            \filldraw[lightgray!65] (\x, \y) rectangle ({\x + 0.8}, {\y + 0.8}); 
%        \foreach \x in {0, 1, 2}
%          \filldraw[yellow!75!orange] (\x, 1) rectangle ({\x + 0.8}, {1 + 0.8}); 
%      \end{tikzpicture}
%      & \inlc{m[2,  ]} & Select a row\\
%      \begin{tikzpicture}[scale = 0.3]
%        \foreach \y in {0, 1, 2}
%          \foreach \x in {1, 2}
%            \filldraw[lightgray!65] (\x, \y) rectangle ({\x + 0.8}, {\y + 0.8}); 
%        \foreach \y in {0, 1, 2}
%          \filldraw[yellow!75!orange] (0, \y) rectangle ({0 + 0.8}, {\y + 0.8}); 
%      \end{tikzpicture}
%      & \inlc{m[ ,1]} & Select a column\\
%      \begin{tikzpicture}[scale = 0.3]
%        \foreach \x in {0, 1, 2}
%          \foreach \y in {0, 1, 2}
%            \filldraw[lightgray!65] (\x, \y) rectangle ({\x + 0.8}, {\y + 0.8}); 
%        \filldraw[yellow!75!orange] (2, 1) rectangle ({2 + 0.8}, {1 + 0.8}); 
%      \end{tikzpicture}
%      & \inlc{m[2,3]} & Select an element
%    \end{tabular}
%  \end{column}          
%  {\color{gray}\vrule}          
%  \begin{column}[t]{0.35\linewidth}
%    \vspace{-10ex}
%    \flushright
%    \inl{t(m)}\\ Transpose
%    
%    \inl{m \%*\% n}\\{ Matrix Multiplication}
%    
%    \inl{solve(m,n)}\\{ Find \inl{x} in \inl{m*x = n}}
%  \end{column}
%\end{block}
%
%{\setbeamercolor{block body}{fg = black, bg = white}
%\begin{seeblock}{Data Frames}{dplyr}
%  {\fontsize{9pt}{11pt}\selectfont\inlc{df <- data.frame(x = 1:3, y = c('a', 'b', 'c'))}\\A special case of a list where all elements are the same length.}
%  
%  \begin{column}{0.35\linewidth}
%    
%    \renewcommand{\arraystretch}{1.75}
%    \color{secondary}
%    \begin{tabular}{| >{\centering\arraybackslash\color{black}} m{0.75cm} | >{\centering\arraybackslash\color{black}} m{0.75cm} |}
%      \hline
%      \rowcolor{codebg}\color{white} x & \color{white}y\\\hline
%      \rowcolor{white} 1 & a\\\hline
%      \rowcolor{white} 2 & b\\\hline
%      \rowcolor{white} 3 & c\\\hline
%    \end{tabular}
%  \end{column}
%%          
%  \begin{column}{0.65\linewidth}
%    \vspace{-3ex}
%    \begin{center}\textbf{List subsetting}\end{center}
%    \begin{tabular}{c >{\centering\arraybackslash} m{0.65cm} c m{0.65cm}}
%      \inlc{df\$x} & 
%      \begin{tikzpicture}[xscale = 0.4, yscale = 0.3]
%        \foreach \y in {0, 1, 2}
%          \filldraw[yellow!75!orange] (0, \y) rectangle ({0 + 0.8}, {\y + 0.8});
%        \filldraw[yellow!70!black] (0, 3) rectangle ({0 + 0.8}, {3 + 0.8}); 
%        \foreach \y in {0, 1, 2}
%          \filldraw[white] (1, \y) rectangle ({1 + 0.8}, {\y + 0.8});
%        \filldraw[codebg] (1, 3) rectangle ({1 + 0.8}, {3 + 0.8}); 
%      \end{tikzpicture} &
%      \inlc{df[[2]]} &
%      \begin{tikzpicture}[xscale = 0.4, yscale = 0.3]
%        \foreach \y in {0, 1, 2}
%          \filldraw[yellow!75!orange] (1, \y) rectangle ({1 + 0.8}, {\y + 0.8});
%        \filldraw[yellow!70!black] (1, 3) rectangle ({1 + 0.8}, {3 + 0.8}); 
%        \foreach \y in {0, 1, 2}
%          \filldraw[white] (0, \y) rectangle ({0 + 0.8}, {\y + 0.8});
%        \filldraw[codebg] (0, 3) rectangle ({0 + 0.8}, {3 + 0.8}); 
%      \end{tikzpicture}
%    \end{tabular}
%    \vspace{-1.5ex}
%    \begin{center}
%      \begin{tikzpicture}[remember picture, overlay]
%        \draw[darkgray, dotted, very thick, rounded corners] (-3, 0) rectangle (3, -1.5);
%      \end{tikzpicture}
%      
%      \textit{Understanding a data frame}
%      
%      \smallskip
%      \begin{tabular}{c l}
%        \inl{view(df)} & See the full data frame.\\
%        \inl{head(df)} & See the first 6 rows.
%      \end{tabular}
%    \end{center}
%  \end{column}
%  
%  \bigskip
%  \textbf{Matrix subsetting}
%  
%  \medskip
%  \begin{column}{0.33\linewidth}
%    \begin{tabular}{c >{\centering\arraybackslash} m{0.65cm}}
%      \inlc{df[ ,2]} &
%      \begin{tikzpicture}[xscale = 0.4, yscale = 0.3]
%        \foreach \y in {0, 1, 2}
%          \filldraw[yellow!70!orange] (1, \y) rectangle ({1 + 0.8}, {\y + 0.8});
%        \filldraw[yellow!75!black] (1, 3) rectangle ({1 + 0.8}, {3 + 0.8}); 
%        \foreach \y in {0, 1, 2}
%          \filldraw[white] (0, \y) rectangle ({0 + 0.8}, {\y + 0.8});
%        \filldraw[codebg] (0, 3) rectangle ({0 + 0.8}, {3 + 0.8}); 
%      \end{tikzpicture}\\[0.75ex]
%      \inlc{df[2,  ]} &
%      \begin{tikzpicture}[xscale = 0.4, yscale = 0.3]
%        \foreach \x in {0, 1} {
%          \filldraw[codebg] (\x, 3) rectangle ({\x + 0.8}, {3 + 0.8}); 
%          \filldraw[white] (\x, 0) rectangle ({\x + 0.8}, {0 + 0.8}); 
%          \filldraw[white] (\x, 2) rectangle ({\x + 0.8}, {2 + 0.8}); 
%          \filldraw[yellow!75!orange] (\x, 1) rectangle ({\x + 0.8}, {1 + 0.8}); 
%        }
%      \end{tikzpicture}\\[0.75ex]
%      \inlc{df[2,2]} &
%      \begin{tikzpicture}[xscale = 0.4, yscale = 0.3]
%        \foreach \x in {0, 1} {
%          \filldraw[codebg] (\x, 3) rectangle ({\x + 0.8}, {3 + 0.8}); 
%          \foreach \y in {0, 1, 2}
%            \filldraw[white] (\x, \y) rectangle ({\x + 0.8}, {\y + 0.8}); 
%        }
%        \filldraw[yellow!75!orange] (1, 1) rectangle ({1 + 0.8}, {1 + 0.8}); 
%      \end{tikzpicture}\\
%    \end{tabular}
%  \end{column}
%  {\color{gray}{\vrule}}
%  \begin{column}{0.24\linewidth}
%    \vspace{-13ex}
%    
%    \inl{nrow(df)}\\Number of rows.
%    
%    \inl{ncol(df)}\\Number of columns.
%    
%    \inl{dim(df)}\\Number of columns and rows.
%  \end{column}
%  {\color{gray}{\vrule}}
%  \begin{column}{0.4\linewidth}
%    \vspace{-12ex}
%    
%    \inl{cbind} - Bind columns.
%    
%    \smallskip
%    {\centering
%    \begin{tikzpicture}[xscale = 0.4, yscale = 0.3]
%      \foreach \x in {0, 1} {
%        \filldraw[codebg] (\x, 3) rectangle ({\x + 0.8}, {3 + 0.8}); 
%        \foreach \y in {0, 1, 2}
%          \filldraw[white] (\x, \y) rectangle ({\x + 0.8}, {\y + 0.8}); 
%      }
%      
%      \foreach \y in {0, 1, 2}
%        \filldraw[red] (2.5, \y) rectangle ({2.5 + 0.8}, {\y + 0.8}); 
%      \filldraw[red!70!black] (2.5, 3) rectangle ({2.5 + 0.8}, {3 + 0.8}); 
%      
%      \draw[->, > = stealth', ultra thick, main] (3.75, 1.5) -- (5.35, 1.5);
%      
%      \foreach \x in {6, 7, 8} {
%        \foreach \y in {0, 1, 2}
%          \filldraw[green] (\x, \y) rectangle ({\x + 0.8}, {\y + 0.8}); 
%        \filldraw[green!70!black] (\x, 3) rectangle ({\x + 0.8}, {3 + 0.8}); 
%      }
%    \end{tikzpicture}}
%    
%    \smallskip
%    \inl{rbind} - Bind rows.
%    
%    \smallskip
%    {\centering
%    \begin{tikzpicture}[xscale = 0.4, yscale = 0.3]
%      \foreach \x in {0, 1} {
%        \filldraw[codebg] (\x, 3) rectangle ({\x + 0.8}, {3 + 0.8}); 
%        \foreach \y in {0, 1, 2}
%          \filldraw[white] (\x, \y) rectangle ({\x + 0.8}, {\y + 0.8}); 
%      }
%      
%      \foreach \x in {2.5, 3.5} {
%        \filldraw[red] (\x, 1) rectangle ({\x + 0.8}, {1 + 0.8}); 
%        \filldraw[red!70!black] (\x, 2) rectangle ({\x + 0.8}, {2 + 0.8}); 
%      }
%      
%      \draw[->, > = stealth', ultra thick, main] (4.75, 1.5) -- (6.35, 1.5);
%      
%      \foreach \x in {7, 8} {
%        \foreach \y in {-0.5, 0.5, 1.5, 2.5}
%          \filldraw[green] (\x, \y) rectangle ({\x + 0.8}, {\y + 0.8}); 
%        \filldraw[green!70!black] (\x, 3.5) rectangle ({\x + 0.8}, {3.5 + 0.8}); 
%      }
%      
%    \end{tikzpicture}}
%        
%  \end{column}
%\end{seeblock}
%}